\chapter{Discussão e Validação Experimental}
\label{cap:discussao}

\section{Introdução}
\label{sec:discussao_intro}

Este capítulo apresenta uma discussão aprofundada das técnicas desenvolvidas no simulador de escalonamento, posicionando-as em relação ao estado da arte e validando experimentalmente suas contribuições. Enquanto o Capítulo~\ref{cap:revisao} estabeleceu o contexto teórico e identificou as lacunas da literatura, e o Capítulo~\ref{cap:resultados} apresentou os resultados experimentais obtidos, este capítulo dedica-se à análise técnica detalhada das estratégias implementadas e à validação das hipóteses de pesquisa.

A discussão está organizada em torno de três eixos principais: (i) a comparação das técnicas desenvolvidas com abordagens clássicas e contemporâneas, evidenciando suas contribuições originais; (ii) a descrição detalhada do pipeline híbrido de otimização, que integra métodos exatos, heurísticos e matheurísticos; e (iii) a validação experimental das contribuições científicas anunciadas no Capítulo~\ref{cap:revisao}, Seção~\ref{sec:posicionamento}.

Conforme discutido no Capítulo~\ref{cap:revisao}, Seção~\ref{sec:estado_arte}, a análise comparativa da literatura evidenciou que a integração profunda entre conformidade regulatória, granularidade temporal fina e arquitetura híbrida de otimização constitui uma lacuna relevante no estado da arte. Este capítulo demonstra como o simulador desenvolvido preenche essa lacuna por meio de técnicas específicas e validadas empiricamente.

\section{Comparação das Técnicas Desenvolvidas com a Literatura e o Estado da Arte}
\label{sec:comparacao_tecnicas}

Esta seção apresenta uma análise comparativa entre as técnicas empregadas no simulador de escalonamento desenvolvido neste trabalho e aquelas encontradas na literatura clássica e contemporânea. A comparação é realizada à luz dos resultados computacionais obtidos no Capítulo~\ref{cap:resultados}, permitindo avaliar não apenas diferenças conceituais, mas também impactos práticos em termos de eficiência, escalabilidade e estabilidade das soluções.

As contribuições introduzidas neste estudo incluem: (i) uma heurística construtiva matricial inédita baseada na distribuição temporal da demanda, (ii) um mecanismo sistemático de transformações elementares sobre a matriz de restrições, (iii) um método LNS guiado por densidade da matriz e (iv) um pipeline híbrido que integra heurística, LNS e MILP com realimentação estrutural.

Tais abordagens diferenciam-se significativamente do estado da arte por introduzirem mecanismos de pré-processamento estrutural e tomada de decisão orientada pela geometria interna das restrições, o que não é reportado nos trabalhos existentes. A comparação técnica detalhada, apresentada no Capítulo~\ref{cap:revisao}, Seção~\ref{sec:estado_arte}, evidencia que as técnicas propostas constituem contribuições originais ao campo do escalonamento regulado.

\subsection{Extensões Recentes do Simulador e Alinhamento com Tendências Atuais}

Além da formulação exata do escalonamento sob o Regulamento (CE) n.º 561/2006, o simulador foi estendido para incorporar uma arquitetura híbrida orientada a experimentos. Essa arquitetura agrega três camadas complementares: (i) uma heurística construtiva gulosa para geração rápida de soluções iniciais, (ii) um mecanismo matheurístico baseado em \textit{Large Neighborhood Search} (LNS) com reotimização local via solver inteiro e (iii) um módulo opcional de aprendizado de máquina supervisionado, empregado como camada de \textit{guidance} (não substitutiva) para decisões locais e estratégicas.

Do ponto de vista da literatura, tais extensões convergem com tendências contemporâneas em otimização combinatória aplicada, nas quais métodos exatos permanecem como referência de qualidade, enquanto heurísticas e matheurísticas são utilizadas para ampliar escalabilidade, reduzir tempo computacional e viabilizar aplicações interativas. Nesse sentido, o simulador não se limita a ``resolver um ILP'', mas estabelece uma plataforma de avaliação comparativa entre modos exato, heurístico e matheurístico, sob um conjunto de restrições legais complexas e altamente acopladas no tempo.

\subsection{Heurística Construtiva como Baseline Experimental}

A heurística gulosa (\textit{greedy initial allocation}) atua como gerador de soluções viáveis com baixo custo computacional, produzindo um \textit{baseline} essencial para: (i) comparação objetiva contra soluções ótimas, (ii) inicialização de métodos de melhoria e (iii) geração de dados para treinamento supervisionado. Em contraste com abordagens clássicas puramente baseadas em regras, a implementação proposta estrutura a decisão de alocação de forma matricial (período × motorista), permitindo calcular indicadores de carga local, carga por motorista e lacunas de demanda (\textit{demand gap}) como sinais fundamentais para guiar a construção da solução.

\subsection{Large Neighborhood Search com Reotimização via Solver Inteiro}

O método LNS implementado utiliza o princípio de destruição e reconstrução parcial da solução. Em cada iteração, um subconjunto de períodos é liberado (vizinhança) e as demais alocações são fixadas (\textit{fixed assignments}); em seguida, um subproblema restrito é reotimizado via solver inteiro. Essa abordagem combina duas propriedades desejáveis: (i) capacidade de exploração (metaheurística) e (ii) consistência com o modelo exato (reotimização com as mesmas restrições legais). Como consequência, obtém-se um mecanismo de melhoria incremental com controle explícito de esforço computacional (número de iterações, tamanho da vizinhança e limites do solver).

\subsection{Aprendizado de Máquina como Camada Opcional de Guidance}

O módulo de aprendizado de máquina foi concebido como camada opcional, com comportamento robusto por \textit{fallback}. Dois modelos supervisionados são utilizados: $f_1$, para estimar a atratividade de uma decisão motorista–período no contexto da heurística gulosa, e $f_2$, para estimar o potencial de melhoria de uma vizinhança candidata no LNS. A integração é não-intrusiva: na ausência do ambiente de ML ou de modelos treinados, o simulador adota automaticamente escores heurísticos determinísticos, preservando reprodutibilidade e estabilidade experimental.

A contribuição metodológica central desse componente está na construção de um pipeline endógeno de dados: o próprio simulador gera \textit{datasets} sintéticos consistentes com o domínio, por meio da comparação entre soluções heurísticas e soluções ótimas (ou de alta qualidade) retornadas pelo solver inteiro. Essa estratégia evita dependência de dados externos e permite controlar distribuição de instâncias, sementes aleatórias e parâmetros regulatórios, oferecendo um ambiente experimental replicável e extensível.

\subsection{Visualização Analítica e Instrumentação Experimental}

Um diferencial relevante do simulador, quando comparado a implementações tipicamente descritas na literatura, é a instrumentação de métricas e gráficos orientados a diagnóstico. Além de indicadores operacionais (cobertura, subutilização e sobrecarga), o sistema incorpora visualizações de estrutura matricial (densidade e padrão de esparsidade), gráficos de convergência do LNS (histórico da função objetivo e marcação de melhorias) e gráficos auxiliares para análise temporal. Essas visualizações transformam o simulador em ferramenta de pesquisa aplicada, viabilizando estudos de sensibilidade, avaliação de \textit{trade-offs} (qualidade versus tempo), e auditoria do comportamento dos algoritmos sob diferentes configurações de restrições.

Com base nas comparações apresentadas, torna-se possível posicionar o simulador desenvolvido nesta dissertação não apenas como uma implementação específica de um modelo de Programação Linear Inteira, mas como uma arquitetura híbrida orientada à experimentação e à investigação científica.

\section{Pipeline Híbrido de Otimização}
\label{sec:pipeline_hibrido}

O simulador desenvolvido nesta dissertação adota uma arquitetura híbrida de otimização, na qual diferentes paradigmas algorítmicos são integrados de forma hierárquica e complementar. Essa abordagem foi concebida para conciliar rigor regulatório, eficiência computacional e flexibilidade operacional, características essenciais em problemas reais de escalonamento de motoristas sob o Regulamento (CE) n.º 561/2006.

O pipeline híbrido proposto combina quatro camadas principais: (i) uma heurística construtiva para geração de soluções iniciais factíveis, (ii) uma análise estrutural baseada na densidade da matriz de restrições, (iii) um processo iterativo de melhoria via \textit{Large Neighborhood Search} (LNS) com reotimização local por MILP e (iv) uma camada opcional de aprendizagem de máquina para apoio à decisão heurística. A Figura~\ref{fig:pipeline-simulador-hibrido} ilustra esse fluxo de forma integrada.

\subsection{Heurística Construtiva e Geração da Solução Inicial}
\label{sec:heuristica_construtiva}

O pipeline inicia-se com a leitura dos dados de entrada, compostos por demanda operacional real por período, parâmetros regulatórios e configurações do cenário. A partir dessas informações, uma heurística construtiva gulosa é aplicada para gerar uma escala inicial factível.

Essa heurística opera de forma sequencial ao longo da linha do tempo discretizada, atribuindo motoristas elegíveis aos períodos conforme a demanda, respeitando as principais restrições regulatórias, como limites de condução contínua, pausas obrigatórias e descansos mínimos. O objetivo desta etapa não é alcançar a solução ótima, mas sim produzir rapidamente uma solução viável que sirva como ponto de partida para etapas posteriores de refinamento.

\subsection{Atualização de Estado e Consistência Regulamentar}
\label{sec:atualizacao_estado}

Após a alocação inicial, o simulador realiza uma atualização completa do estado regulatório de cada motorista, incluindo carga de trabalho acumulada, condução contínua, pausas e janelas de descanso diário, semanal e quinzenal. Essa etapa é fundamental para garantir que qualquer modificação subsequente na solução preserve a consistência com o histórico temporal, evitando violações indiretas das regras legais.

\subsection{Análise Estrutural da Matriz de Restrições}
\label{sec:analise_estrutural}

Com a solução inicial estabelecida, o modelo procede à análise estrutural da matriz de restrições associada ao problema. Em particular, é avaliada a densidade da matriz, indicador que reflete o grau de acoplamento entre variáveis e restrições.

A densidade é utilizada como critério decisório: se o modelo apresenta uma estrutura suficientemente esparsa e estável, a solução heurística pode ser considerada adequada, sendo aceita diretamente. Caso contrário, o pipeline avança para a fase de melhoria iterativa via LNS.

\subsection{Large Neighborhood Search com Reotimização MILP}
\label{sec:lns_milp}

Na etapa de \textit{Large Neighborhood Search}, a solução corrente é parcialmente destruída, liberando subconjuntos específicos de decisões, como janelas temporais críticas ou grupos de motoristas com maior carga acumulada. O restante da solução é mantido fixo, preservando a consistência global.

Para cada vizinhança liberada, é formulado um subproblema de Programação Linear Inteira, resolvido exatamente pelo solver CP-SAT. Essa estratégia permite explorar regiões promissoras do espaço de soluções com alto grau de rigor, sem o custo computacional de reotimizar o problema completo.

Após a resolução do subproblema, a solução obtida é avaliada. Caso represente uma melhoria em relação à solução atual, ela é incorporada ao estado global; caso contrário, o processo retorna à etapa de LNS, selecionando uma nova vizinhança.

\subsection{Camada Opcional de Aprendizagem de Máquina}
\label{sec:camada_ml}

De forma complementar, o pipeline incorpora uma camada opcional de aprendizagem de máquina. Modelos supervisionados podem ser utilizados para atribuir escores a pares motorista–período, auxiliando a heurística construtiva, bem como para priorizar vizinhanças com maior probabilidade de gerar melhorias no LNS.

É importante destacar que essa camada não substitui o modelo matemático, nem altera diretamente restrições ou a função objetivo. Seu papel é exclusivamente o de orientar decisões heurísticas, mantendo o caráter determinístico e reproduzível do simulador.

\subsection{Critério de Parada e Solução Final}
\label{sec:criterio_parada}

O processo iterativo prossegue até que não sejam observadas melhorias adicionais, ou até que limites de iteração ou tempo computacional sejam atingidos. A solução final resultante é, então, submetida a uma validação completa das restrições legais, garantindo conformidade com o Regulamento (CE) n.º 561/2006.

Esse pipeline híbrido permite explorar eficientemente o espaço de soluções, combinando rapidez inicial, refinamento iterativo e rigor matemático, alinhando-se plenamente às necessidades operacionais e científicas do escalonamento de motoristas em contextos reais.

\begin{figure}[H]
    \centering
    \includegraphics[width=0.92\textwidth]{template/figuras/fluxogeral.png}
    \caption{Pipeline do simulador híbrido proposto: geração de solução inicial por heurística (com ML opcional $f_1$), melhoria iterativa por LNS (com seleção de vizinhanças opcional $f_2$), reconstrução local via CP-SAT/MILP com fixação parcial e validação final de conformidade com o Regulamento (CE) n.º 561/2006.}
    \label{fig:pipeline-simulador-hibrido}
\end{figure}

\section{Validação das Hipóteses de Pesquisa}
\label{sec:validacao_hipoteses}

As contribuições científicas anunciadas no Capítulo~\ref{cap:revisao}, Seção~\ref{sec:posicionamento}, foram sistematicamente validadas por meio dos experimentos apresentados no Capítulo~\ref{cap:resultados}. Esta seção consolida essa validação, estabelecendo conexões diretas entre as hipóteses teóricas, os resultados empíricos e as evidências experimentais obtidas.

\subsection{Validação da Contribuição 1: Modelagem Matemática}

A primeira contribuição científica proposta refere-se à formulação matemática completa e detalhada do escalonamento sob o Regulamento (CE) n.º 561/2006, com granularidade temporal de 15 minutos. Os experimentos demonstraram que o modelo alcançou 100\% de conformidade legal em todos os cenários testados, confirmando que a discretização temporal adotada permite representar adequadamente janelas móveis, dependências acumulativas e restrições de pausas obrigatórias.

\subsection{Validação da Contribuição 2: Heurística e Matheurísticas}

A heurística construtiva matricial e o método LNS guiado por densidade foram validados experimentalmente. O modo heurístico demonstrou capacidade de gerar soluções viáveis em tempos inferiores a 1 segundo, enquanto o método LNS apresentou reduções de até 40\% no tempo computacional em comparação com o método exato, mantendo qualidade de solução próxima ao ótimo.

\subsection{Validação da Contribuição 3: Análise Estrutural}

A aplicação de operações elementares sobre a matriz de restrições permitiu análises de densidade e esparsidade que auxiliaram na compreensão do comportamento do solver. A densidade média observada situou-se entre 15\% e 35\%, confirmando a esparsidade estrutural do modelo e sua adequação para resolução via CP-SAT.

\subsection{Validação da Contribuição 4: Plataforma Experimental}

A plataforma híbrida desenvolvida demonstrou capacidade de operar em três modos de resolução distintos (exato, heurístico e LNS), permitindo comparações sistemáticas e reprodutíveis. Os indicadores apresentados no Capítulo~\ref{cap:resultados} confirmam a robustez e a aplicabilidade da arquitetura proposta.

\subsection{Validação da Contribuição 5: Viabilidade Prática}

Os tempos de resolução inferiores a 2 segundos para cenários de 24 horas confirmam a viabilidade de aplicações em contextos reais de replanejamento dinâmico e sistemas de apoio à decisão operacional. Essa evidência valida a hipótese de que modelos PLI com granularidade fina podem ser computacionalmente viáveis quando adequadamente implementados.

\section{Síntese da Discussão}
\label{sec:sintese_discussao}

Em síntese, a revisão comparativa evidencia que a principal contribuição desta dissertação reside na integração profunda entre conformidade regulatória, modelagem temporal de alta granularidade e uma arquitetura híbrida de otimização, aspectos que não aparecem de forma combinada nos trabalhos revisados. Esse posicionamento consolida o simulador como uma contribuição original ao estado da arte em escalonamento de motoristas sob restrições legais complexas.

As validações apresentadas neste capítulo confirmam que as contribuições teóricas anunciadas no Capítulo~\ref{cap:revisao} foram empiricamente demonstradas por meio dos experimentos do Capítulo~\ref{cap:resultados}, estabelecendo uma cadeia completa de fundamentação, desenvolvimento, experimentação e validação científica.
