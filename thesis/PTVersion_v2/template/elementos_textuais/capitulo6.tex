\chapter{Implementação Computacional}\label{capitulo5}
\section{Introdução}

Este capítulo descreve a implementação computacional do modelo matemático
apresentado no Capítulo~4, bem como o desenvolvimento de um simulador
interativo concebido para apoiar a análise, validação e experimentação
de estratégias de escalonamento de motoristas sob o Regulamento
(CE) n.{\textordmasculine}~561/2006.

A implementação não se limita à simples resolução de um modelo de
Programação Linear Inteira (PLI), mas materializa uma plataforma
experimental híbrida, capaz de integrar métodos exatos, heurísticos,
matheurísticos e, de forma opcional, mecanismos de aprendizado de máquina.
Essa abordagem permite investigar, de maneira controlada e reprodutível,
o impacto de diferentes estratégias de resolução, funções objetivo e
configurações de restrições legais sobre a qualidade das soluções obtidas
e o esforço computacional requerido.

O núcleo normativo do simulador é constituído pelo modelo de PLI descrito
no capítulo anterior, resolvido por meio do solver CP-SAT do OR-Tools.
Esse núcleo garante que todas as soluções finais produzidas pelo sistema
sejam matematicamente consistentes e juridicamente válidas, independentemente
do método de resolução empregado. Heurísticas construtivas, estratégias
de Large Neighborhood Search (LNS) e modelos de aprendizado supervisionado
são integrados exclusivamente como camadas auxiliares, atuando na geração
de soluções iniciais, na exploração de vizinhanças promissoras ou na
priorização de decisões locais, sem alterar as restrições legais ou a
formulação matemática do problema.

A arquitetura computacional foi projetada de forma modular, organizada
em camadas bem definidas: interface de parametrização, modelagem,
resolução e pós-processamento. Essa separação favorece a extensibilidade
do sistema, permitindo a inclusão de novos métodos de resolução,
indicadores analíticos ou fontes de dados, além de facilitar a manutenção
e a reprodutibilidade dos experimentos. Do ponto de vista metodológico,
o simulador atua como um elo entre a formulação teórica e a validação
empírica, assegurando consistência entre modelo matemático, implementação
computacional e análise de resultados.

Os experimentos apresentados nesta dissertação utilizam dados de demanda
oriundos de operações reais de transporte rodoviário, previamente tratados
e anonimizados, de modo a preservar confidencialidade operacional e
conformidade ética. A granularidade temporal adotada é de 15 minutos,
refletindo a frequência real com que pedidos de carga são recebidos e
registrados na operação analisada. Embora o simulador ofereça mecanismos
para geração de dados sintéticos, tais recursos são empregados apenas para
fins de teste, validação metodológica e treinamento de modelos auxiliares,
não sendo utilizados na análise principal dos resultados.

Além de seu uso como ferramenta de pesquisa, o simulador foi concebido
com caráter educacional e exploratório, permitindo ao usuário ativar ou
desativar individualmente restrições legais, comparar estratégias de
otimização e visualizar indicadores operacionais de forma interativa.
Essa flexibilidade amplia o escopo do trabalho, tornando a plataforma
adequada tanto para investigações científicas quanto para análises
experimentais de apoio à tomada de decisão em contextos reais de
planejamento de transporte.

Este capítulo apresenta, portanto, a arquitetura, os principais módulos
e os fluxos computacionais do simulador desenvolvido, estabelecendo a
base operacional necessária para os experimentos e análises apresentados
no Capítulo~6.

\section{Arquitetura Geral da Solução}

A solução computacional desenvolvida neste trabalho foi concebida segundo
uma arquitetura modular em camadas, cujo objetivo é garantir clareza
conceitual, extensibilidade e reprodutibilidade dos experimentos.
Essa arquitetura permite que diferentes estratégias de resolução e
otimização coexistam sob um mesmo arcabouço matemático e legal, mantendo
consistência entre modelagem, implementação e análise empírica.

De forma geral, o simulador é estruturado em quatro módulos principais,
conforme ilustrado ao longo deste capítulo:

\begin{enumerate}
    \item \textbf{Interface de Parametrização};
    \item \textbf{Camada de Modelagem Matemática};
    \item \textbf{Camada de Resolução};
    \item \textbf{Pós-processamento e Análise de Resultados}.
\end{enumerate}

Cada módulo desempenha um papel específico no fluxo computacional,
permitindo separar decisões de entrada, formulação do problema,
estratégias de resolução e avaliação das soluções obtidas.

\subsection{Interface de Parametrização}

A interface de parametrização, desenvolvida em \textit{Streamlit},
constitui o ponto de entrada do simulador. Por meio dessa interface,
o usuário pode configurar de forma interativa todos os elementos
relevantes do problema, incluindo:

\begin{itemize}
    \item horizonte de planejamento e granularidade temporal;
    \item perfil de demanda operacional por período;
    \item número máximo de motoristas disponíveis;
    \item tipo de variáveis (binárias, inteiras ou relaxadas);
    \item função objetivo a ser utilizada;
    \item método de resolução (exato, heurístico ou LNS);
    \item ativação ou desativação individual das restrições legais.
\end{itemize}

Essa flexibilidade permite a construção de múltiplos cenários
experimentais a partir de um mesmo núcleo matemático, favorecendo
análises de sensibilidade e comparações controladas entre estratégias.
Além disso, a interface atua como um mecanismo de documentação implícita
dos experimentos, pois cada execução é totalmente determinada pelo
conjunto de parâmetros selecionados.

\subsection{Camada de Modelagem Matemática}

A camada de modelagem é responsável por instanciar o modelo de
Programação Linear Inteira descrito no Capítulo~4. Nessa etapa,
são construídos:

\begin{itemize}
    \item os conjuntos e índices do problema;
    \item as variáveis de decisão;
    \item a função objetivo selecionada;
    \item o conjunto de restrições legais e operacionais ativas.
\end{itemize}

Um aspecto central da arquitetura é que a ativação das restrições
do Regulamento (CE) n.{\textordmasculine}~561/2006 é realizada de forma
parametrizada. Cada regra legal pode ser habilitada ou desabilitada
individualmente, sem necessidade de alterar o código-fonte da modelagem.
Essa abordagem permite estudar o impacto isolado de cada restrição,
além de possibilitar usos educacionais e experimentais do simulador.

Independentemente da estratégia de resolução adotada posteriormente,
toda solução final permanece ancorada nessa camada de modelagem,
assegurando validade jurídica e rigor matemático.

\subsection{Camada de Resolução}

A camada de resolução define como o problema modelado será efetivamente
resolvido. No simulador, distingue-se explicitamente o \textit{modo de
resolução} do \textit{modo de otimização}. O primeiro refere-se à
estratégia algorítmica empregada, enquanto o segundo define o critério
objetivo utilizado pelo modelo matemático.

Três modos de resolução são suportados:

\begin{itemize}
    \item \textbf{Modo exato}: o modelo completo de PLI é resolvido
    diretamente pelo solver CP-SAT, buscando soluções ótimas dentro
    dos limites de tempo e tolerância definidos;
    
    \item \textbf{Modo heurístico}: uma heurística construtiva gulosa
    gera rapidamente uma solução viável inicial, priorizando o
    atendimento da demanda com baixo custo computacional;
    
    \item \textbf{Modo LNS}: uma abordagem matheurística baseada em
    \textit{Large Neighborhood Search}, na qual partes da solução
    corrente são iterativamente liberadas e reotimizadas via solver
    inteiro, permitindo melhorias progressivas.
\end{itemize}

Nos modos heurístico e LNS, o solver CP-SAT continua desempenhando papel
central, seja validando soluções iniciais, seja reconstruindo
subproblemas restritos. Dessa forma, mesmo estratégias aproximadas
permanecem alinhadas ao modelo exato e às restrições legais.

\subsection{Pós-processamento e Análise de Resultados}

Após a obtenção da solução final, o simulador executa uma etapa de
pós-processamento responsável pela extração de indicadores operacionais,
estatísticas globais e métricas temporais. Entre os principais resultados
gerados destacam-se:

\begin{itemize}
    \item cobertura da demanda por período;
    \item número total de motoristas ativos;
    \item esforço operacional agregado;
    \item padrões de sobrecarga e subutilização;
    \item estabilidade temporal da alocação;
    \item métricas de desempenho computacional.
\end{itemize}

Esses indicadores são apresentados ao usuário por meio de tabelas e
visualizações gráficas, permitindo uma interpretação clara e objetiva
do comportamento da solução ao longo do horizonte de planejamento.

\subsection{Considerações Arquiteturais}

A arquitetura adotada permite que o simulador atue simultaneamente como:

\begin{itemize}
    \item ferramenta de validação do modelo matemático;
    \item ambiente experimental para comparação de métodos;
    \item plataforma educacional para estudo de restrições legais;
    \item base para futuras extensões industriais ou acadêmicas.
\end{itemize}

Ao separar claramente modelagem, resolução e análise, o sistema garante
transparência metodológica e facilita a evolução do trabalho, mantendo
coerência entre teoria, implementação e experimentação.

\section{Interface de Parametrização}

A interface de parametrização constitui o principal meio de interação entre
o usuário e o simulador desenvolvido neste trabalho. Implementada em
\textit{Streamlit}, ela permite configurar de forma intuitiva e controlada
todos os elementos relevantes do problema de escalonamento, garantindo que
cada experimento seja completamente definido por um conjunto explícito de
parâmetros de entrada.

Do ponto de vista conceitual, a interface não atua apenas como um mecanismo
de entrada de dados, mas como uma camada de controle experimental, assegurando
reprodutibilidade, transparência e consistência entre os cenários analisados.

\subsection{Configuração do Horizonte Temporal}

O primeiro grupo de parâmetros define o horizonte de planejamento e a
granularidade temporal do problema. O simulador permite trabalhar com
horizontes de diferentes durações, incluindo:

\begin{itemize}
    \item cenários de 24 horas;
    \item cenários multi-diários (7, 15 ou 30 dias);
    \item experimentos de curta duração para validação incremental.
\end{itemize}

A granularidade temporal adotada neste trabalho é de 15 minutos, resultando
em 96 períodos por dia. Essa escolha reflete diretamente a frequência real
com que pedidos de carga chegam ao sistema operacional da empresa analisada,
sendo cada solicitação registrada e tratada dentro do respectivo intervalo
temporal.

Essa discretização fina permite representar com maior fidelidade a dinâmica
operacional do transporte rodoviário, além de viabilizar a correta aplicação
das restrições temporais previstas na legislação europeia.

\subsection{Inserção da Demanda Operacional}

A interface permite ao usuário definir a demanda operacional por período de
três formas distintas:

\begin{itemize}
    \item inserção manual direta;
    \item carregamento a partir de arquivos externos;
    \item geração sintética controlada.
\end{itemize}

Nos experimentos apresentados nesta dissertação, a demanda utilizada é
oriunda de dados reais da operação da empresa, previamente tratados e
anonimizados. Esses dados representam a quantidade mínima de motoristas
necessária em cada período de 15 minutos, refletindo a variabilidade
operacional observada na prática.

A possibilidade de geração sintética de demanda permanece disponível no
simulador exclusivamente para fins de teste, validação metodológica e
experimentação controlada, não sendo utilizada na análise principal dos
resultados apresentados neste trabalho.

\subsection{Seleção do Modo de Resolução}

A interface permite selecionar explicitamente o modo de resolução a ser
utilizado, distinguindo três estratégias algorítmicas:

\begin{itemize}
    \item \textbf{Solução exata}: resolução completa do modelo de Programação
    Linear Inteira via solver CP-SAT;
    \item \textbf{Heurística construtiva}: geração rápida de uma solução viável
    por meio de uma estratégia gulosa;
    \item \textbf{LNS (Large Neighborhood Search)}: abordagem matheurística
    baseada em destruição e reconstrução parcial da solução.
\end{itemize}

Essa escolha permite ao usuário analisar compromissos entre qualidade da
solução e esforço computacional, além de viabilizar comparações diretas entre
métodos sob um mesmo conjunto de parâmetros.

\subsection{Seleção da Função Objetivo}

De forma independente do modo de resolução, o usuário pode selecionar o
critério de otimização adotado pelo modelo matemático. Atualmente, o
simulador suporta:

\begin{itemize}
    % \item maximização da resposta à demanda;
    \item minimização do número total de motoristas ativos.
\end{itemize}

Essa separação entre modo de resolução e função objetivo permite maior
flexibilidade experimental, pois o mesmo problema pode ser resolvido com
diferentes estratégias algorítmicas e diferentes objetivos operacionais,
sem alteração da estrutura básica do modelo.

\subsection{Ativação das Restrições Legais}

Um dos diferenciais da interface desenvolvida é a possibilidade de ativar ou
desativar individualmente as restrições associadas ao Regulamento (CE)
n.{\textordmasculine}~561/2006. Entre as restrições configuráveis destacam-se:

\begin{itemize}
    \item limites diários, semanais e quinzenais de condução;
    \item pausas obrigatórias após condução contínua;
    \item descanso diário normal ou reduzido;
    \item descanso semanal.
\end{itemize}

Essa funcionalidade permite realizar análises de sensibilidade, simular
cenários hipotéticos e avaliar o impacto isolado de cada regra legal sobre
a solução final. Importante ressaltar que, independentemente da configuração
escolhida, o modelo matemático mantém consistência interna, aplicando apenas
as restrições explicitamente ativadas.

\subsection{Parâmetros Computacionais e Controle do Solver}

A interface também expõe parâmetros computacionais relevantes, como:

\begin{itemize}
    \item limite máximo de tempo de execução;
    \item tolerância de \textit{gap} relativo;
    \item número máximo de motoristas disponíveis;
    \item parâmetros de controle do processo LNS.
\end{itemize}

Esses parâmetros permitem adaptar o comportamento do simulador a diferentes
níveis de complexidade e escala, tornando-o adequado tanto para experimentos
rápidos quanto para análises mais aprofundadas.

\subsection{Reprodutibilidade e Consistência Experimental}

Todos os parâmetros selecionados na interface determinam completamente a
instância do problema resolvida pelo simulador. Dessa forma, para um conjunto
fixo de parâmetros de entrada, o comportamento do sistema é determinístico,
assegurando reprodutibilidade dos experimentos realizados ao longo deste
trabalho.

Essa característica é fundamental para garantir consistência entre os
resultados apresentados nos capítulos experimentais e para permitir a
comparação rigorosa entre diferentes métodos de resolução.

\section{Configuração das Restrições Legais}

Um dos principais diferenciais do simulador desenvolvido neste trabalho é a
possibilidade de ativar ou desativar individualmente as restrições legais
associadas ao Regulamento (CE) n.{\textordmasculine}~561/2006. Essa
configurabilidade permite explorar de forma controlada o impacto de cada
regra sobre o escalonamento dos motoristas, sem comprometer a consistência
matemática do modelo.

Do ponto de vista metodológico, todas as restrições legais são formuladas de
forma independente no modelo matemático. A ativação ou desativação de uma
restrição corresponde, na prática, à inclusão ou exclusão explícita do
respectivo conjunto de inequações no problema de otimização.

\subsection{Motivação para a Ativação Seletiva}

A possibilidade de ativar seletivamente as restrições legais atende a
diferentes objetivos experimentais, entre os quais destacam-se:

\begin{itemize}
    \item análise de sensibilidade em relação a regras específicas;
    \item avaliação do impacto isolado de cada restrição sobre a solução;
    \item simulação de cenários educacionais e exploratórios;
    \item comparação entre soluções parcialmente reguladas e plenamente
    conformes à legislação.
\end{itemize}

Essa abordagem é particularmente relevante em problemas regulados, nos quais
o efeito combinado de múltiplas regras pode mascarar a contribuição
individual de cada uma para a complexidade e o custo da solução.

\subsection{Conjunto de Restrições Implementadas}

O simulador permite configurar, de forma independente, as principais regras
previstas no Regulamento (CE) n.{\textordmasculine}~561/2006, incluindo:

\begin{itemize}
    \item limites diários de condução (9h e extensões até 10h);
    \item limites semanais de condução (56h);
    \item limites quinzenais de condução (90h em 14 dias);
    \item pausas obrigatórias após 4h30 de condução contínua;
    \item descanso diário normal (11h) e reduzido (9h);
    \item descanso semanal normal e reduzido;
    \item restrições sobre número máximo de extensões diárias por semana.
\end{itemize}

Cada uma dessas regras pode ser ativada ou desativada diretamente na interface,
permitindo compor diferentes conjuntos regulatórios conforme o cenário de
análise.

\subsection{Interpretação Operacional da Configuração}

Quando uma restrição está ativada, o modelo matemático impõe integralmente as
condições associadas, garantindo conformidade legal plena em relação àquela
regra específica. Quando desativada, o modelo simplesmente não inclui as
inequações correspondentes, permitindo avaliar o comportamento do sistema
na ausência daquela limitação.

É importante destacar que a desativação de uma restrição não implica
violação implícita da legislação na solução final, mas sim uma decisão
experimental consciente, utilizada exclusivamente para fins de análise,
validação e comparação metodológica.

\subsection{Consistência do Modelo sob Diferentes Configurações}

Independentemente da combinação de restrições selecionada, o modelo mantém
consistência interna e viabilidade matemática. Todas as restrições foram
formuladas de forma modular, evitando dependências implícitas que poderiam
tornar o modelo inválido em determinadas configurações.

Essa característica garante que o simulador possa ser utilizado tanto para
estudos regulatórios rigorosos quanto para investigações exploratórias,
sem necessidade de reformulação estrutural do modelo matemático.

\subsection{Relação com os Experimentos Realizados}

Nos experimentos apresentados nesta dissertação, as restrições legais foram
ativadas de acordo com o cenário analisado. Em particular, os resultados
principais consideram a ativação completa do conjunto de regras previstas no
Regulamento (CE) n.{\textordmasculine}~561/2006, garantindo plena conformidade
legal das soluções obtidas.

Cenários com restrições parcialmente ativadas foram utilizados de forma
complementar para análise de sensibilidade e validação do comportamento do
modelo, não sendo apresentados em detalhe nos resultados principais, a fim
de evitar a proliferação excessiva de artefatos gráficos.

\section{Construção da Matriz de Restrições}

A matriz de restrições constitui o núcleo estrutural da implementação
computacional do modelo de escalonamento. Ela representa, de forma explícita
e organizada, todas as restrições legais, operacionais e temporais que
governam o problema, permitindo sua resolução por meio de técnicas de
Programação Linear Inteira e Programação por Restrições.

Do ponto de vista matemático, cada linha da matriz corresponde a uma
restrição do modelo, enquanto cada coluna está associada a uma variável de
decisão. A construção dessa matriz reflete diretamente a formulação
apresentada no Capítulo~4, garantindo consistência entre teoria e
implementação.

\subsection{Estrutura e Organização da Matriz}

A matriz de restrições é construída de forma incremental a partir dos
seguintes blocos principais:

\begin{itemize}
    \item restrições de atendimento da demanda;
    \item restrições de ativação e presença dos motoristas;
    \item limites diários, semanais e quinzenais de condução;
    \item restrições de pausas obrigatórias;
    \item restrições de descanso diário e semanal;
    \item vínculos de continuidade temporal.
\end{itemize}

Cada conjunto de restrições é inserido de maneira modular, permitindo que
seja ativado ou desativado conforme a configuração definida pelo usuário na
interface. Essa abordagem garante flexibilidade experimental sem comprometer
a integridade do modelo.

\subsection{Discretização Temporal e Encadeamento}

A discretização do horizonte em períodos de 15 minutos resulta em uma forte
estrutura temporal na matriz, caracterizada por padrões repetitivos e
dependências locais entre períodos consecutivos. Esse encadeamento se
manifesta, por exemplo, nas restrições de continuidade de jornada, acúmulo
de condução e reinício após períodos de descanso.

Como consequência, a matriz apresenta uma estrutura quase diagonal, com
coeficientes não nulos concentrados em regiões específicas, o que é altamente
favorável para solvers baseados em busca combinatória e propagação de
restrições, como o CP-SAT.

\subsection{Esparsidade da Matriz}

Uma característica relevante da matriz construída é seu elevado grau de
esparsidade. Mesmo para cenários de múltiplos dias, a proporção de elementos
não nulos situa-se tipicamente entre 15\% e 35\%, dependendo do conjunto de
restrições ativadas e do horizonte temporal considerado.

Essa esparsidade reduz significativamente o custo computacional associado à
propagação de restrições e à exploração do espaço de soluções, contribuindo
para tempos de resolução reduzidos mesmo em instâncias de maior escala.

\subsection{Operações Elementares e Análise Estrutural}

Além da construção direta da matriz de restrições, o simulador permite a
aplicação de operações elementares, tais como:

\begin{itemize}
    \item troca de linhas;
    \item multiplicação de linhas por constantes;
    \item combinação linear controlada de restrições.
\end{itemize}

Essas operações não alteram o conjunto de soluções viáveis do problema, mas
podem ser utilizadas para fins educacionais, análise estrutural e diagnóstico
do comportamento numérico do modelo. Em particular, elas permitem estudar o
impacto da organização da matriz sobre sua densidade e sobre o desempenho do
solver.

\subsection{Versões da Matriz e Transformações}

Durante o processo de modelagem, podem ser observadas diferentes versões da
matriz de restrições, desde uma formulação inicial até versões finais após
ajustes estruturais e operações elementares. Essas transformações preservam
rigorosamente a equivalência matemática do problema, atuando apenas sobre sua
representação algébrica.

Essa distinção entre formulação conceitual e representação matricial é
importante para compreender como escolhas de implementação influenciam o
desempenho computacional, sem modificar o significado do modelo.

\subsection{Relação com o Desempenho Computacional}

A forma como a matriz de restrições é construída e organizada tem impacto
direto sobre o desempenho do solver. A combinação de esparsidade, estrutura
temporal e modularidade favorece a atuação do CP-SAT, permitindo resolver
instâncias com milhares de variáveis e restrições em tempos compatíveis com
uso prático.

Os resultados experimentais apresentados no Capítulo~6 evidenciam essa
eficiência, confirmando que a estrutura matricial adotada é adequada para o
problema de escalonamento de motoristas sob regulamentação europeia.

\section{Execução do Solver}

 Após a definição completa do modelo matemático e da matriz de restrições, o
 problema de escalonamento é submetido ao processo de resolução por meio do
 solver CP-SAT, integrante da biblioteca OR-Tools. Embora o modelo seja
 formulado como um problema de Programação Linear Inteira, o CP-SAT adota uma
 abordagem híbrida, combinando técnicas de programação por restrições,
 satisfatibilidade booleana e otimização inteira.

 Essa escolha permite explorar de forma eficiente a forte estrutura temporal e
 a esparsidade do modelo, superando limitações frequentemente observadas em
 solvers MILP tradicionais quando aplicados a problemas regulados de grande
 escala.

 \subsection{Inicialização e Parâmetros do Solver}

 O simulador inicializa o solver com um conjunto explícito de parâmetros
 computacionais, que podem ser configurados diretamente pela interface. Entre
 os principais parâmetros destacam-se:

 \begin{itemize}
     \item limite máximo de tempo de execução;
     \item tolerância de \textit{gap} relativo entre solução atual e melhor
     limite inferior;
     \item nível de verbosidade dos logs;
     \item limites associados ao processo iterativo do LNS.
 \end{itemize}

 Esses parâmetros permitem adaptar o comportamento do solver ao tamanho do
 problema e aos objetivos do experimento, possibilitando desde execuções rápidas
 até análises mais aprofundadas.

 \subsection{Critérios de Parada}

 A execução do solver é encerrada quando pelo menos um dos seguintes critérios
 é atendido:

 \begin{itemize}
     \item obtenção de solução ótima comprovada;
     \item atingimento do limite máximo de tempo;
     \item convergência dentro da tolerância de \textit{gap} especificada;
     \item interrupção controlada em modos heurísticos ou matheurísticos.
 \end{itemize}

 Esses critérios garantem controle explícito sobre o esforço computacional,
 evitando execuções excessivamente longas e permitindo comparações justas entre
 diferentes estratégias de resolução.

 \subsection{Estados de Solução}

 Ao término da execução, o solver retorna um estado que indica a situação da
 resolução, podendo assumir, entre outros, os seguintes valores:

 \begin{itemize}
     \item \texttt{OPTIMAL}: solução ótima encontrada;
     \item \texttt{FEASIBLE}: solução viável encontrada dentro dos limites;
     \item \texttt{INFEASIBLE}: nenhuma solução viável existe;
     \item \texttt{UNBOUNDED}: problema mal formulado;
     \item \texttt{NOT\_SOLVED}: execução interrompida antes da obtenção de
     solução.
 \end{itemize}

 Nos modos heurístico e LNS, soluções viáveis são aceitas mesmo na ausência de
 prova de otimalidade, desde que respeitem todas as restrições ativadas.

 \subsection{Execução nos Diferentes Modos de Resolução}

 No modo exato, o modelo completo é submetido diretamente ao CP-SAT, que realiza
 a busca por soluções ótimas respeitando integralmente todas as restrições
 legais e operacionais.

 No modo heurístico, o solver pode ser utilizado de forma limitada ou não ser
 acionado, dependendo da configuração, sendo o foco a obtenção rápida de uma
 solução viável inicial.

 No modo LNS, o solver é invocado repetidamente para resolver subproblemas
 restritos, nos quais apenas subconjuntos de variáveis são liberados a cada
 iteração. Esse processo permite melhorar progressivamente a solução, mantendo
 controle explícito sobre o esforço computacional.

 \subsection{Extração da Solução}

 Uma vez concluída a execução, o simulador extrai de forma sistemática os
 valores das variáveis de decisão, construindo:

 \begin{itemize}
     \item a matriz de alocação motorista--período;
     \item o número total de motoristas ativos;
     \item a carga de trabalho por motorista;
     \item métricas agregadas de cobertura e eficiência.
 \end{itemize}

 Essa extração é realizada de maneira uniforme para todos os modos de
 resolução, garantindo comparabilidade direta entre soluções exatas,
 heurísticas e matheurísticas.

 \subsection{Relação com os Resultados Experimentais}

 Os dados retornados pelo solver alimentam diretamente os indicadores e
 visualizações apresentados no Capítulo~6. Tempos de execução, estados de
 solução, valores da função objetivo e métricas derivadas são registrados de
 forma automática, assegurando consistência entre a implementação
 computacional e a análise empírica.

 Dessa forma, a execução do solver constitui o elo fundamental entre o modelo
 matemático e os resultados experimentais discutidos posteriormente.

 \section{Fluxo Computacional do Simulador}

 O simulador desenvolvido segue um fluxo computacional bem definido e
 determinístico, estruturado em etapas sequenciais que garantem
 reprodutibilidade, consistência experimental e comparabilidade entre os
 diferentes modos de resolução disponíveis.

 Independentemente da estratégia algorítmica selecionada (exata, heurística ou
 LNS), o fluxo geral de execução mantém a mesma estrutura lógica, diferenciando-se
 apenas nos mecanismos internos de resolução.

 \subsection{Etapas do Fluxo de Execução}

 O fluxo computacional do simulador pode ser descrito pelas seguintes etapas
 principais:

 \begin{enumerate}
     \item \textbf{Leitura e validação dos parâmetros de entrada}: o sistema
     coleta, por meio da interface, os parâmetros do problema, incluindo
     horizonte temporal, demanda, restrições legais ativadas, modo de resolução
     e função objetivo.
     
     \item \textbf{Construção do modelo matemático}: com base nos parâmetros
     definidos, o simulador instancia as variáveis de decisão, a função objetivo
     selecionada e o conjunto de restrições ativas, conforme o modelo formal
     descrito no Capítulo~4.
     
     \item \textbf{Montagem da matriz de restrições}: o modelo é convertido para
     sua representação matricial interna, preservando a estrutura esparsa e o
     encadeamento temporal característicos do problema.
     
     \item \textbf{Execução do método de resolução}: de acordo com o modo
     selecionado, o sistema executa:
     \begin{itemize}
         \item a resolução exata via CP-SAT;
         \item a heurística construtiva para geração de solução inicial;
         \item o processo iterativo de Large Neighborhood Search (LNS).
     \end{itemize}
     
     \item \textbf{Extração da solução}: os valores das variáveis são extraídos
     de forma padronizada, permitindo a construção da matriz de alocação,
     métricas agregadas e indicadores operacionais.
     
     \item \textbf{Pós-processamento e análise}: o sistema calcula indicadores
     de desempenho, gera visualizações gráficas e registra métricas
     computacionais, como tempo de execução e estado da solução.
 \end{enumerate}

 \subsection{Determinismo e Reprodutibilidade}

 Para um conjunto fixo de parâmetros de entrada, o fluxo computacional do
 simulador é determinístico, assegurando que execuções repetidas produzam os
 mesmos resultados. Essa propriedade é essencial para a validação científica
 dos experimentos e para a comparação rigorosa entre diferentes estratégias de
 resolução.

 Mesmo nos modos heurístico e LNS, nos quais decisões locais e processos
 iterativos estão presentes, o uso de critérios controlados e a ausência de
 componentes estocásticos garantem consistência entre execuções equivalentes.

 \subsection{Fluxo Específico por Modo de Resolução}

 No modo exato, o fluxo computacional segue diretamente da construção do modelo
 para a execução do solver CP-SAT, que explora o espaço de soluções até atingir
 os critérios de parada definidos.

 No modo heurístico, a geração da solução ocorre por meio de regras construtivas
 gulosas, com foco na obtenção rápida de uma solução viável, respeitando as
 restrições legais ativadas.

 No modo LNS, o fluxo incorpora ciclos iterativos adicionais, nos quais
 subconjuntos de variáveis são liberados a cada iteração, mantendo o restante da
 solução fixo. A cada ciclo, um subproblema restrito é resolvido pelo solver
 inteiro, permitindo melhorias progressivas da solução.

 Apesar dessas diferenças internas, o fluxo de entrada e saída permanece
 uniforme, o que garante comparabilidade direta entre os resultados obtidos nos
 três modos.

 \subsection{Integração com os Resultados Experimentais}

 O fluxo computacional descrito neste capítulo fornece a base operacional para
 todos os experimentos apresentados no Capítulo~6. A uniformidade na coleta de
 métricas, na extração das soluções e no cálculo dos indicadores assegura que as
 diferenças observadas nos resultados sejam atribuídas exclusivamente às
 estratégias de resolução adotadas, e não a variações no processo de execução.

 Dessa forma, o fluxo computacional do simulador estabelece o elo entre a
 formulação matemática, a implementação computacional e a validação empírica
 dos métodos propostos.

 \section{Modos de Resolução e Estratégias de Otimização}

 No simulador desenvolvido, é feita uma distinção clara entre os conceitos de
 \textit{modo de resolução} e \textit{estratégia de otimização}. O modo de
 resolução refere-se ao método algorítmico empregado para explorar o espaço de
 soluções, enquanto a estratégia de otimização define o critério adotado pela
 função objetivo do modelo matemático.

 Essa separação conceitual permite combinar livremente diferentes abordagens de
 resolução com diferentes objetivos operacionais, ampliando o escopo analítico
 da plataforma experimental.

 \subsection{Modo de Resolução Exato}

 No modo de resolução exato, o problema completo de escalonamento é formulado
 como um modelo de Programação Linear Inteira e resolvido diretamente pelo
 solver CP-SAT. Nesse modo, todas as variáveis, restrições legais ativadas e a
 função objetivo selecionada são consideradas simultaneamente, permitindo a
 busca por soluções ótimas comprovadas, sempre que o tempo computacional
 disponível o permita.

 Esse modo fornece uma referência de qualidade para os demais métodos,
 servindo como base para validação das soluções heurísticas e
 matheurísticas.

 \subsection{Modo Heurístico}

 O modo heurístico utiliza uma estratégia construtiva gulosa para gerar
 rapidamente uma solução viável. A heurística opera de forma incremental,
 priorizando o atendimento da demanda por período e respeitando os principais
 limites operacionais e legais ativadas.

 Embora não haja garantia de otimalidade nesse modo, ele apresenta baixo custo
 computacional e é particularmente adequado para cenários de grande escala ou
 situações em que respostas rápidas são necessárias. Além disso, a solução
 gerada pode ser utilizada como ponto de partida para métodos mais avançados,
 como o LNS.

 \subsection{Modo Matheurístico (LNS)}

 O modo LNS (\textit{Large Neighborhood Search}) combina elementos heurísticos
 com reotimizações locais via solver inteiro. A partir de uma solução inicial,
 subconjuntos de variáveis são liberados a cada iteração, mantendo o restante da
 solução fixo. Em seguida, um subproblema restrito é resolvido pelo solver,
 permitindo melhorias progressivas da solução.

 Essa abordagem explora de forma eficiente a estrutura temporal do problema,
 equilibrando qualidade da solução e esforço computacional. O LNS permite
 controlar explicitamente o tamanho das vizinhanças e o número de iterações,
 tornando-o adequado para análises experimentais comparativas.

 \subsection{Estratégias de Otimização}

 De forma independente do modo de resolução, o simulador permite selecionar a
 estratégia de otimização adotada pelo modelo matemático. As principais
 estratégias implementadas são:

 \begin{itemize}
     \item \textbf{Maximização da resposta à demanda}: prioriza o atendimento
     máximo da demanda operacional ao longo do horizonte de planejamento,
     admitindo, quando necessário, demanda não atendida de forma controlada;
     \item \textbf{Minimização do número de motoristas ativos}: busca reduzir o
     uso da força de trabalho, assumindo atendimento integral da demanda.
 \end{itemize}

 Essas estratégias refletem objetivos operacionais distintos e permitem avaliar
 compromissos entre cobertura e eficiência de recursos.

 \subsection{Combinação entre Modos e Estratégias}

 A arquitetura do simulador permite combinar qualquer modo de resolução com
 qualquer estratégia de otimização. Por exemplo, a minimização do número de
 motoristas pode ser realizada tanto por meio de uma resolução exata quanto por
 heurísticas ou LNS.

 Essa flexibilidade possibilita análises comparativas sistemáticas, nas quais
 os efeitos do método de resolução e da função objetivo podem ser avaliadas de
 forma isolada ou combinada.

 \subsection{Relação com os Experimentos Apresentados}

 Nos experimentos apresentados no Capítulo~6, os três modos de resolução
 (exato, heurístico e LNS) são avaliados sob estratégias de otimização
 equivalentes, garantindo comparabilidade direta dos resultados. As diferenças
 observadas em termos de qualidade da solução, tempo computacional e estabilidade
 são, portanto, atribuídas exclusivamente às características dos métodos de
 resolução adotados.

 Essa abordagem reforça a validade experimental do estudo e permite uma
 interpretação clara dos compromissos envolvidos em cada estratégia.

 \section{Geração de Indicadores e Gráficos}

 A geração de indicadores e visualizações gráficas constitui uma etapa essencial
 do simulador, pois permite avaliar de forma objetiva a qualidade das soluções
 obtidas e interpretar o comportamento dos diferentes métodos de resolução.
 Todos os indicadores apresentados são derivados diretamente das variáveis de
 decisão do modelo matemático, assegurando consistência entre formulação,
 implementação e análise empírica.

 Os indicadores foram projetados para responder a questões operacionais
 específicas, como cobertura da demanda, utilização da força de trabalho,
 estabilidade temporal e balanceamento da carga entre motoristas.

 \subsection{Indicadores de Cobertura da Demanda}

 O indicador de cobertura da demanda mede, para cada período discretizado, a
 diferença entre a demanda operacional e o número de motoristas efetivamente
 alocados. Esse indicador permite identificar períodos de subcobertura,
 cobertura exata ou sobrealocação.

 A partir dessa informação, são derivados indicadores agregados, como a taxa
 global de cobertura e a quantidade total de demanda não atendida, utilizados
 diretamente na análise comparativa entre métodos.

 \subsection{Indicadores de Utilização e Sobrecarga}

 Para avaliar o uso da força de trabalho, o simulador calcula métricas de
 utilização por motorista, incluindo:

 \begin{itemize}
     \item carga total de trabalho por motorista;
     \item média de carga entre motoristas ativos;
     \item identificação de sobrecargas individuais.
 \end{itemize}

 Esses indicadores permitem verificar se a solução concentra excessivamente a
 carga em poucos motoristas ou se promove uma distribuição mais equilibrada,
 aspecto particularmente relevante em cenários regulados.

 \subsection{Indicadores Temporais e Estabilidade}

 Indicadores temporais são utilizados para analisar a estabilidade da solução
 ao longo do horizonte de planejamento. Entre eles destacam-se:

 \begin{itemize}
     \item desvio padrão da cobertura por período;
     \item variação do número de motoristas ativos ao longo do tempo;
     \item regularidade dos padrões de alocação.
 \end{itemize}

 Essas métricas permitem avaliar não apenas a qualidade pontual da solução, mas
 também seu comportamento dinâmico, aspecto importante para aplicações
 operacionais reais.

 \subsection{Indicadores Computacionais}

 Além dos indicadores operacionais, o simulador registra métricas
 computacionais associadas ao processo de resolução, tais como:

 \begin{itemize}
     \item tempo total de execução;
     \item número de iterações (no caso do LNS);
     \item estado final da solução (ótima, viável, interrompida);
     \item \textit{gap} relativo entre solução e melhor limite conhecido.
 \end{itemize}

 Esses indicadores permitem comparar o esforço computacional exigido por cada
 modo de resolução, complementando a análise da qualidade das soluções obtidas.

 \subsection{Visualizações Gráficas}

 Os indicadores calculados são apresentados por meio de visualizações gráficas
 interativas, incluindo gráficos de linhas, barras e distribuições. Essas
 visualizações permitem explorar padrões temporais, identificar picos de
 demanda e analisar diferenças entre métodos de resolução de forma intuitiva.

 Embora o simulador gere uma ampla variedade de gráficos, nesta dissertação são
 apresentados apenas aqueles considerados mais representativos para a análise
 comparativa entre os métodos exato, heurístico e LNS, evitando a proliferação
 excessiva de artefatos visuais.

 \subsection{Relação com os Resultados Experimentais}

 Todos os indicadores e gráficos descritos nesta seção são utilizados
 diretamente nos experimentos apresentados no Capítulo~6. A padronização na
 geração e interpretação dessas métricas assegura que as conclusões extraídas
 sejam fundamentadas em evidências quantitativas consistentes e comparáveis
 entre os diferentes métodos avaliados.

 \section{Módulo de Operações Elementares}

 Além da resolução do problema de escalonamento, o simulador incorpora um módulo
 específico para a aplicação de operações elementares sobre a matriz de
 restrições. Esse módulo foi concebido com objetivos predominantemente
 educacionais, analíticos e diagnósticos, não interferindo no processo de
 otimização utilizado para a geração dos resultados experimentais principais.

 As operações elementares implementadas incluem transformações clássicas de
 álgebra linear, tais como troca de linhas, multiplicação de linhas por
 constantes e combinações lineares controladas entre restrições.

 \subsection{Objetivos do Módulo}

 A inclusão do módulo de operações elementares atende a três objetivos
 principais:

 \begin{itemize}
     \item apoiar atividades educacionais relacionadas à modelagem matemática
     e à Programação Linear Inteira;
     \item permitir a análise estrutural da matriz de restrições, em especial
     no que diz respeito à densidade, organização e encadeamento temporal;
     \item auxiliar no diagnóstico de problemas numéricos ou estruturais durante
     a fase de desenvolvimento do modelo.
 \end{itemize}

 Esse módulo possibilita explorar a relação entre a formulação algébrica do
 problema e seu impacto sobre o desempenho computacional do solver, sem alterar
 o conjunto de soluções viáveis.

 \subsection{Preservação da Equivalência Matemática}

 Todas as operações disponibilizadas no simulador são cuidadosamente
 implementadas de modo a preservar a equivalência matemática do sistema de
 restrições. Em particular, as transformações aplicadas não modificam o espaço
 de soluções viáveis nem o valor ótimo do problema, atuando apenas sobre sua
 representação algébrica.

 Essa propriedade é fundamental para garantir que o uso do módulo não comprometa
 a validade dos experimentos realizados, sendo sua aplicação restrita a
 contextos exploratórios e analíticos.

 \subsection{Análise da Densidade da Matriz}

 Um dos usos práticos do módulo de operações elementares é a análise da densidade
 da matriz de restrições antes e depois de transformações específicas. O
 simulador calcula métricas de densidade estrutural, permitindo avaliar como
 diferentes formas de organização das restrições influenciam o desempenho do
 solver.

 Essa análise é particularmente relevante em problemas de grande escala, nos
 quais pequenas variações na estrutura da matriz podem impactar significativamente
 o tempo de resolução.

 \subsection{Relação com os Resultados Apresentados}

 Embora o módulo de operações elementares não seja utilizado diretamente na
 geração das soluções analisadas no Capítulo~6, ele desempenhou papel importante
 durante o desenvolvimento e validação do modelo, auxiliando na compreensão da
 estrutura do problema e na identificação de formulações mais eficientes.

 Dessa forma, o módulo contribui para a robustez metodológica do trabalho,
 reforçando a transparência e o rigor científico adotados na implementação
 computacional.

 \section{Extensões Heurísticas e Metaheurísticas}

 Embora o núcleo do simulador seja fundamentado em um modelo de Programação Linear
 Inteira resolvido de forma exata pelo solver CP-SAT, a plataforma foi estended
 para incorporar métodos heurísticos e metaheurísticos. Essas extensões têm como
 objetivo ampliar o escopo experimental, reduzir tempos computacionais em cenários
 de maior escala e permitir a análise comparativa entre diferentes estratégias de
 resolução sob um mesmo conjunto de restrições legais.

 As abordagens heurísticas e metaheurísticas implementadas não substituem o modelo
 matemático exato, mas atuam como mecanismos complementares, sempre mantendo a
 validade jurídica e a consistência estrutural do problema.

 \subsection{Heurística Construtiva Gulosa}

 A primeira extensão implementada consiste em uma heurística construtiva gulosa
 (\textit{greedy initial allocation}), responsável por gerar rapidamente soluções
 viáveis a partir da demanda operacional por período.

 Essa heurística opera de forma incremental, priorizando períodos com maior
 demanda e alocando motoristas de acordo com regras locais de viabilidade,
 respeitando limites operacionais básicos e restrições legais essenciais. O
 objetivo principal dessa abordagem é produzir uma solução inicial consistente,
 com baixo custo computacional, que possa ser utilizada de forma independente ou
 como ponto de partida para métodos de melhoria.

 A heurística gulosa apresenta desempenho computacional significativamente superior
 ao método exato em cenários de grande escala, porém não garante otimalidade global,
 sendo particularmente adequada para análises exploratórias e aplicações em tempo
 quase real.

 \subsection{Large Neighborhood Search (LNS)}

 Para superar as limitações inerentes às heurísticas construtivas, o simulador
 incorpora uma abordagem de \textit{Large Neighborhood Search} (LNS), caracterizada
 pela destruição e reconstrução parcial da solução.

 No método LNS, a partir de uma solução inicial viável (gerada pela heurística
 gulosa ou pelo solver exato), subconjuntos de períodos são iterativamente
 liberados, enquanto o restante da solução permanece fixo. Em cada iteração, um
 problema restrito é reconstruído e resolvido novamente pelo solver inteiro,
 permitindo melhorias progressivas da solução.

 Essa estratégia combina a robustez da otimização exata com a flexibilidade
 exploratória típica das metaheurísticas, oferecendo um equilíbrio eficaz entre
 qualidade da solução e esforço computacional.

 \subsection{Controle de Relaxação e Iterações}

 O simulador permite controlar explicitamente parâmetros associados ao processo
 LNS, tais como o tamanho da vizinhança liberada, o número máximo de iterações e o
 nível de relaxação aplicado em cada etapa.

 Esse controle possibilita estudar o comportamento do algoritmo sob diferentes
 configurações, analisando o impacto de decisões como a intensidade da destruição,
 a frequência de reotimizações e os critérios de parada adotados.

 \subsection{Garantia de Validade das Soluções}

 Um aspecto central da implementação é que todas as soluções finais geradas pelos
 métodos heurísticos e metaheurísticos são sempre validadas pelo modelo de
 Programação Linear Inteira e pelo solver CP-SAT. Mesmo quando uma solução é
 construída heurística ou parcialmente, sua aceitação final depende do atendimento
 integral das restrições legais e operacionais.

 Essa estratégia garante que nenhuma solução inválida do ponto de vista jurídico
 ou matemático seja utilizada nos experimentos apresentados nesta dissertação.

 \subsection{Relação com os Experimentos Computacionais}

 As extensões heurísticas e metaheurísticas desempenham papel fundamental nos
 experimentos discutidos no Capítulo~6. Em particular, os resultados permitem
 comparar:

 \begin{itemize}
     \item qualidade da solução obtida;
     \item tempo computacional;
     \item estabilidade da cobertura da demanda;
     \item eficiência na utilização da força de trabalho.
 \end{itemize}

 Essas comparações evidenciam os compromissos entre métodos exatos, heurísticos e
 metaheurísticos, reforçando a importância de abordagens híbridas em problemas de
 escalonamento sob restrições legais complexas.

 \section{Integração de Aprendizado de Máquina}

 Além das abordagens exatas, heurísticas e metaheurísticas, o simulador incorpora
 um módulo experimental de aprendizado de máquina supervisionado, cujo objetivo é
 auxiliar decisões locais e estratégicas durante o processo de otimização. Essa
 integração foi concebida como uma camada complementar, não substitutiva, mantendo
 plena compatibilidade com o modelo matemático exato e com as restrições legais do
 Regulamento (CE) n.\textordmasculine~561/2006.

 O aprendizado de máquina não altera diretamente a formulação do problema, a
 função objetivo nem o conjunto de restrições. Sua função é orientar escolhas
 heurísticas, reduzindo o espaço de busca e priorizando decisões potencialmente
 mais promissoras, sem comprometer a validade jurídica ou matemática das soluções.

 \subsection{Motivação para o Uso de Aprendizado Supervisionado}

 Problemas de escalonamento sob restrições temporais rígidas apresentam elevada
 complexidade combinatória, especialmente quando resolvidos repetidamente em
 cenários de simulação. Embora solvers exatos sejam capazes de produzir soluções
 ótimas, o custo computacional pode crescer rapidamente com o tamanho do
 horizonte e da força de trabalho.

 Nesse contexto, o aprendizado supervisionado surge como uma ferramenta para
 capturar padrões recorrentes observados em soluções ótimas ou quase ótimas,
 auxiliando o processo decisório em etapas heurísticas e metaheurísticas, sem
 abrir mão do rigor do modelo exato.

 \subsection{Modelos Supervisionados Implementados}

 Foram definidos dois modelos supervisionados distintos, com papéis claramente
 separados no simulador.

 O primeiro modelo, denominado $f_1$, atua no nível local, avaliando decisões de
 alocação motorista--período durante a heurística construtiva. Esse modelo recebe
 como entrada características associadas ao período, à demanda, ao estado do
 motorista e ao contexto operacional, retornando um escore que representa a
 atratividade relativa daquela alocação.

 O segundo modelo, denominado $f_2$, opera em nível agregado no contexto do
 algoritmo LNS. Seu objetivo é avaliar subconjuntos de períodos candidatos à
 liberação, estimando o potencial de melhoria da solução caso aquela vizinhança
 seja explorada. Dessa forma, o modelo auxilia na priorização de vizinhanças mais
 promissoras, reduzindo o número de iterações improdutivas.

 \subsection{Arquitetura Resiliente e Comportamento Padrão}

 A integração com aprendizado de máquina foi projetada de forma resiliente. Caso
 os modelos não estejam disponíveis, não estejam treinados ou o ambiente de
 execução não suporte o módulo de aprendizado, o simulador recorre
 automaticamente a regras heurísticas clássicas e determinísticas.

 Esse mecanismo garante que o comportamento padrão do simulador seja sempre
 reproduzível e independente do aprendizado de máquina, evitando qualquer
 dependência estrutural ou viés não controlado nos experimentos.

 \subsection{Geração de Dados e Treinamento dos Modelos}

 Para viabilizar o aprendizado supervisionado, foi desenvolvido um pipeline
 automático de geração de dados. A partir de múltiplas instâncias sintéticas de
 demanda, o simulador executa soluções heurísticas e soluções exatas via solver
 inteiro, registrando decisões intermediárias, estados do sistema e métricas de
 desempenho.

 Os dados coletados são organizados em dois conjuntos distintos: um dataset
 voltado ao treinamento do modelo $f_1$, com informações locais de decisão, e um
 dataset voltado ao modelo $f_2$, composto por características agregadas das
 vizinhanças exploradas no LNS.

 O treinamento dos modelos é realizado externamente ao fluxo principal de
 otimização, utilizando algoritmos de \textit{gradient boosting}. Os modelos
 treinados são posteriormente carregados dinamicamente pelo simulador, não
 interferindo na formulação matemática nem no processo de validação das soluções.

 \subsection{Papel do Aprendizado de Máquina nos Experimentos}

 Nos experimentos apresentados nesta dissertação, o uso de aprendizado de máquina
 é tratado como um recurso opcional e experimental. Os resultados principais
 utilizam o comportamento determinístico do simulador, permitindo comparações
 justas entre métodos exatos, heurísticos e metaheurísticos.

 Quando ativado, o aprendizado supervisionado atua como um mecanismo de
 aceleração e refinamento, potencialmente reduzindo tempos de convergência e
 melhorando a estabilidade das soluções, sem alterar os critérios formais de
 avaliação adotados.

 \subsection{Garantia de Rigor Metodológico}

 É importante destacar que todas as soluções finais, independentemente do uso ou
 não de aprendizado de máquina, são sempre validadas pelo modelo de Programação
 Linear Inteira e pelo solver CP-SAT. O aprendizado supervisionado não possui
 autonomia para violar restrições, modificar objetivos ou aceitar soluções
 inválidas.

 Dessa forma, a integração de aprendizado de máquina preserva rigor científico,
 transparência metodológica e conformidade legal, atuando exclusivamente como uma
 camada auxiliar de apoio à decisão.

 \section{Geração de Datasets e Treinamento dos Modelos}

 A integração de aprendizado de máquina no simulador exigiu o desenvolvimento de
 um pipeline específico para geração de dados supervisionados, totalmente
 compatível com o domínio do problema e com o modelo matemático subjacente. Esse
 pipeline foi concebido para garantir consistência entre os dados utilizados no
 treinamento e as decisões efetivamente tomadas pelo sistema de escalonamento.

 \subsection{Origem dos Dados Supervisionados}

 Os datasets utilizados no treinamento dos modelos supervisionados não são
 provenientes de fontes externas nem de bases genéricas. Eles são gerados
 diretamente pelo próprio simulador, a partir da execução controlada de múltiplos
 cenários de escalonamento.

 Cada instância de dados é construída a partir de:
 \begin{itemize}
     \item curvas sintéticas de demanda por período;
     \item diferentes configurações de restrições legais ativas;
     \item soluções heurísticas iniciais;
     \item soluções exatas obtidas via solver inteiro;
     \item histórico de decisões intermediárias durante o processo de otimização.
 \end{itemize}

 Essa estratégia garante que os dados de treinamento reflitam fielmente a lógica
 do problema, suas restrições legais e os critérios operacionais adotados no
 modelo matemático.

 \subsection{Dataset para o Modelo Local de Alocação ($f_1$)}

 O primeiro conjunto de dados foi projetado para treinar o modelo local de decisão
 $f_1$, utilizado durante a heurística construtiva. Cada observação do dataset
 corresponde a uma decisão candidata de alocação motorista--período.

 As variáveis explicativas incluem, entre outras:
 \begin{itemize}
     \item demanda do período;
     \item nível de cobertura acumulada;
     \item estado do motorista (ativo, descanso, condução acumulada);
     \item posição temporal no horizonte;
     \item indicadores locais de carga e utilização.
 \end{itemize}

 O rótulo associado a cada observação indica a qualidade daquela decisão, medida
 com base na solução final obtida (ótima ou quase ótima), permitindo ao modelo
 aprender padrões de alocação mais eficientes.

 \subsection{Dataset para Avaliação de Vizinhanças no LNS ($f_2$)}

 O segundo dataset é voltado ao treinamento do modelo $f_2$, responsável por
 avaliar vizinhanças candidatas no algoritmo LNS. Nesse caso, cada observação
 representa um subconjunto de períodos liberados para reotimização em uma
 iteração do LNS.

 As características utilizadas incluem:
 \begin{itemize}
     \item tamanho da vizinhança;
     \item posição temporal dos períodos liberados;
     \item nível de violação ou folga das restrições;
     \item variação da função objetivo nas iterações anteriores;
     \item indicadores agregados de estabilidade e carga.
 \end{itemize}

 O rótulo indica o ganho efetivo obtido após a reotimização daquela vizinhança,
 permitindo ao modelo aprender quais padrões tendem a gerar melhorias mais
 significativas.

 \subsection{Pipeline de Geração Automática}

 A geração dos datasets é realizada de forma automática e offline, por meio de
 execuções repetidas do simulador em diferentes cenários. Esse processo não
 interfere no fluxo principal de otimização e não ocorre durante a execução dos
 experimentos analisados nesta dissertação.

 Cada execução gera registros estruturados que são armazenados em formato tabular,
 permitindo posterior auditoria, reprodutibilidade e controle de versões dos
 datasets utilizados.

 \subsection{Treinamento dos Modelos Supervisionados}

 O treinamento dos modelos $f_1$ e $f_2$ é realizado externamente ao simulador,
 utilizando algoritmos de \textit{gradient boosting}, devido à sua capacidade de
 modelar relações não lineares e lidar eficientemente com variáveis heterogêneas.

 Os modelos treinados são posteriormente exportados e carregados dinamicamente
 pelo simulador, sem necessidade de recompilação ou modificação do núcleo de
 otimização.

 \subsection{Separação entre Treinamento e Avaliação}

 É importante destacar que os dados utilizados para treinamento dos modelos não
 são reutilizados nos experimentos de avaliação apresentados nesta dissertação.
 Essa separação evita vazamento de informação e garante que os resultados
 observados não sejam artificialmente favorecidos pelo aprendizado supervisionado.

 Nos experimentos principais, o simulador é avaliado tanto com quanto sem o uso
 dos modelos treinados, permitindo comparar o impacto real do aprendizado de
 máquina sobre desempenho computacional e qualidade das soluções.

 \subsection{Considerações Metodológicas}

 A estratégia adotada garante que o aprendizado supervisionado:
 \begin{itemize}
     \item não introduz viés externo ao problema;
     \item não altera a formulação matemática;
     \item não compromete a validade legal das soluções;
     \item atua exclusivamente como mecanismo auxiliar.
 \end{itemize}

 Dessa forma, o uso de aprendizado de máquina reforça o caráter experimental e
 exploratório da plataforma, sem comprometer o rigor científico exigido para a
 análise dos resultados apresentados.

 \section{Visualização Avançada e Análise Iterativa}

 A visualização dos resultados desempenha papel central no simulador desenvolvido,
 não apenas como mecanismo de apresentação, mas como ferramenta analítica para
 interpretação do comportamento dos algoritmos, diagnóstico de soluções e validação
 das hipóteses experimentais. Para esse fim, foram implementados módulos de
 visualização capazes de representar tanto indicadores agregados quanto informações
 temporais detalhadas.

 \subsection{Visualizações Temporais da Solução}

 Um conjunto fundamental de gráficos apresenta a evolução temporal da demanda, da
 capacidade alocada e da cobertura efetiva ao longo do horizonte de planejamento.
 Essas visualizações permitem verificar, período a período, se a solução obtida
 atende integralmente à demanda operacional ou se existem déficits pontuais.

 Adicionalmente, são gerados gráficos de margem de segurança, que evidenciam
 situações de sobrecarga ou subutilização, permitindo avaliar o grau de robustez da
 solução frente a variações operacionais. Esses gráficos são particularmente úteis
 para identificar períodos críticos e padrões recorrentes de instabilidade.

 \subsection{Indicadores de Estabilidade e Eficiência}

 Para além da cobertura da demanda, o simulador calcula indicadores de estabilidade
 e eficiência associados à alocação de motoristas. Entre eles, destacam-se:
 \begin{itemize}
     \item o desvio padrão da cobertura ao longo do tempo;
     \item a distribuição da carga de trabalho por motorista;
     \item o tempo médio de presença ativa;
     \item a relação entre esforço alocado e demanda atendida.
 \end{itemize}

 Esses indicadores são apresentados tanto de forma numérica quanto gráfica,
 permitindo avaliar se a solução, embora viável, apresenta comportamento regular
 ou excessivamente concentrado em determinados motoristas ou períodos.

 \subsection{Análise Iterativa no Modo LNS}

 No modo de resolução LNS, visualizações adicionais são geradas para acompanhar o
 processo iterativo de melhoria da solução. Em particular, são apresentados gráficos
 que mostram:
 \begin{itemize}
     \item a evolução do valor da função objetivo ao longo das iterações;
     \item o impacto das liberações parciais da solução;
     \item o comportamento do nível de relaxamento aplicado;
     \item a frequência e magnitude das melhorias obtidas.
 \end{itemize}

 Essas visualizações permitem analisar empiricamente a convergência do método,
 identificando se o algoritmo se estabiliza prematuramente ou continua a produzir
 ganhos marginais ao longo das iterações.

 \subsection{Comparação entre Modos de Resolução}

 O simulador permite a visualização comparativa direta entre os modos exato,
 heurístico e LNS. Gráficos consolidados exibem diferenças em termos de:
 \begin{itemize}
     \item tempo computacional;
     \item qualidade da solução;
     \item número de motoristas ativos;
     \item estabilidade temporal da alocação.
 \end{itemize}

 Essa comparação visual facilita a identificação de compromissos entre esforço
 computacional e qualidade da solução, reforçando o caráter experimental da
 plataforma.

 \subsection{Visualizações para Diagnóstico e Auditoria}

 Além das visualizações orientadas a desempenho, o simulador oferece gráficos
 voltados ao diagnóstico e à auditoria das soluções. Mapas de calor da matriz de
 alocação permitem identificar padrões de ativação, blocos contínuos de trabalho e
 eventuais irregularidades temporais.

 Essas representações visuais auxiliam na verificação manual da conformidade legal,
 facilitando a identificação de violações sutis que poderiam não ser evidentes em
 indicadores agregados.

 \subsection{Integração com o Processo Experimental}

 As visualizações avançadas não são utilizadas apenas como saída final, mas como
 parte integrante do processo experimental. Elas orientam ajustes de parâmetros,
 avaliação de estratégias heurísticas e calibração de métodos LNS e de aprendizado
 de máquina.

 Dessa forma, o módulo de visualização contribui diretamente para o refinamento do
 simulador e para a interpretação dos resultados apresentados nos capítulos
 experimentais seguintes, fortalecendo a conexão entre implementação
 computacional e análise empírica.

 \section{Posicionamento do Simulador como Plataforma Experimental}

 A implementação computacional desenvolvida ao longo deste trabalho ultrapassa o
 papel tradicional de um resolvedor de problemas de Programação Linear Inteira,
 configurando-se como uma plataforma experimental híbrida para o estudo do
 escalonamento de motoristas sob regulamentação europeia.

 Diferentemente de abordagens focadas exclusivamente na obtenção de uma solução
 ótima, o simulador foi concebido para permitir a investigação sistemática de
 diferentes estratégias de resolução, objetivos operacionais e configurações de
 restrições, mantendo sempre como núcleo normativo o modelo matemático exato
 apresentado nos capítulos anteriores.

 \subsection{Unificação de Estratégias de Resolução}

 A plataforma integra, em um mesmo ambiente computacional, métodos exatos,
 heurísticos e matheurísticos, possibilitando comparações diretas sob condições
 controladas. A Programação Linear Inteira resolvida via CP-SAT atua como referência
 de validade e qualidade, enquanto a heurística construtiva e o método LNS oferecem
 alternativas computacionalmente mais eficientes para cenários de maior escala ou
 com restrições temporais mais severas.

 Essa unificação permite analisar, de forma rigorosa, os compromissos entre
 qualidade da solução, esforço computacional e estabilidade operacional, aspecto
 essencial em aplicações reais de transporte rodoviário.

 \subsection{Separação entre Modelo, Resolução e Análise}

 Um dos princípios centrais da arquitetura do simulador é a separação clara entre:
 \begin{itemize}
     \item o modelo matemático normativo;
     \item os métodos de resolução empregados;
     \item os mecanismos de análise e visualização.
 \end{itemize}

 Essa separação garante que modificações em estratégias heurísticas, parâmetros do
 solver ou módulos de aprendizado de máquina não comprometam a validade do modelo
 legal subjacente. Ao mesmo tempo, facilita a extensão da plataforma para novos
 cenários, restrições ou objetivos sem necessidade de reformulação estrutural.

 \subsection{Reprodutibilidade e Controle Experimental}

 O simulador foi projetado para assegurar reprodutibilidade dos experimentos.
 Para um conjunto fixo de parâmetros de entrada — horizonte temporal, demanda,
 restrições legais e modo de resolução — o comportamento do sistema é determinístico,
 permitindo a replicação integral dos resultados apresentados.

 Além disso, o controle explícito sobre a ativação de restrições, critérios de
 otimização e estratégias de resolução possibilita a realização de estudos de
 sensibilidade e análises comparativas rigorosas, fundamentais para avaliação
 científica.

 \subsection{Integração Opcional de Aprendizado de Máquina}

 O aprendizado de máquina foi integrado como uma camada complementar, não como
 substituto da otimização exata. Sua utilização restringe-se à priorização de
 decisões locais e à seleção de vizinhanças promissoras no contexto do LNS, sem
 alterar diretamente restrições ou funções objetivo.

 Essa abordagem preserva o rigor matemático do modelo, ao mesmo tempo em que amplia
 o potencial experimental da plataforma, permitindo investigar a interação entre
 otimização combinatória e aprendizado supervisionado de forma controlada e
 interpretável.

 \subsection{Aplicabilidade Científica e Industrial}

 Do ponto de vista científico, o simulador oferece um ambiente completo para a
 investigação de problemas de escalonamento sob restrições legais complexas,
 viabilizando estudos comparativos, análises de escalabilidade e experimentação com
 estratégias híbridas.

 Do ponto de vista industrial, a arquitetura modular, a capacidade de operar com
 dados reais e a interface interativa tornam a plataforma adequada para aplicações
 em contextos corporativos, estudos de viabilidade operacional e suporte à tomada
 de decisão, preservando confidencialidade e conformidade regulatória.

 \subsection{Síntese do Capítulo}

 Em síntese, o simulador desenvolvido consolida-se como uma plataforma experimental
 robusta, flexível e extensível, que conecta rigor matemático, implementação
 computacional e análise empírica. Ele fornece a base operacional necessária para os
 experimentos e comparações apresentados nos capítulos seguintes, assegurando
 consistência metodológica entre formulação teórica, implementação e validação dos
 resultados.
