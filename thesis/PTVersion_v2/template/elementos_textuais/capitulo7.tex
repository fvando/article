\chapter{Resultados}\label{cap:resultados}

\section{Introdução}

Este capítulo apresenta os resultados computacionais obtidos a partir da execução do modelo de Programação Linear Inteira (PLI) e da implementação computacional descrita no Capítulo~\ref{cap:implementacao}. Os experimentos foram conduzidos utilizando dados reais de demanda operacional, com granularidade temporal de 15 minutos, respeitando integralmente as restrições impostas pelo Regulamento (CE) n.º~561/2006.

O objetivo principal desta etapa é avaliar a viabilidade prática, a qualidade das soluções e o desempenho computacional do modelo proposto, considerando diferentes estratégias de resolução implementadas no simulador. Em particular, são analisados três modos de resolução: (i) solução exata via CP-SAT, (ii) heurística construtiva gulosa e (iii) abordagem matheurística baseada em Large Neighborhood Search (LNS).

Este capítulo inicia-se com a análise detalhada dos resultados obtidos pelo método exato, que serve como referência normativa e de qualidade para a comparação posterior com os métodos heurísticos e metaheurísticos.


\section{Roteiro Experimental}

Os resultados apresentados neste capítulo seguem um roteiro experimental estruturado, concebido para avaliar de forma progressiva o comportamento do simulador sob diferentes estratégias de resolução e níveis de complexidade.

O roteiro foi organizado nas seguintes etapas:
\begin{itemize}
    \item validação do modelo matemático por meio da solução exata;
    \item avaliação do desempenho computacional do solver CP-SAT;
    \item análise da qualidade das soluções em termos de cobertura, estabilidade e eficiência;
    \item comparação entre métodos exato, heurístico e LNS;
    \item discussão dos compromissos entre qualidade da solução e esforço computacional.
\end{itemize}

Essa abordagem sistemática garante rastreabilidade, reprodutibilidade e rigor científico, permitindo que cada estratégia seja analisada sob condições controladas e comparáveis.


\section{Configuração Geral dos Cenários}

Os cenários experimentais foram configurados com base nos seguintes parâmetros principais:

\begin{itemize}
    \item solver: CP-SAT (OR-Tools);
    \item granularidade temporal: 15 minutos por período;
    \item horizonte de referência apresentado neste capítulo: 24 horas (96 períodos);
    \item variável principal: \(x_{d,t}\) (binária);
    \item ativação integral das restrições legais;
    \item demanda real, variável e não uniforme por período.
\end{itemize}

Embora o simulador suporte horizontes de 7, 15 e 30 dias, este capítulo apresenta inicialmente os resultados detalhados do cenário diário de 24 horas, por se tratar do menor horizonte completo capaz de capturar todas as restrições diárias. Posteriormente, na Seção~\ref{sec:horizontes-estendidos}, são apresentados os resultados computacionais para horizontes de 7 e 15 dias, avaliando explicitamente os limites de escalabilidade do método exato frente à abordagem matheurística proposta.

Essas configurações permitem avaliar não apenas a viabilidade legal das escalas em ciclos curtos, mas também a estabilidade e robustez do modelo em planejamentos de médio prazo.

\begin{table}[H]
\centering
\caption{Parâmetros gerais e restrições ativas nos cenários experimentais.}
\label{tab:parametros-experimentos}
\begin{tabular}{lccc}
\toprule
\textbf{Restrição (Regulam. CE 561/2006)} & \textbf{24 Horas} & \textbf{7 Dias} & \textbf{15 Dias} \\
\midrule
\multicolumn{4}{l}{\textit{Parâmetros Gerais}} \\
Horizonte Temporal & 24 h & 168 h & 336 h \\
Granularidade ($p$) & 15 min & 15 min & 15 min \\
Perfil de Configuração & P2 Strict & W1 Core & B1 Core \\
\midrule
\multicolumn{4}{l}{\textit{Restrições Legais}} \\
Cobertura de Demanda & \checkmark & \checkmark & \checkmark \\
Limite Diário de Condução (9h) & \checkmark & \checkmark & \checkmark \\
Pausas Obrigatórias (45 min) & \checkmark & \checkmark & \checkmark \\
Repouso Diário Mínimo (11h) & \checkmark & -- & -- \\
Limite Semanal de Condução (56h) & -- & \checkmark & \checkmark \\
Limite Quinzenal de Condução (90h) & -- & -- & \checkmark \\
Repouso Semanal & -- & -- & -- \\
Repouso Quinzenal & -- & -- & -- \\
\bottomrule
\multicolumn{4}{l}{\footnotesize \checkmark: Restrição Ativa \quad --: Restrição Inativa}
\end{tabular}
\end{table}

\section{Execução do Solver}

A execução do solver CP-SAT para o cenário de 24 horas, com todas as restrições legais ativas, permite extrair indicadores de desempenho e parâmetros de busca. Maiores detalhes sobre a interface de configuração do simulador podem ser consultados no Apêndice~\ref{cap:interface-simulador}.

Observa-se que o solver retorna soluções ótimas de forma consistente, com tempos computacionais médios inferiores a dois segundos, mesmo em instâncias com milhares de variáveis e restrições. Em média, os modelos resolvidos apresentaram cerca de 9.600 variáveis e aproximadamente 7.600 restrições lineares.

% Esse desempenho confirma a adequação da formulação proposta e do uso do CP-SAT para aplicações operacionais, permitindo replanejamentos frequentes e análises exploratórias em tempo quase real.

Esse desempenho confirma que a formulação PLI, associada ao CP-SAT, é adequada para aplicação prática no setor, permitindo replanejamento rápido e análises exploratórias sob condições operacionais realistas.



\section{Comportamento da Demanda}

A demanda utilizada nos experimentos apresenta elevada variabilidade ao longo do horizonte analisado, característica típica de operações de transporte rodoviário com janelas de atendimento contínuas e múltiplos centros operacionais.

% A Figura~\ref{fig:demanda-experimentos} ilustra a curva de demanda considerada.

% \begin{figure}[H]
%     \centering
%     \includegraphics[width=0.92\textwidth]{template/figuras/T12.png}
%     \caption{Curva de demanda operacional por período.}
%     \label{fig:demanda-experimentos}
% \end{figure}

Observam-se períodos de pico com elevada necessidade de motoristas, intercalados com intervalos de menor demanda. Essa irregularidade não foi suavizada intencionalmente, pois constitui um elemento essencial para testar a robustez do modelo frente a cenários realistas, nos quais variações abruptas são frequentes.

O bom desempenho do modelo mesmo sob essas condições reforça sua aplicabilidade prática.


\section{Comparação Demanda \textit{versus} Alocação}

A Figura~\ref{fig:demanda-vs-cobertura-2} apresenta a comparação entre a demanda operacional e a alocação de motoristas obtida pelo método exato.

\begin{figure}[H]
    \centering
    \includegraphics[width=\textwidth]{template/figuras/demanda_capacidade_exact_new.png}
    \caption{Comparação entre demanda e cobertura gerada pelo modelo exato.}
    \label{fig:demanda-vs-cobertura-2}
\end{figure}

Observa-se elevada aderência entre a curva de demanda e a alocação resultante, indicando que o modelo atende aos requisitos operacionais sem introduzir sobrealocação significativa. Em períodos de menor demanda, a alocação reduz-se de forma consistente, refletindo o critério de eficiência operacional adotado.

% Esse comportamento confirma que o modelo responde adequadamente às variações temporais da demanda, mantendo equilíbrio entre cobertura e uso racional da força de trabalho.

Esse comportamento confirma que, no modo de minimização do número total de motoristas, o modelo responde adequadamente às variações temporais da demanda, evitando sobrealocação desnecessária e mantendo aderência estrita à necessidade operacional, sem comprometer a conformidade legal.




Para avaliar a qualidade das soluções obtidas, foram calculados diversos indicadores estatísticos, entre eles cobertura da demanda, sobrecarga, subutilização e desvio padrão da cobertura ao longo do tempo. Exemplos destes indicadores extraídos diretamente da interface do simulador encontram-se no Apêndice~\ref{cap:interface-simulador}.

% Os resultados indicam cobertura média superior a 99\%, baixa incidência de sobrecarga e subutilização restrita a períodos de menor demanda. O desvio padrão reduzido da cobertura evidencia elevada regularidade temporal, aspecto fundamental para estabilidade operacional e previsibilidade das escalas.

Os indicadores obtidos evidenciam elevada regularidade temporal da solução no cenário analisado, aspecto particularmente relevante em contextos operacionais, pois reduz oscilações abruptas na escala e facilita o planejamento logístico e a gestão de recursos humanos.

A matriz de restrições é construída com base nas regras legais e vínculos temporais, resultando em uma estrutura esparsa e quase diagonal. A estrutura matricial influencia diretamente o desempenho do solver, a propagação de restrições e a velocidade de convergência. Exemplos da visualização estrutural da matriz podem ser vistos no Apêndice~\ref{cap:interface-simulador}.

A estrutura matricial influencia diretamente:

\begin{itemize}
    \item o desempenho do solver;
    \item a propagação de restrições;
    \item a velocidade de convergência.
\end{itemize}

A esparsidade entre 15\% e 35\% favorece significativamente os algoritmos do CP-SAT, conforme discutido em \cite{erdman2022}.

\subsection{Resultados Computacionais — Método Exato}
\label{subsec:resultados-Exato}

Esta subseção apresenta os resultados obtidos por meio do método exato, fundamentado na resolução direta do modelo matemático de otimização. O procedimento adotado visa à obtenção de soluções factíveis e ótimas, respeitando integralmente as restrições legais e operacionais que caracterizam o problema. Embora mecanismos construtivos possam ser utilizados para acelerar a convergência inicial, a solução final é garantida pelo método exato, assegurando optimalidade dentro dos limites computacionais estabelecidos.

Os experimentos foram conduzidos no mesmo cenário operacional diário de 24 horas, com granularidade de 15 minutos (96 períodos), garantindo comparabilidade direta entre os métodos.

\subsubsection{Indicadores Globais de Desempenho}
\label{subsubsec:kpi-heuristic-exact}

\begin{table}[H]
\centering
\caption{Indicadores globais de desempenho da solução exata (24 horas).}
\label{tab:kpi-exact}
\begin{tabular}{lc}
\toprule
\textbf{Indicador} & \textbf{Valor} \\
\midrule
Cobertura Global & 100.0\% \\
Motoristas Ativos & 114 \\
Eficiência Média & 88.5\% \\
Risco Operacional & Baixo \\
Status da Solução & Ótimo \\
Tempo de Resolução & 4.2 s \\
\bottomrule
\end{tabular}
\end{table}

\begin{table}[H]
\centering
\caption{Indicadores globais de desempenho da solução heurística (24 horas).}
\label{tab:kpi-heuristic}
\begin{tabular}{lc}
\toprule
\textbf{Indicador} & \textbf{Valor} \\
\midrule
Cobertura Global & 100.0\% \\
Motoristas Ativos & 132 \\
Eficiência Média & 76.4\% \\
Risco Operacional & Nulo \\
Status da Solução & Viável \\
Tempo de Resolução & 0.1 s \\
\bottomrule
\end{tabular}
\end{table}

\begin{table}[H]
\centering
\caption{Indicadores globais de desempenho da solução LNS (24 horas).}
\label{tab:kpi-lns}
\begin{tabular}{lc}
\toprule
\textbf{Indicador} & \textbf{Valor} \\
\midrule
Cobertura Global & 100.0\% \\
Motoristas Ativos & 114 \\
Eficiência Média & 88.2\% \\
Risco Operacional & Controlado \\
Status da Solução & Ótimo \\
Tempo de Resolução & 35.0 s \\
\bottomrule
\end{tabular}
\end{table}

\subsubsection{Atendimento da Demanda e Utilização da Capacidade}
\label{subsubsec:demanda-capacidade-lns}

A Figura~\ref{fig:demanda-capacidade-lns} apresenta a relação entre demanda solicitada, capacidade alocada e atendimento efetivo após a aplicação do LNS.

Em comparação com o método heurístico, observa-se uma redução da capacidade excedente, com melhor alinhamento entre a capacidade alocada e a demanda real ao longo dos períodos. Essa melhoria evidencia a capacidade do LNS de reorganizar a solução inicial, eliminando redundâncias e concentrando recursos nos períodos em que são efetivamente necessários.

Embora o comportamento permaneça conservador, o padrão obtido pelo LNS aproxima-se daquele observado no modelo exato, porém com custo computacional significativamente inferior.

\begin{figure}[H]
\centering
\includegraphics[width=\textwidth]{template/figuras/demanda_capacidade_lns_new.png}
\caption{Demanda, capacidade alocada e atendimento efetivo por período — método LNS.}
\label{fig:demanda-capacidade-lns}
\end{figure}

\subsubsection{Distribuição do Esforço Operacional}
\label{subsubsec:esforco-lns}

A Figura~\ref{fig:esforco-lns} ilustra a distribuição do esforço operacional entre os motoristas ativos após a aplicação do LNS.

Os resultados indicam uma redução da carga média por motorista quando comparada ao método heurístico, bem como uma distribuição mais equilibrada do esforço. O índice de Gini permanece baixo, indicando que as melhorias obtidas não ocorrem à custa de concentração excessiva da carga de trabalho em subconjuntos específicos de motoristas.

Esse comportamento confirma que o LNS é eficaz na correção de desequilíbrios locais introduzidos pela heurística construtiva, promovendo soluções mais estáveis e operacionalmente robustas.

\begin{figure}[H]
\centering
\includegraphics[width=\textwidth]{template/figuras/esforco_operacional_lns_new.png}
\caption{Distribuição do esforço operacional dos motoristas ativos — método LNS.}
\label{fig:esforco-lns}
\end{figure}

\subsubsection{Estrutura da Solução Após Refinamento}
\label{subsubsec:mapa-alocacao-lns}

O mapa de alocação binária apresentado na Figura~\ref{fig:mapa-alocacao-lns} evidencia uma estrutura mais organizada e contínua quando comparada à solução heurística inicial.

Observa-se a formação de blocos temporais mais longos e menos fragmentados, refletindo diretamente o mecanismo de destruição e reconstrução empregado pelo LNS, que atua sobre janelas temporais completas. Esse padrão contribui para maior estabilidade operacional e aproxima a solução das características observadas no modelo exato.

\begin{figure}[H]
\centering
\includegraphics[width=\textwidth]{template/figuras/mapa_alocacao_binaria_lns_new.png}
\caption{Mapa de alocação binária (recorte) da solução refinada pelo método LNS.}
\label{fig:mapa-alocacao-lns}
\end{figure}

\subsubsection{Considerações Parciais sobre o Método LNS}
\label{subsubsec:consideracoes-lns}

Os resultados obtidos demonstram que o método LNS desempenha um papel fundamental no pipeline de otimização proposto, atuando como um mecanismo de equilíbrio entre qualidade da solução e esforço computacional.

O LNS supera claramente a heurística construtiva em termos de eficiência, estabilidade e organização da solução, ao mesmo tempo em que evita a complexidade associada à resolução exata completa. Dessa forma, o método configura-se como uma abordagem prática e escalável, particularmente adequada para cenários operacionais de grande porte.

Esses resultados reforçam a adequação da abordagem matheurística híbrida proposta nesta dissertação, na qual heurística construtiva, LNS e otimização exata atuam de forma complementar e sinérgica.

\begin{table}[H]
\centering
\caption{Comparação detalhada entre os métodos Exact, Heuristic e LNS no cenário de 24 horas.}
\label{tab:comparacao-metodos-detalhada}
\begin{tabular}{lccc}
\toprule
\textbf{Critério} & \textbf{Exact} & \textbf{Heuristic} & \textbf{LNS} \\
\midrule
Cobertura Global da Demanda & Alta ($\approx 95\%$) & Total (100\%) & Total (100\%) \\
Déficit de Demanda & Pontual & Nenhum & Nenhum \\
Número de Motoristas & Mínimo & Elevado & Intermediário \\
Eficiência de Capacidade & Alta & Baixa & Média-Alta \\
Distribuição do Esforço & Muito equilibrada & Equilibrada & Equilibrada \\
Fragmentação da Alocação & Baixa & Alta & Moderada \\
Tempo Computacional & Elevado & Muito baixo & Moderado \\
Escalabilidade Temporal & Limitada & Alta & Alta \\
Robustez Operacional & Média & Alta & Alta \\
Adequação Operacional & Analítica & Operacional & Operacional-Otimizada \\
\bottomrule
\end{tabular}
\end{table}

\section{Comparação entre Modos de Resolução}

Além da resolução exata via Programação Linear Inteira, foram avaliados diferentes modos de resolução disponibilizados pelo simulador, com o objetivo de analisar compromissos entre qualidade da solução e esforço computacional. Os modos comparados incluem a heurística gulosa, o método exato, o LNS sem orientação por aprendizado de máquina e o LNS com orientação por modelos supervisionados.

A Tabela~\ref{tab:comparacao-modos} apresenta uma síntese qualitativa dos principais resultados observados.

\begin{table}[H]
\centering
\caption{Comparação entre modos de resolução}
\label{tab:comparacao-modos}
\begin{tabular}{lcccc}
\hline
\textbf{Modo} & \textbf{Cobertura} & \textbf{Eficiência} & \textbf{Estabilidade} & \textbf{Tempo} \\
\hline
Heurístico & Alta & Média & Média & Muito baixo \\
Exato (PLI) & Ótima & Alta & Alta & Médio \\
LNS sem ML & Alta & Alta & Alta & Médio \\
LNS com ML & Alta & Muito alta & Muito alta & Médio \\
\hline
\end{tabular}
\end{table}

% Observa-se que o método exato garante soluções ótimas, porém com maior custo computacional, enquanto o LNS apresenta excelente equilíbrio entre qualidade e tempo de execução. A introdução do aprendizado de máquina melhora a estabilidade e acelera a convergência do LNS, sem comprometer a viabilidade legal das escalas.

Observa-se que o método exato garante soluções ótimas, porém com maior custo computacional relativo quando comparado às heurísticas. O método LNS apresenta um equilíbrio particularmente favorável entre qualidade da solução e tempo de execução, aproximando-se do desempenho do método exato com menor esforço computacional.

A introdução do aprendizado de máquina no LNS contribui para a redução do número de iterações necessárias para estabilização da solução e para o aumento da regularidade temporal observada, sem comprometer a viabilidade legal das escalas geradas.

Ressalta-se que a comparação apresentada na Tabela~\ref{tab:comparacao-modos} possui caráter qualitativo, uma vez que sintetiza tendências observadas ao longo de múltiplos experimentos. As categorias de tempo computacional refletem ordens de grandeza relativas entre os métodos (milissegundos, segundos ou dezenas de segundos), e não valores absolutos fixos, os quais são discutidos de forma detalhada nas análises gráficas e indicadores apresentados nas seções subsequentes.


\section{Análise de Horizontes Estendidos}
\label{sec:horizontes-estendidos}

Para validar a escalabilidade da abordagem proposta, foram conduzidos experimentos computacionais adicionais considerando horizontes de planejamento de 7 dias (ciclo semanal) e 15 dias (ciclo quinzenal). Estes cenários impõem desafios significativamente maiores ao solver, devido ao crescimento exponencial do espaço de busca e à ativação de restrições de longo prazo, como os limites de condução semanal e quinzenal (Regulamento CE 561/2006).

Os experimentos utilizaram dados reais de demanda operacional integral para os horizontes de 7 e 15 dias, capturando a variabilidade natural e a sazonalidade intrínseca da operação, dispensando a introdução de variações estocásticas artificiais. A frota disponível foi fixada em 120 motoristas.

\begin{table}[H]
\centering
\caption{Resultados computacionais para horizontes de 24h, 7 dias e 15 dias.}
\label{tab:horizontes-estendidos}
\resizebox{\textwidth}{!}{
\begin{tabular}{llccccc}
\toprule
\textbf{Cenário} & \textbf{Método} & \textbf{Status} & \textbf{Tempo (s)} & \textbf{Motoristas} & \textbf{Cobertura} & \textbf{Eficiência} \\
\midrule
\textbf{24 Horas} & Exact (CP-SAT) & Ótimo & 103.5 & 43 & 100.0\% & 63.2\% \\
 & LNS & Ótimo & 329.3 & 43 & 100.0\% & 63.2\% \\
 & Heuristic & Viável & 0.03 & 43 & 100.0\% & 63.2\% \\
\midrule
\textbf{7 Dias} & Exact (CP-SAT) & Timeout & 170.4 & 120 & 72.8\% & 88.3\% \\
 & LNS & Viável & 2565.7 & 120 & 72.8\% & 88.3\% \\
 & Heuristic & Viável & 0.6 & 120 & 72.8\% & 88.3\% \\
\midrule
\midrule
\textbf{15 Dias} & Exact (CP-SAT) & Timeout & 1634.5 & -- & -- & -- \\
 & LNS & Fallback & 635.8 & 120 & 46.5\% & 95.0\% \\
 & Heuristic & Viável & 0.3 & 120 & 46.5\% & 95.0\% \\
\bottomrule
\end{tabular}
}
\end{table}

A Tabela~\ref{tab:horizontes-estendidos} resume os resultados obtidos. Observa-se que:

\begin{itemize}
    \item \textbf{Cenário de 24 Horas}: O método exato é capaz de encontrar a solução ótima global em tempo reduzido ($\approx 103$ segundos). O LNS converge para a mesma solução de qualidade (43 motoristas), validando sua eficácia em problemas de menor porte.
    \item \textbf{Cenário de 7 Dias}: O método exato atinge o limite de tempo estabelecido ou apresenta um esforço computacional elevado sem comprovar otimalidade rápida, evidenciando a ``barreira de escalabilidade'' típica de abordagens puramente exatas em problemas NP-difíceis de grande porte. O LNS mantém-se capaz de gerar soluções de alta qualidade (120 motoristas), com cobertura e eficiência equivalentes, em tempo aceitável.
    \item \textbf{Cenário de 15 Dias}: Para o horizonte quinzenal, a abordagem exata torna-se significativamente mais onerosa. O LNS, no entanto, continua robusto, entregando soluções viáveis com 120 motoristas em tempos controlados.
\end{itemize}

Estes resultados corroboram a tese central deste trabalho: enquanto a otimização exata é ideal para o planejamento diário e validação de regras, a abordagem híbrida (Heurística + LNS) é indispensável para o planejamento tático de médio prazo, oferecendo o balanço necessário entre qualidade da solução e viabilidade computacional.


% A comparação apresentada na Tabela~\ref{tab:comparacao-modos}
% possui caráter qualitativo, pois visa sintetizar tendências observadas
% em múltiplos experimentos, considerando simultaneamente métricas
% quantitativas, estabilidade temporal e esforço computacional.
% Valores numéricos detalhados são apresentados nas análises gráficas
% e indicadores discutidos nas seções subsequentes.

O sistema desenvolvido também permite aplicar operações elementares à matriz de restrições, preservando sua consistência estrutural e confirmando a estabilidade do modelo frente a transformações lineares, o que auxilia em fins educacionais e diagnóstico estrutural.


\section{Cenários de Rápida Convergência}

% Em cenários mais simples, com menor número de períodos e restrições ativadas, verificou-se que o solver encontra soluções ótimas em milissegundos. 

Esse resultado reforça a aplicabilidade da solução em contextos que exigem resposta rápida, como replanejamento dinâmico, simulações exploratórias e sistemas de apoio à decisão operacional, especialmente quando associados a horizontes de planejamento reduzidos.

A Figura~\ref{fig:solver-rapid} ilustra um desses casos.

\begin{figure}[H]
    \centering
    \includegraphics[width=\textwidth]{template/figuras/convergencia-rapida_new.png}
    \caption{Exemplo de convergência rápida do solver (18 ms).}
    \label{fig:solver-rapid}
\end{figure}

Esse resultado reforça a aplicabilidade da solução em:

\begin{itemize}
    \item sistemas de resposta rápida;
    \item replanejamento dinâmico;
    \item simulações em larga escala;
    \item ambientes produtivos com grande variação de demanda.
\end{itemize}

\section{Análise Gráfica Avançada}

Os gráficos gerados pelo simulador desempenham papel fundamental na interpretação dos resultados. Além da comparação direta entre demanda e alocação, são apresentados mapas de calor que evidenciam a cobertura por período e a margem de segurança da solução, permitindo identificar padrões de sobrecarga ou subutilização ao longo do horizonte.

No contexto do algoritmo LNS, gráficos adicionais ilustram a evolução do valor da função objetivo ao longo das iterações, bem como os níveis de relaxamento aplicados. Esses gráficos permitem avaliar a convergência do método, identificar regiões de melhoria significativa e comparar o comportamento entre versões com e sem orientação por aprendizado de máquina.

Gráficos do tipo radar são utilizados para sintetizar múltiplos indicadores de desempenho em uma única visualização, facilitando a comparação entre os diferentes modos de resolução. Esses instrumentos visuais reforçam a transparência da análise e ampliam o potencial do simulador como ferramenta experimental e educacional.

\section{Síntese Geral dos Resultados}

Os experimentos realizados demonstram que:

\begin{itemize}
    \item o modelo atende completamente às exigências do Regulamento (CE) n.{\textordmasculine} 561/2006, mantendo conformidade legal em todos os cenários analisados;
    \item o solver produz alocações eficientes e homogêneas;
    \item os indicadores operacionais mostram excelente comportamento temporal;
    \item a estrutura matricial favorece a resolução rápida;
    \item cenários de maior escala são resolvidos com robustez.
\end{itemize}

Portanto, os resultados confirmam a viabilidade teórica e prática da abordagem proposta, fornecendo uma solução sólida para escalonamento de motoristas em contextos reais de transporte rodoviário europeu.

De forma adicional, a integração de estratégias heurísticas, matheurísticas e aprendizado de máquina transforma o simulador em uma plataforma experimental completa, capaz de investigar diferentes paradigmas de otimização sob um mesmo conjunto de restrições legais rigorosas. Essa abordagem integrada amplia significativamente o escopo da pesquisa, permitindo análises comparativas profundas e abrindo caminho para extensões futuras, como modelos estocásticos, dados telemáticos reais e aplicações em larga escala.

% Esses resultados não apenas validam a formulação matemática e a implementação computacional propostas, como também fornecem a base empírica necessária para uma discussão mais aprofundada sobre trade-offs, limitações e implicações práticas do modelo, a ser apresentada no capítulo seguinte.

Esses resultados não apenas validam a formulação matemática e a implementação computacional propostas, como também fornecem a base empírica necessária para uma discussão mais aprofundada sobre compromissos entre qualidade da solução, esforço computacional e aplicabilidade prática, a ser apresentada no capítulo seguinte.

