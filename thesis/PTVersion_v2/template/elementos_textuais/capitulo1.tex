\chapter{Introdução}\label{capitulo1}

O transporte rodoviário de cargas é um dos pilares fundamentais da logística moderna, sendo responsável por grande parte da movimentação de mercadorias em diversas regiões do mundo, especialmente na União Europeia. A eficiência desse modal depende diretamente da gestão adequada de seus recursos, em particular, da força de trabalho dos motoristas.

Nesse contexto, o escalonamento de motoristas sob restrições regulatórias emerge como um problema de otimização combinatória de elevada complexidade, com impacto direto na eficiência operacional, nos custos e na conformidade legal das empresas de transporte.

%\addcontentsline{toc}{chapter}{Elementos Pré-Textuais}

\section{Contextualização}

O transporte rodoviário de cargas constitui um componente central da economia europeia, assegurando fluxos contínuos de bens e garantindo a integração logística entre os Estados-Membros. Nesse contexto, os motoristas profissionais constituem o recurso humano central da operação, tornando a gestão das suas jornadas um fator crítico para a eficiência, segurança e conformidade legal do setor.

Nos últimos anos, abordagens híbridas que combinam otimização exata, heurísticas e, de forma complementar, técnicas de aprendizado de máquina têm sido exploradas como forma de acelerar a convergência e melhorar a qualidade das soluções iniciais, especialmente em problemas de grande escala e natureza temporal.

A União Europeia possui uma das regulamentações mais rigorosas do mundo no que diz respeito aos tempos de condução, pausas e períodos de descanso dos motoristas. O \textit{Regulamento (CE) n.{\textordmasculine} 561/2006}, complementado pela \textit{Diretiva 2002/15/CE} e pelo \textit{Regulamento (UE) n.{\textordmasculine} 165/2014}, estabelece limites estritos que visam proteger a sa{\'u}de dos motoristas, prevenir acidentes e promover condi{\c c}{\~o}es de trabalho seguras e humanizadas \cite{eu5612006, eu200215, eu1652014}. Essas normas afetam diretamente a forma como as escalas s{\~a}o constru{\'\i}das, exigindo aten{\c c}{\~a}o constante ao cumprimento de requisitos di{\'a}rios, semanais e quinzenais.

Entretanto, na pr{\'a}tica, a elabora{\c c}{\~a}o manual de escalas {\'e} propensa a erros, especialmente em opera{\c c}{\~o}es de grande escala, caracterizadas por alta variabilidade de demanda, m{\'u}ltiplas janelas temporais, restri{\c c}{\~o}es acumulativas e depend{\^e}ncias temporais entre os per{\'\i}odos de trabalho. Pequenas decis{\~o}es tomadas em um per{\'\i}odo podem invalidar toda a escala futura, resultando em infra{\c c}{\~o}es legais, custos adicionais e riscos operacionais.

Nesse cen{\'a}rio, abordagens matem{\'a}ticas baseadas em Programa{\c c}{\~a}o Linear Inteira (PLI) emergem como ferramentas adequadas para formalizar e resolver problemas de escalonamento, oferecendo rigor, precis{\~a}o e capacidade de lidar com m{\'u}ltiplas restri{\c c}{\~o}es simultaneamente \cite{hillier2015, taha2017, nemhauser1988}. De forma complementar, solvers modernos como o OR-Tools disponibilizam recursos computacionais avan{\c c}ados que permitem resolver modelos inteiros de grande escala em tempos reduzidos \cite{googleortools}.

Além do modelo de otimização em si, esta pesquisa materializa um ambiente de simulação e análise, no qual o usuário pode configurar parâmetros operacionais, habilitar/desabilitar restrições legais e analisar, de forma sistemática, como tais escolhas impactam a viabilidade, o número de motoristas necessários e a qualidade da cobertura da demanda. Esse ambiente inclui métricas e gráficos de diagnóstico (por exemplo, cobertura por slot, subutilização, sobrecarga, risco e estabilidade temporal), bem como análises estruturais do modelo por meio de visualizações de matrizes e medidas como densidade, oferecendo suporte à interpretação do comportamento do solver e à validação do modelo.


Esta disserta{\c c}{\~a}o insere-se nesse contexto, propondo um modelo abrangente de PLI para o escalonamento de motoristas em conformidade com o Regulamento (CE) n.{\textordmasculine} 561/2006, apoiado por uma implementa{\c c}{\~a}o computacional interativa e orientada {\`a} simula{\c c}{\~a}o de cen{\'a}rios operacionais. Parte desse desenvolvimento foi previamente publicada em \cite{moreira2025}, que fundamenta e inspira a formula{\c c}{\~a}o aprofundada apresentada nesta pesquisa.

\section{Problema da Pesquisa}

O problema desta disserta{\c c}{\~a}o consiste em determinar como alocar motoristas ao longo de um horizonte temporal discretizado, de forma a: (i) atender {\`a} demanda operacional por per{\'\i}odo; (ii) respeitar o Regulamento (CE) n.{\textordmasculine} 561/2006; (iii) minimizar o n{\'u}mero total de motoristas utilizados;
(iv) garantir escalas cont{\'\i}nuas, regulares e operacionalmente vi{\'a}veis; (v) permitir an{\'a}lises de sensibilidade para diferentes cen{\'a}rios.

Adicionalmente, quando o modelo exato se torna computacionalmente custoso em instâncias maiores, torna-se relevante investigar estratégias de busca e melhoria incremental de soluções, incorporadas no próprio simulador, tais como heurísticas construtivas e busca em grandes vizinhanças, preservando a conformidade legal e permitindo análises comparativas entre modos de resolução.

Trata-se de um contexto caracterizado pelas restrições acumulativas sobre condução diária, semanal e quinzenal; pela necessidade de pausas obrigatórias; pela variabilidade da demanda; pelas dependências temporais entre períodos sucessivos; e pelo grande número de combinações possíveis entre motoristas e períodos.

\section{Quest{\~a}o da Pesquisa}

\begin{quote}
Como formular e implementar uma abordagem de otimização matemática capaz de gerar escalas de motoristas legalmente v{\'a}lidas, operacionalmente adequadas e alinhadas ao Regulamento (CE) n.{\textordmasculine} 561/2006?
\end{quote}

\section{Objetivo Geral}

Desenvolver, implementar e avaliar um sistema de otimização matemática para o escalonamento de motoristas no transporte rodovi{\'a}rio europeu, assegurando conformidade com o Regulamento (CE) n.{\textordmasculine} 561/2006 e promovendo melhoria do desempenho operacional.

\section{Objetivos Espec{\'\i}ficos}

(i) Formular matematicamente o problema, definindo vari{\'a}veis, par{\^a}metros e restri{\c c}{\~o}es; (ii) Representar as exig{\^e}ncias legais do Regulamento (CE) n.{\textordmasculine} 561/2006 sob a forma de inequa{\c c}{\~o}es lineares; (iii) Implementar o modelo utilizando Python, o solver moderno CP-SAT e uma arquitetura híbrida; (iv) Desenvolver uma interface interativa para simula{\c c}{\~a}o de cen{\'a}rios e visualiza{\c c}{\~a}o de resultados; (v) Validar o modelo em diferentes horizontes (7 e 15 dias); (vi) Comparar a aloca{\c c}{\~a}o gerada com a demanda real ou simulada; (vii) Avaliar m{\'e}tricas de desempenho, como cobertura, subutiliza{\c c}{\~a}o e sobrecarga; (viii) Analisar a estabilidade do modelo e o impacto das transforma{\c c}{\~o}es matriciais. (ix) Implementar e avaliar modos complementares de resolução e melhoria de soluções (por exemplo, geração de solução inicial e busca em vizinhanças), bem como mecanismos de relaxação e registro de histórico de convergência; (x) Estruturar rotinas de experimentação com geração de instâncias/datasets e comparação sistemática de desempenho por meio de métricas e visualizações no simulador.

\section{Justificativa}

(i) \textbf{Relev{\^a}ncia legal e seguran{\c c}a rodovi{\'a}ria}: o descumprimento do Regulamento (CE) n.{\textordmasculine} 561/2006 implica multas, riscos {\`a} vida dos motoristas e impactos negativos na reputa{\c c}{\~a}o das empresas; (ii) \textbf{Complexidade operacional}: a variabilidade de demanda e a natureza acumulativa das restri{\c c}{\~o}es dificultam a elabora{\c c}{\~a}o manual de escalas; (iii) \textbf{Contribuição científica}: poucos trabalhos integram o Regulamento (CE) n.{\textordmasculine} 561/2006 a modelos inteiros formais; esta disserta{\c c}{\~a}o contribui para preencher essa lacuna, em linha com \cite{erdman2022, moreira2025}; (iv) \textbf{Aplicabilidade industrial}: o modelo possui potencial de aplicação em empresas de transporte, podendo evoluir para solu{\c c}{\~o}es SaaS e integra{\c c}{\~a}o com sistemas telem{\'a}ticos.

\section{Estrutura da Dissertação}

Esta dissertação está organizada em oito capítulos, além das referências, apêndices e anexos, estruturados de forma a conduzir o leitor desde a contextualização do problema até a consolidação das contribuições científicas e perspectivas futuras.

O \textbf{Capítulo 1} apresenta a introdução do trabalho, incluindo a contextualização do problema, a definição da questão de pesquisa, os objetivos geral e específicos, a justificativa do estudo e a descrição da estrutura da dissertação.

O \textbf{Capítulo 2} é dedicado à revisão da literatura, abordando os fundamentos do transporte rodoviário europeu, as restrições regulatórias aplicáveis, os principais problemas de escalonamento, técnicas de modelagem com Programação Linear Inteira, metaheurísticas e o uso de solvers modernos como o OR-Tools CP-SAT.

O \textbf{Capítulo 3} apresenta as bases legais e normativas do escalonamento de motoristas, com uma análise detalhada do Regulamento (CE) n.º 561/2006, discutindo suas implicações diretas na formulação do modelo matemático.

O \textbf{Capítulo 4} detalha a Metodologia e o Modelo Matemático proposto. Abrange a discretização temporal, a definição de conjuntos, parâmetros e variáveis de decisão, bem como as funções objetivo e o conjunto completo de restrições legais. Introduz também a arquitetura híbrida que combina otimização exata, heurísticas, Large Neighborhood Search (LNS) e integração com aprendizado de máquina.

O \textbf{Capítulo 5} trata da implementação computacional, descrevendo o simulador interativo desenvolvido, a interface de parametrização, os modos de resolução, os fluxos computacionais e as funcionalidades avançadas, como o Benchmark Mode e os indicadores de esforço operacional (Gini/Lorenz).

O \textbf{Capítulo 6} apresenta os resultados experimentais, incluindo a análise do comportamento da demanda, a comparação entre modos de resolução (Exato, Heurístico e LNS) e a validação dos indicadores operacionais e computacionais.

O \textbf{Capítulo 7} realiza uma análise comparativa da literatura, confrontando os modelos e técnicas desenvolvidos neste trabalho com o estado da arte e discutindo o pipeline de otimização proposto.

Finalmente, o \textbf{Capítulo 8} apresenta as conclusões do trabalho, sintetizando as contribuições, os resultados alcançados, as limitações identificadas e as propostas para trabalhos futuros.

As referências bibliográficas, o glossário, os apêndices e os anexos complementam o trabalho, fornecendo detalhamentos técnicos adicionais e documentação normativa.
