\chapter{Conclusões e Trabalhos Futuros}
\label{cap:conclusoes}

\section{Considerações Iniciais}
\label{sec:conclusoes_iniciais}

O problema central abordado nesta dissertação é a complexidade combinatória do escalonamento de motoristas sob a ótica do Regulamento (CE) n.º 561/2006. As transportadoras enfrentam o desafio de atender uma demanda variável em alta resolução temporal, enquanto estão restritas a uma "teia" de exigências legais interdependentes que punem severamente a falta de conformidade. Identificou-se que a literatura frequentemente simplifica essas regras ou utiliza granulosidades temporais grosseiras, criando um hiato entre a teoria e a realidade operacional.

Este capítulo apresenta uma síntese integrada do trabalho desenvolvido, consolida os principais resultados obtidos, valida as contribuições científicas anunciadas no Capítulo~\ref{cap:revisao} e reflete sobre os limites da escalabilidade e o impacto prático da abordagem híbrida proposta.

\section{Síntese do Trabalho Desenvolvido}
\label{sec:sintese_trabalho}

A pesquisa evoluiu de uma formulação matemática clássica de escalonamento para o desenvolvimento de uma plataforma computacional híbrida, capaz de operar em diferentes modos de resolução e de oferecer um ambiente experimental completo para análise, validação e comparação de estratégias de otimização.

Inicialmente, foi proposta uma formulação completa de Programação Linear Inteira para o escalonamento de motoristas, com discretização temporal em períodos de 15 minutos, permitindo representar com precisão as regras de condução contínua, pausas obrigatórias, descansos diários normais e reduzidos, limites semanais e quinzenais. Essa granularidade viabilizou a modelagem explícita de janelas temporais móveis e dependências acumulativas, frequentemente simplificadas ou tratadas de forma aproximada na literatura.

Na sequência, o modelo matemático foi implementado computacionalmente utilizando Python e o solver CP-SAT do OR-Tools, explorando sua capacidade de lidar com grandes conjuntos de variáveis binárias e restrições fortemente acopladas no tempo. A implementação foi estruturada de forma modular, separando claramente as camadas de interface, modelagem, solver e pós-processamento.

O trabalho avançou ainda mais ao estender o simulador para além da otimização exata, incorporando heurísticas construtivas, métodos matheurísticos baseados em \textit{Large Neighborhood Search} (LNS) e mecanismos experimentais de aprendizado de máquina supervisionado. Essa evolução transformou o sistema em uma plataforma híbrida de experimentação, capaz de explorar compromissos entre qualidade da solução, esforço computacional e estabilidade temporal.

\section{Principais Resultados e Evidências Empíricas}
\label{sec:resultados_empiricos}

Os resultados experimentais obtidos ao longo do trabalho evidenciam a robustez e a aplicabilidade da abordagem proposta. Em cenários com horizontes de planejamento variando entre 7, 15 e 30 dias, o modelo foi capaz de gerar escalas plenamente conformes à legislação europeia, mantendo elevada aderência entre demanda operacional e cobertura efetiva.

Os indicadores operacionais analisados mostraram cobertura média superior a 99\%, com níveis mínimos de sobrecarga e subutilização, concentrados predominantemente em períodos de baixa demanda. O comportamento temporal das soluções apresentou elevada estabilidade, sem oscilações abruptas ou padrões erráticos, fator essencial para a aceitação prática em ambientes operacionais reais.

A integração opcional de aprendizado de máquina atuou como camada de \textit{guidance}, auxiliando decisões locais e estratégicas sem comprometer a estabilidade, a reprodutibilidade ou a validade legal das soluções.

Em suma, este trabalho prova que: (i) é possível alcançar conformidade legal plena (100\%) em modelos PLI de alta fidelidade sem sacrificar a viabilidade computacional; (ii) a arquitetura híbrida (MILP + LNS + Heurística) supera as limitações de solvers exatos tradicionais em horizontes de médio prazo; e (iii) a automação do escalonamento permite que as transportadoras unam eficiência máxima de recursos com a preservação rigorosa da segurança e saúde do motorista.

\section{Validação das Contribuições Científicas}
\label{sec:validacao_contribuicoes}

As contribuições científicas anunciadas no Capítulo~\ref{cap:revisao}, Seção~\ref{sec:posicionamento}, foram sistematicamente validadas ao longo dos experimentos apresentados no Capítulo~\ref{cap:resultados}:

\textbf{Contribuição 1 (Modelagem):} A formulação PLI completa do Regulamento (CE) n.º 561/2006 foi validada por meio da obtenção de 100\% de conformidade legal em todos os cenários testados, confirmando que a granularidade temporal de 15 minutos permite representar adequadamente janelas móveis e dependências acumulativas.

\textbf{Contribuição 2 (Algorítmica):} A heurística construtiva matricial e o método LNS guiado por densidade foram validados experimentalmente, demonstrando capacidade de gerar soluções viáveis rapidamente (modo heurístico) e de melhorar progressivamente a qualidade da solução (modo LNS) com reduções de até 40\% no tempo computacional em comparação com o método exato.

\textbf{Contribuição 3 (Análise Estrutural):} A aplicação de operações elementares sobre a matriz de restrições permitiu análises de densidade e esparsidade que auxiliaram na compreensão do comportamento do solver e na validação da estabilidade numérica do modelo.

\textbf{Contribuição 4 (Plataforma Experimental):} A plataforma híbrida desenvolvida demonstrou capacidade de operar em três modos de resolução distintos, permitindo comparações sistemáticas e reprodutíveis, conforme evidenciado nas Tabelas~\ref{tab:comparacao_metodos} (Capítulo~\ref{cap:metodologia}).

\textbf{Contribuição 5 (Validação Prática):} Os tempos de resolução inferiores a 2 segundos para cenários de 24 horas confirmam a viabilidade de aplicações em contextos reais de replanejamento dinâmico e sistemas de apoio à decisão operacional.

\section{Limitações Identificadas}
\label{sec:limitacoes}

Apesar dos resultados positivos, algumas limitações devem ser reconhecidas. A demanda operacional foi tratada como determinística, não contemplando incertezas associadas a atrasos, ausências, eventos imprevistos ou flutuações estocásticas. Em ambientes reais, tais fatores podem impactar significativamente a viabilidade das escalas.

O modelo também não incorpora explicitamente aspectos espaciais, como roteamento, distâncias ou tempos de deslocamento, focando exclusivamente no escalonamento temporal. Embora essa separação seja metodologicamente válida, uma integração mais profunda com problemas de roteamento poderia ampliar ainda mais a aplicabilidade prática.

No que se refere ao aprendizado de máquina, sua avaliação quantitativa ainda é limitada a cenários experimentais controlados. Estudos mais amplos, com maior diversidade de instâncias e validação estatística rigorosa, são necessários para quantificar de forma conclusiva seus ganhos.

Adicionalmente, embora o simulador tenha sido validado com dados reais anonimizados, a integração completa com sistemas telemáticos em produção (tacógrafos digitais, TMS, FMS) ainda não foi realizada, constituindo uma etapa natural para trabalhos futuros.

\section{Direções para Trabalhos Futuros}
\label{sec:trabalhos_futuros}

Diversas extensões naturais podem ser exploradas a partir desta pesquisa:

\textbf{Modelos Estocásticos e Robustos:} Incorporação de incertezas operacionais por meio de programação estocástica ou otimização robusta, permitindo gerar escalas mais resilientes a perturbações e variações de demanda não previstas.

\textbf{Integração com Roteamento de Veículos:} Desenvolvimento de formulações conjuntas do tipo \textit{Driver Scheduling + Vehicle Routing Problem} (DS+VRP), combinando decisões temporais de escalonamento com decisões espaciais de roteamento em um modelo unificado.

\textbf{Integração com Dados Telemáticos Reais:} Conexão direta com tacógrafos digitais e sistemas de gestão de frotas (TMS/FMS) para validação empírica em ambientes produtivos e calibração do modelo com dados operacionais em tempo real.

\textbf{Expansão do Módulo de Aprendizado de Máquina:} Investigação de técnicas mais avançadas, como aprendizado por reforço, aprendizado profundo ou modelos híbridos supervisionados-não supervisionados para orientação de decisões heurísticas e matheurísticas.

\textbf{Plataforma SaaS Multiempresa:} Evolução do simulador para uma plataforma \textit{Software as a Service} (SaaS) voltada ao mercado de transporte rodoviário, com interface web, APIs REST, autenticação multiusuário e conformidade com GDPR.

\textbf{Extensão para Outras Regulamentações:} Adaptação do modelo para regulamentações de outros países ou regiões (Brasil, EUA, Ásia-Pacífico), permitindo comparações internacionais e ampliando o escopo de aplicação da plataforma.

\textbf{Análises de Sensibilidade Avançadas:} Estudos sistemáticos sobre o impacto de variações nos parâmetros regulatórios (por exemplo, redução de limites diários ou aumento de pausas obrigatórias) sobre a viabilidade operacional e custos logísticos.

\section{Considerações Finais}
\label{sec:consideracoes_finais}

Em síntese, esta dissertação demonstra que a combinação coerente de Programação Linear Inteira, heurísticas, matheurísticas e aprendizado de máquina constitui uma abordagem poderosa para o escalonamento de motoristas sob restrições legais complexas. O simulador desenvolvido estabelece uma base sólida tanto para aplicações práticas quanto para investigações científicas futuras, contribuindo de forma significativa para o avanço da Pesquisa Operacional aplicada ao transporte rodoviário europeu.

As contribuições científicas consolidadas neste trabalho, validadas empiricamente por meio de experimentos sistemáticos, posicionam esta pesquisa como uma referência relevante no estado da arte do escalonamento regulado, abrindo caminhos para extensões acadêmicas e aplicações industriais que podem impactar positivamente a eficiência, segurança e conformidade legal do setor de transporte rodoviário de cargas.
