\chapter{Modelagem e Metodologia Experimental}\label{cap:metodologia}

\section{Introdução}

Este capítulo apresenta o desenvolvimento completo do modelo de otimização matemática, formulado com restrições lineares inteiras, aplicado ao escalonamento de motoristas sob o Regulamento (CE) n.{\textordmasculine} 561/2006, integrando sua formulação com a metodologia experimental adotada para validação e análise de desempenho. Embora formulado como um problema linear inteiro, a resolução computacional beneficia-se da tecnologia \textit{Constraint Programming} (CP-SAT) e de técnicas híbridas para lidar eficientemente com a natureza combinatória das restrições.

A metodologia combina rigor matemático, técnicas de Pesquisa Operacional e ferramentas computacionais modernas, garantindo reprodutibilidade, precisão e aplicabilidade prática.  A pesquisa pode ser classificada como:
\begin{itemize}
    \item \textbf{Aplicada}: pois aborda um problema concreto do setor de transporte rodoviário.
    \item \textbf{Quantitativa}: devido ao uso de modelagem matemática e experimentação.
    \item \textbf{Metodológica}: por propor um artefato científico (o modelo ILP).
    \item \textbf{Experimental}: pela realização de testes controlados em cen{\'a}rios variados.
\end{itemize}

Essa abordagem está alinhada aos fundamentos apresentados em \cite{hillier2015, taha2017, nemhauser1988}, bem como ao trabalho aplicado apresentado em \cite{moreira2025}.


O objetivo do modelo é gerar escalas legalmente válidas, eficientes e
capazes de atender à demanda operacional, respeitando simultaneamente
limites de condução, pausas obrigatórias e períodos mínimos de descanso.

Formulações baseadas em PLI têm sido amplamente utilizadas para problemas
de escalonamento com fortes dependências temporais, sobretudo em contextos
regulados como transporte, saúde e serviços críticos
\cite{savelsbergh1995, pillac2013}.
Neste trabalho, o modelo proposto estende a formulação apresentada em
\cite{moreira2025}, incorporando maior granularidade temporal, variáveis
auxiliares e uma estrutura de otimização compatível com o solver CP-SAT
do OR-Tools \cite{googleortools}.

Diferentemente de abordagens puramente multiobjetivo, o modelo adota uma
estrutura lexicográfica em duas fases, refletindo prioridades operacionais
reais: primeiro garantir cobertura da demanda e minimizar o número de
motoristas ativos, e somente depois refinar a solução por critérios de
balanceamento e suavização de carga.


\section{Formulação Matemática do Modelo}

Esta seção apresenta a formulação completa do modelo matemático desenvolvido para o problema de escalonamento. Diferentemente de abordagens puramente multiobjetivo, o modelo adota uma estrutura lexicográfica em duas fases, refletindo prioridades operacionais reais: primeiro garantir cobertura da demanda e minimizar o número de motoristas ativos, e somente depois refinar a solução por critérios de balanceamento e suavização de carga.

\section{Discretização Temporal}

O horizonte de planejamento é discretizado em períodos fixos de duração
$\Delta = 15$ minutos, formando o conjunto:
\[
T = \{1,2,\ldots,|T|\}.
\]

Essa discretização reflete a frequência real de chegada dos pedidos de
carga na operação analisada e permite representar de forma linear
restrições originalmente não lineares, como janelas móveis de descanso
e limites acumulados de condução \cite{pillac2013, erdman2022}.

\section{Conjuntos e Índices}

\begin{itemize}
    \item $D$: conjunto de períodos discretizados ($d \in D$);
    \item $T$: conjunto de motoristas ($t \in T$);
    \item $W$: janelas agregadas de planejamento (dias, semanas e quinzenas).
\end{itemize}

\section{Parâmetros}

\begin{align*}
demanda_d      &: \text{demanda operacional no período } d,\\
\Delta         &: \text{duração do período (15 minutos)},\\
L^{dia}        &: \text{limite diário de condução},\\
L^{sem}        &: \text{limite semanal de condução},\\
L^{14d}        &: \text{limite quinzenal de condução},\\
R^{dia}        &: \text{descanso diário mínimo},\\
R^{sem}        &: \text{descanso semanal mínimo}.
\end{align*}

Esses parâmetros refletem diretamente as exigências do Regulamento
(CE) n.º 561/2006 e da Diretiva 2002/15/CE \cite{eu5612006, eu200215}.

\section{Variáveis de Decisão}

\subsection*{Ativação do motorista}
\[
Z_t =
\begin{cases}
1, & \text{se o motorista } t \text{ é ativado},\\
0, & \text{caso contrário}.
\end{cases}
\]

\subsection*{Presença no período}
\[
Y_{d,t} =
\begin{cases}
1, & \text{se o motorista } t \text{ está presente no período } d,\\
0, & \text{caso contrário}.
\end{cases}
\]

\subsection*{Atendimento de demanda}
\[
X_{d,t} \in \mathbb{Z}_{+}, \quad 0 \le X_{d,t} \le \text{capacidade por slot}.
\]

\subsection*{Demanda não atendida}
\[
U_d \ge 0.
\]

\subsection*{Carga por motorista e carga máxima}
\[
L_t = \sum_{d \in D} X_{d,t}, \qquad L_{\max} \ge L_t.
\]

A separação entre variáveis de ativação, presença e carga segue práticas
consolidadas na literatura de escalonamento, permitindo representar de
forma explícita decisões operacionais e restrições legais
\cite{savelsbergh1995, pinheiro2007staffscheduling}.

\section{Função Objetivo Lexicográfica}

O modelo utiliza uma estrutura de otimização em duas fases, resolvidas
sequencialmente pelo solver.

\subsection{Fase 1 — Cobertura e Minimização de Motoristas}

A Fase 1 possui prioridade absoluta e busca:

\begin{enumerate}
    \item minimizar a demanda não atendida;
    \item minimizar o número de motoristas ativados;
    \item incentivar maior utilização dos motoristas ativos.
\end{enumerate}

A função objetivo é dada por:
\[
\min \;
\alpha \sum_{d \in D} U_d
+ \beta \sum_{t \in T} Z_t
- \gamma \sum_{d \in D} \sum_{t \in T} X_{d,t},
\]
com $\alpha \gg \beta \gg \gamma$, garantindo prioridade lexicográfica.

Estruturas desse tipo são amplamente utilizadas quando múltiplos objetivos
conflitantes precisam ser tratados de forma hierárquica
\cite{pillac2013, erdman2022}.

\subsection{Fase 2 — Balanceamento e Refinamento}

Após fixar os valores ótimos de $\sum U_d$ e $\sum Z_t$, a Fase 2 refina
a solução, buscando:

\begin{itemize}
    \item reduzir presença desnecessária;
    \item penalizar excesso de carga acima de um limite suave;
    \item suavizar picos de carga entre motoristas.
\end{itemize}

A função objetivo da Fase 2 é:
\[
\min \;
\omega_1 \sum_{d,t} Y_{d,t}
+ \omega_2 \sum_{t} \max(0, L_t - L^{soft})
+ \omega_3 L_{\max}.
\]

Esses critérios promovem maior equidade e robustez operacional, sendo
frequentemente empregados como etapa de refinamento em problemas de
escalonamento \cite{savelsbergh1995}.

\section{Restrições do Modelo}

O modelo matemático incorpora um conjunto abrangente de restrições
legais e operacionais derivadas do Regulamento (CE) n.º 561/2006.
Entretanto, nem todas as restrições são necessariamente ativadas
simultaneamente em todos os experimentos.

No simulador desenvolvido, cada restrição pode ser ativada ou desativada
de forma independente por meio de parâmetros de configuração,
permitindo a análise controlada do impacto individual e combinado das
regras legais sobre a solução.
Essa abordagem possibilita estudos comparativos entre diferentes níveis
de rigor regulatório, sem alterar a estrutura fundamental do modelo.

Formalmente, o conjunto de restrições ativas em um experimento é definido
por um vetor binário de ativação, sendo que apenas as restrições
selecionadas são incluídas no modelo resolvido pelo solver.

\subsection{Atendimento da Demanda}

Quando a restrição de cobertura obrigatória está ativada, o atendimento
da demanda é modelado como uma restrição rígida:

\[
\sum_{t \in T} X_{d,t} + U_d = demanda_d, \quad \forall d \in D.
\]

Caso essa restrição não seja ativada, a variável $U_d$ passa a representar
déficits operacionais admissíveis, utilizados exclusivamente para fins
de avaliação e cálculo de indicadores, sem inviabilizar a solução.


\subsection{Vinculação entre presença e ativação}
\[
Y_{d,t} \le Z_t, \quad \forall d,t.
\]

\subsection{Capacidade por período}
\[
X_{d,t} \le \text{capacidade} \cdot Y_{d,t}.
\]

\subsection{Limites legais de condução}

As restrições apresentadas a seguir são incluídas no modelo apenas quando
explicitamente ativadas no cenário experimental considerado.

Restrições diárias, semanais e quinzenais são modeladas por janelas móveis,
conforme exigido pelo Regulamento (CE) n.º 561/2006.

\subsection{Pausas e descansos}
Pausas após 4,5 horas de condução contínua e descansos diários e semanais
são representados por restrições lineares baseadas em janelas deslizantes.

\section{Discussão sobre Linearização}

As regras legais apresentam dependências temporais não lineares, tratadas
por meio de discretização, variáveis auxiliares e constantes Big-M.
Essa abordagem é amplamente utilizada para manter a linearidade da
formulação e compatibilidade com solvers inteiros modernos
\cite{pillac2013, googleortools}.

\section{Observações Computacionais}

O modelo apresenta dezenas de milhares de variáveis e restrições, mas
possui matriz esparsa e forte estrutura temporal. Solvers modernos, como
o CP-SAT, exploram essas características de forma eficiente, permitindo
resolver cenários reais em tempos compatíveis com uso operacional
\cite{erdman2022, googleortools}.


\section{Metodologia de Implementação e Experimentação}

Esta seção descreve o processo de implementação computacional do modelo matemático apresentado anteriormente, bem como a estrutura metodológica adotada para sua validação experimental.

\section{Etapa 1: Estudo da Legislação Europeia}

Nesta etapa, foi realizada uma análise minuciosa dos documentos legais que regem a jornada dos motoristas profissionais na Uni{\~a}o Europeia, incluindo:

\begin{itemize}
    \item \textit{Regulamento (CE) n.{\textordmasculine} 561/2006};
    \item \textit{Diretiva 2002/15/CE};
    \item \textit{Regulamento (UE) n.{\textordmasculine} 165/2014}.
\end{itemize}

As restrições extraídas desses documentos foram traduzidas em condições formais, como:
\begin{itemize}
    \item limites de condu{\c c}{\~a}o diária, semanal e quinzenal;
    \item pausas obrigatórias;
    \item descanso diário normal ou reduzido;
    \item descanso semanal;
    \item regras de continuidade temporal.
\end{itemize}

O desafio principal foi converter regras textuais em inequações lineares, conforme discutido em \cite{pillai2019} e explorado experimentalmente em \cite{moreira2025}.

\section{Etapa 2: Formulação Matemática}

Com base nos requisitos legais e operacionais, o problema foi formalizado como um modelo de Programa{\c c}{\~a}o Linear Inteira (PLI). Os principais elementos definidos foram:

\begin{itemize}
    \item conjuntos (motoristas, per{\'\i}odos, janelas temporais);
    \item vari{\'a}veis de decisão (alocação, demanda não atendida, ativação, condução acumulada);
    \item parâmetros (demanda, limites legais, duração dos períodos);
    \item função objetivo (minimização lexicográfica em duas fases: (1) cobertura de demanda e (2) eficiência de recursos);
    \item restrições (legais e operacionais).
\end{itemize}

A função objetivo é estruturada de forma lexicográfica, priorizando a confiabilidade operacional antes da eficiência de custos. O critério primário consiste na minimização da demanda não atendida ($\sum U_d$) ao longo do horizonte de planejamento. Uma vez garantida a máxima cobertura possível, o modelo avança para o critério secundário: a minimização do número de motoristas ativos ($\sum Z_t$). Em caso de empate técnico, adota-se um terceiro critério de estabilidade, favorecendo a distribuição equilibrada das tarefas. Essa hierarquização reflete a realidade do transporte rodoviário, onde o descumprimento de prazos possui impacto reputacional e contratual superior ao custo marginal de ativação de recursos.

A formulação foi construída com base em técnicas clássicas de otimização combinatoria \cite{nemhauser1988}, \cite{dantzig1963}, mas adaptada ao contexto legislativo europeu. O modelo final é apresentado em detalhes no Capítulo~4, incluindo linearizações e implicações lógicas.

\section{Etapa 3: Implementação Computacional}

A implementação foi realizada em Python, com uso do solver CP-SAT do s \cite{googleortools}. Essa etapa envolveu:

\begin{itemize}
    \item criação programática das vari{\'a}veis do modelo;
    \item construção da matriz de restrições;
    \item aplicação das regras legais e dos vínculos sequenciais;
    \item configuração de parâmetros do solver (limites de tempo, cortes, heur{\'\i}sticas);
    \item desenvolvimento de uma interface interativa em Streamlit;
    \item geração automática de relat{\'o}rios, gráficos e indicadores operacionais.
\end{itemize}

A linguagem Python facilitou o uso de estruturas de dados eficientes, como arrays NumPy \cite{numpydocs} e tabelas Pandas \cite{pandasdocs}, além de permitir modularidade na construção do modelo.

\section{Etapa 4: Definição dos Cenários Experimentais}

Quatro tipos de cen{\'a}rio foram definidos para validação:

\begin{itemize}
    \item \textbf{24 horas}: cenário básico, discretizado em 96 períodos de 15 minutos, utilizado para validação das restrições diárias de condução, pausas e início de jornada.
    \item \textbf{7 dias}: cenário compacto, útil para validar a recorrência das restrições diárias ao longo de múltiplos ciclos consecutivos.
    \item \textbf{15 dias}: cenário intermediário, crítico para a avaliação de restrições quinzenais e efeitos acumulativos de carga de trabalho.
    \item \textbf{30 dias}: cenário de escala real, empregado para testar a robustez temporal do modelo e sua aplicabilidade em horizontes de planejamento prolongados.
\end{itemize}

% Cada cen{\'a}rio utilizou uma curva de demanda variável, simulando altas e baixas operacionais típicas do setor.

Os dados utilizados na definição das curvas de demanda e dos cenários experimentais são oriundos de registros operacionais reais de uma empresa do setor de transporte rodoviário. Por razões de confidencialidade, os dados foram previamente anonimizados e agregados, preservando-se apenas as características temporais e estruturais necessárias à modelagem.

A granularidade temporal adotada, correspondente a períodos de 15 minutos, reflete a frequência com que pedidos de carga são efetivamente recebidos e registrados nos sistemas operacionais da empresa. Cada período representa, portanto, um intervalo de tempo no qual novas demandas podem surgir e ser contabilizadas no respectivo \emph{slot} temporal, permitindo capturar de forma fiel a dinâmica operacional observada na prática.


Parâmetros como duração de períodos, restrições legais e tamanho da força de trabalho foram configuráveis via interface.

Embora múltiplos cenários experimentais tenham sido avaliados, a apresentação gráfica detalhada de todos os casos resultaria em um volume excessivo de artefatos visuais, sem ganho proporcional de poder explicativo. Dessa forma, optou-se por apresentar de forma detalhada apenas o cenário de 24 horas, que permite uma análise clara e intuitiva do comportamento do modelo e dos métodos de resolução. Os demais cenários são analisados por meio de métricas agregadas e comparativas, evidenciando a consistência e a escalabilidade dos resultados.


\section{Etapa 5: Análise dos Resultados}

Após a execução do solver, foram analisados:

% \begin{itemize}
%     \item conformidade legal (restrições diárias, semanais e quinzenais);
%     \item cobertura da demanda por período;
%     \item sobrecarga e subutilização;
%     \item estabilidade temporal da alocação;
%     \item tempo computacional e qualidade da solução;
%     \item impacto da estrutura matricial no desempenho do solver.
% \end{itemize}

% A avaliação baseou-se em indicadores estatísticos e gráficos, discutidos no Capítulo~6.

% \section{Performance and Robustness Indicators}

% The following indicators are used to evaluate the quality, robustness, and operational feasibility of the workforce allocation solution.

\begin{description}

\item[\textbf{Estabilidade de Cobertura ($\sigma$)}]  
\textit{Standard deviation of coverage levels across time slots.}  

Mede a variabilidade da cobertura ao longo dos períodos de tempo. Valores baixos indicam uma alocação estável e homogênea, enquanto valores elevados revelam flutuações significativas entre excesso e falta de cobertura, caracterizando menor robustez temporal.

\item[\textbf{Slots com Déficit de Demanda}]  
\textit{Number of time slots where allocated workers are fewer than required demand.}  

Representa a quantidade de períodos em que a demanda não é totalmente atendida. Esse indicador reflete diretamente falhas operacionais e potenciais violações do nível de serviço.

\item[\textbf{Demanda Total Não Atendida}]  
\textit{Total amount of unmet demand aggregated over all time slots.}  

Quantifica o volume total de demanda não coberta ao longo do horizonte de planejamento. Diferentemente do número de slots deficitários, este indicador mede a magnitude acumulada do déficit.

\item[\textbf{Índice Global de Cobertura}]  
\textit{Overall ratio between allocated capacity and required demand.}  

Avalia o nível médio de atendimento do sistema. Valores próximos de $1.0$ indicam equilíbrio entre oferta e demanda; valores inferiores indicam subalocação, enquanto valores superiores sugerem excesso estrutural de capacidade.

\item[\textbf{Gravidade do Risco}]  
\textit{Weighted severity score based on uncovered demand and its distribution over time.}  

Consolida a intensidade, duração e distribuição temporal dos déficits de cobertura em um único índice, permitindo classificar a solução em níveis de risco operacional.

\item[\textbf{Eficiência da Força de Trabalho}]  
\textit{Ratio between productive allocated work and total available workforce capacity.}  

Indica o grau de utilização efetiva da capacidade total dos trabalhadores. Alta eficiência sugere baixo nível de ociosidade, enquanto valores reduzidos indicam superdimensionamento ou má distribuição dos recursos.

\item[\textbf{Custo Estimado}]  
\textit{Estimated operational cost derived from workforce allocation and penalty terms.}  

Representa o custo total da solução, considerando horas trabalhadas, penalidades por déficit ou excesso e demais componentes definidos na função objetivo do modelo.

\item[\textbf{Estabilidade Temporal}]  
\textit{Consistency of workforce allocation patterns over consecutive time slots.}  

Avalia a suavidade das transições na alocação ao longo do tempo. Soluções com alta estabilidade reduzem mudanças abruptas, simplificando a execução operacional e aumentando a previsibilidade.

\item[\textbf{Capacidade em Excesso}]  
\textit{Total amount of allocated capacity exceeding the required demand.}  

Mede o desperdício de recursos decorrente da superalocação. Este indicador é fundamental para analisar o trade-off entre robustez da solução e eficiência econômica.

\end{description}



\subsection{Abordagem heurística e matheurística proposta}

O simulador desenvolvido permite a execução do modelo de escalonamento por meio de três modos distintos de resolução, selecionáveis pelo usuário: \textit{Exato}, \textit{Heurístico} e \textit{Large Neighborhood Search (LNS)}. Cada modo corresponde a uma estratégia de resolução específica, mantendo-se inalterada a formulação matemática do problema.

No modo \textit{Exato}, o problema é resolvido integralmente por meio de um solver de programação inteira, visando a obtenção da solução ótima global, sujeito às limitações computacionais inerentes a problemas de grande escala. O modo \textit{Heurístico} emprega uma estratégia construtiva gulosa, capaz de produzir soluções viáveis de forma rápida, servindo tanto como solução inicial quanto como referência para análises comparativas. Já o modo \textit{LNS} combina a solução heurística inicial com ciclos iterativos de destruição e reconstrução parcial, utilizando modelos MILP locais para refinar progressivamente a solução.


Embora o modelo de programação inteira mista (MILP) formulado neste trabalho permita obter soluções ótimas para instâncias de porte moderado, a sua aplicação direta em cenários de grande escala (muitos motoristas, múltiplos dias e diferentes perfis de demanda) pode tornar-se computacionalmente onerosa. Além disso, em ambientes operacionais dinâmicos, como o transporte rodoviário de mercadorias, nem sempre é necessário obter a solução ótima global; muitas vezes, soluções de boa qualidade obtidas em tempos reduzidos são preferíveis.

Com esse contexto, propõe-se uma arquitetura de \textbf{Framework de Fallback Inteligente} centrada na maturidade operacional superior. Em vez de uma busca exaustiva por otimalidade teórica que possa ignorar restrições temporais críticas, o sistema utiliza uma abordagem hierárquica estruturada em: (i) uma heurística construtiva de alta precisão que gera uma escala inicial factível (\textit{fail-safe}), e (ii) um motor de resolução exata operando em duas fases lexicográficas:
\begin{enumerate}
    \item \textbf{Fase de Cobertura}: Minimiza-se a demanda não atendida ($U_d$), utilizando a solução heurística como \textit{warm start}.
    \item \textbf{Fase de Recursos}: Com a cobertura da fase anterior fixada (ou relaxada por um pequeno $\epsilon$), minimiza-se o número de motoristas ativos ($Z_t$).
\end{enumerate}

Esta hierarquia permite que o sistema mantenha a operabilidade plena sob o rigor do Regulamento (CE) 561/2006. Se o solver atingir o limite de tempo (\textit{timeout}) antes de concluir a prova de otimalidade da Fase 2, o framework recupera automaticamente a melhor solução disponível, garantindo 100\% de cobertura e validade legal, superando em eficiência os modelos simplificados que atingiam "otimalidade" apenas por falta de rigor nas restrições.

A heurística construtiva opera em uma linha do tempo discretizada (intervalos de 15 minutos) e, para cada período, atribui motoristas elegíveis de forma gulosa, respeitando as principais restrições regulatórias (condução contínua máxima, pausas mínimas, descanso diário e semanal, limites semanais e quinzenais). Esta fase produz rapidamente uma solução inicial factível, ainda que potencialmente distante do ótimo.

Na segunda camada, aplica-se uma matheurística LNS: a solução corrente é parcialmente destruída em sub-regiões (por exemplo, um dia específico, um subconjunto de motoristas ou janelas temporais críticas com sobrecarga de demanda), e um subproblema restrito é reotimizado com o modelo MILP proposto, sob limite de tempo. Assim, o modelo exato deixa de atuar sobre toda a instância e passa a ser utilizado como um \textit{oráculo} de melhoria local, combinando a robustez de soluções exatas com a flexibilidade exploratória típica de heurísticas.

Por fim, a terceira camada explora técnicas de aprendizagem de máquina para apoiar a heurística e a matheurística. A partir de instâncias menores, nas quais o MILP consegue atingir a solução ótima em tempo aceitável, constroem-se conjuntos de dados que relacionam padrões de alocação (por motorista, por período e por janela temporal) à qualidade da solução. Modelos supervisionados podem então ser treinados para: (a) atribuir escores de preferência a pares motorista-período, auxiliando a etapa construtiva, e (b) estimar a probabilidade de melhoria ao selecionar determinadas vizinhanças na LNS, priorizando regiões da solução com maior potencial de ganho.

Essa arquitetura híbrida (MILP + heurística + LNS + ML) permite, por um lado, preservar o rigor e a completude do modelo regulatório desenvolvido e, por outro, oferecer tempos de resposta compatíveis com o uso em sistemas de apoio à decisão em contexto real, seja em ambiente acadêmico (simulações) ou em produto (como no contexto da plataforma Ottimizia).

\subsubsection{Heurística construtiva para geração de escala inicial}

A seguir apresenta-se a heurística construtiva proposta para geração de uma escala inicial factível.

\begin{algorithm}[!ht]
\small
\caption{Heurística construtiva baseada em blocos e janelas deslizantes (Driver-Based)}
\label{alg:heuristica_construtiva}
\DontPrintSemicolon
\KwIn{
    Conjunto de motoristas $D$; \\
    Conjunto de períodos discretizados $T$ e Demanda $\mathrm{dem}(t)$; \\
    Tamanho mínimo do bloco de trabalho $\mathrm{B}_{\min}$ (ex: 4h30); \\
    Máximo de demandas por motorista $\mathrm{Cap}_{\max}$.
}
\KwOut{
    Escala inicial $y_{d,t} \in \{0,1\}$.
}

Inicializar $y_{d,t} \gets 0$ e cobertura atual $\mathrm{cov}(t) \gets 0$\;

\ForEach{motorista $d \in D$}{
    \If{cobertura total já é suficiente}{ \textbf{parar}; }
    
    definir melhor ganho $\mathrm{gain}^{\star} \gets -1$ e melhor início $t^{\star} \gets -1$\;
    
    \tcc{Busca por janela deslizante (Sliding Window)}
    \For{$t = 1$ \KwTo $|T| - \mathrm{B}_{\min}$}{
        calcular ganho marginal da janela $[t, t+\mathrm{B}_{\min}]$:
        $g \gets \sum_{k=t}^{t+\mathrm{B}_{\min}} \mathbb{I}(\mathrm{dem}(k) > \mathrm{cov}(k))$\;
        
        \If{$g > \mathrm{gain}^{\star}$}{
            $\mathrm{gain}^{\star} \gets g$\;
            $t^{\star} \gets t$\;
        }
    }
    
    \If{$t^{\star} \ge 0$ e $\mathrm{gain}^{\star} > 0$}{
        alocar motorista $d$ no intervalo base $[t^{\star}, t^{\star}+\mathrm{B}_{\min}]$\;
        atualizar $y_{d,k} \gets 1$ e $\mathrm{cov}(k) \gets \mathrm{cov}(k) + 1$ para $k$ no intervalo\;
        
        \tcc{Extensão gulosa (direita) até capacidade máxima}
        $t_{\text{ext}} \gets t^{\star} + \mathrm{B}_{\min}$\;
        \While{ainda tem capacidade e $t_{\text{ext}} < |T|$ e há demanda não atendida em $t_{\text{ext}}$}{
             $y_{d,t_{\text{ext}}} \gets 1$\;
             $\mathrm{cov}(t_{\text{ext}}) \gets \mathrm{cov}(t_{\text{ext}}) + 1$\;
             $t_{\text{ext}} \gets t_{\text{ext}} + 1$\;
        }
    }
}

\Return{$y_{d,t}$}\;
\end{algorithm}

\subsubsection{Matheurística LNS baseada em MILP}

A heurística construtiva gera uma solução inicial $y$. Para melhorar essa solução, propõe-se uma matheurística do tipo \textit{Large Neighborhood Search} (LNS), na qual partes da escala são destruídas e reotimizadas por meio do modelo MILP.

\begin{algorithm}[!ht]
\small
\caption{LNS matheurística para melhoria da escala}
\label{alg:lns_matheuristica}
\DontPrintSemicolon
\KwIn{
    Solução inicial $y$ (escala factível); \\
    Conjunto de motoristas $D$, períodos $T$ e demais parâmetros; \\
    Número máximo de iterações $\mathrm{it}_{\max}$; \\
    Tempo máximo de resolução do subproblema MILP $t_{\max}^{\text{MILP}}$.
}
\KwOut{
    Solução melhorada $y^{\text{best}}$.
}

Definir função objetivo $F(y)$ (por exemplo, número de motoristas utilizados, soma de horas extras, penalizações por slots não atendidos etc.)\;
Inicializar $y^{\text{curr}} \gets y$ e $y^{\text{best}} \gets y$\;

\For{$\mathrm{it} = 1$ \KwTo $\mathrm{it}_{\max}$}{
    selecionar um tipo de vizinhança $V$ (por exemplo: dia, subconjunto de motoristas, janela crítica)\;
    
    \eIf{$V =$ ``dia''}{
        escolher um dia $d_{\text{dia}}$ e definir subconjunto de períodos $T_V \subset T$ correspondentes a esse dia\;
        definir subconjunto de motoristas $D_V \gets D$ (todos os motoristas são candidatos para aquele dia)\;
    }{
        \If{$V =$ ``motoristas''}{
            escolher subconjunto de motoristas $D_V \subset D$ (por exemplo, os mais sobrecarregados)\;
            definir subconjunto de períodos $T_V \gets T$ (todos os períodos)\;
        }
        \If{$V =$ ``janela crítica''}{
            selecionar subconjunto de períodos $T_V \subset T$ com slots não atendidos ou muito próximos a violações\;
            definir subconjunto de motoristas $D_V \subset D$ que atuam em $T_V$\;
        }
    }
    
    construir subproblema MILP restrito a $D_V$ e $T_V$:
    \begin{itemize}
        \item variáveis de decisão $y_{d,t}$ apenas para $d \in D_V$, $t \in T_V$;
        \item manter fixas todas as demais decisões de $y^{\text{curr}}$ fora da vizinhança;
        \item preservar todas as restrições regulatórias (diárias, semanais, quinzenais);
        \item respeitar consistência com a solução fixa (por exemplo, limites semanais acumulados).
    \end{itemize}
    
    resolver o subproblema MILP com limite de tempo $t_{\max}^{\text{MILP}}$, obtendo uma solução local $y^{V}$ (se viável)\;
    
    \If{foi encontrada solução viável $y^{V}$}{
        construir uma solução candidata $y^{\text{cand}}$ substituindo, em $y^{\text{curr}}$, as decisões de $D_V \times T_V$ pelas provenientes de $y^{V}$\;
        \If{$F(y^{\text{cand}}) < F(y^{\text{curr}})$}{
            atualizar $y^{\text{curr}} \gets y^{\text{cand}}$\;
            \If{$F(y^{\text{curr}}) < F(y^{\text{best}})$}{
                atualizar $y^{\text{best}} \gets y^{\text{curr}}$\;
            }
        }
    }
}

\Return{$y^{\text{best}}$}\;
\end{algorithm}

\subsubsection{Integração com aprendizagem de máquina}

Para potencializar a eficiência da LNS, propõe-se o uso de modelos de aprendizagem de máquina para guiar a seleção de vizinhanças e priorizar regiões da solução com maior potencial de melhoria.

\begin{algorithm}[!ht]
\small
\caption{Esquema geral de integração com aprendizagem de máquina}
\label{alg:ml_guided_lns}
\DontPrintSemicolon
\KwIn{
    Conjunto de instâncias de treinamento; \\
    Soluções ótimas (ou de alta qualidade) obtidas via MILP em instâncias menores; \\
    Heurística construtiva e LNS descritas anteriormente.
}
\KwOut{
    Modelo de ML para apoio à decisão na LNS.
}

\textbf{Fase offline (treinamento)}:\;
\ForEach{instância de treinamento}{
    resolver o MILP até a otimalidade ou até um limite de tempo elevado, obtendo solução de referência $y^{\star}$\;
    gerar solução heurística $y^{H}$ (via heurística construtiva)\;
    extrair \textit{features} por período, motorista e janelas temporais (por exemplo: carga acumulada, folgas, distância até limites regulatórios, presença de slots não atendidos)\;
    rotular:
    \begin{itemize}
        \item pares $(d,t)$ como bons/ruins com base na diferença entre $y^{H}$ e $y^{\star}$;
        \item vizinhanças (conjuntos de períodos/motoristas) como promissoras se sua reotimização aproxima $y^{H}$ de $y^{\star}$.
    \end{itemize}
}
treinar modelos supervisionados (por exemplo, gradiente reforçado ou \textit{gradient boosting}) para:
\begin{itemize}
    \item $f_1(d,t) \to$ estimar a qualidade de atribuir o motorista $d$ ao período $t$;
    \item $f_2(V) \to$ estimar o ganho esperado de reotimizar uma vizinhança $V$.
\end{itemize}

\textbf{Fase online (uso na LNS)}:\;
Na heurística construtiva, utilizar $f_1$ como componente do escore $\mathrm{score}(d,t)$\;
Na LNS, ao selecionar vizinhanças, avaliar múltiplas opções $V_1,\dots,V_k$ e priorizar aquelas com maior $f_2(V_i)$\;

\Return{modelos $f_1$ e $f_2$ integrados ao processo heurístico}\;
\end{algorithm}

% ============================================================
% SEÇÃO DE METODOLOGIA — INTEGRAÇÃO COM OS INDICADORES DO SIMULADOR
% ============================================================


\section{Metodologia de Avaliação}
\label{sec:metodologia-avaliacao}

A avaliação das soluções geradas pelos métodos implementados no simulador
--- Programação Inteira (exato), heurística construtiva e método matheurístico LNS ---
baseia-se em um conjunto estruturado de indicadores quantitativos, gráficos analíticos
e métricas consolidadas na literatura de scheduling e otimização combinatória
\cite{pinedo2016scheduling, talbi2009metaheuristics, papadimitriou1998combinatorial, santamaria2025driver}.

Cada indicador descrito nesta seção contribui para a análise multidimensional de desempenho,
permitindo verificar simultaneamente: qualidade da solução, estabilidade temporal, segurança operacional,
eficiência de cobertura e custo computacional.

As subseções seguintes descrevem formalmente cada métrica, acompanhadas de suas figuras correspondentes.

% ============================================================
% \subsection{Demanda por Período}
% \label{subsec:demanda-periodo}

% A curva de demanda representa o número de motoristas requeridos em cada intervalo de tempo.
% Esse gráfico caracteriza a carga operacional e estabelece a referência mínima de cobertura esperada.

% \begin{figure}[H]
%     \centering
%     %\includegraphics[width=\textwidth]{fig/demanda_periodo.pdf}
%     \caption{Demanda operacional por período.}
%     \label{fig:demanda-periodo}
% \end{figure}

% Picos de demanda visíveis na \autoref{fig:demanda-periodo}
% implicam maior pressão sobre o algoritmo, enquanto períodos de baixa carga
% permitem redistribuição estratégica de motoristas.

% % ============================================================
% \subsection{Escalonamento Gerado (Driver Scheduling)}
% \label{subsec:driver-scheduling}

% O gráfico da solução detalha a quantidade de motoristas alocados por período pelo modelo de otimização.

% \begin{figure}[H]
%     \centering
%     %\includegraphics[width=\textwidth]{fig/driver_scheduling.pdf}
%     \caption{Motoristas alocados por período na solução gerada.}
%     \label{fig:driver-scheduling}
% \end{figure}

% Comparações entre demanda (Figura~\ref{fig:demanda-periodo})
% e solução (Figura~\ref{fig:driver-scheduling})
% apontam aderência, excesso ou déficit operacional.

% % ============================================================
% \subsection{Taxa de Cobertura}
% \label{subsec:coverage-rate}

% A taxa de cobertura \(Coverage_t\) é definida como:

% \[
% Coverage_t = \frac{\text{Motoristas}_t}{Demanda_t}.
% \]

% \begin{figure}[H]
%     \centering
%     %\includegraphics[width=\textwidth]{fig/coverage_rate.pdf}
%     \caption{Taxa de cobertura por período.}
%     \label{fig:coverage-rate}
% \end{figure}

% Valores próximos de 1 indicam aderência adequada, enquanto valores inferiores apontam risco operacional.
% A \autoref{fig:coverage-rate} sintetiza esse comportamento visualmente.

% % ============================================================
% \subsection{Índice de Sobrecarga}
% \label{subsec:overload-index}

% O índice de sobrecarga é dado por:

% \[
% Overload_t = 
% \frac{
%     \max(0, \text{Motoristas}_t - Demanda_t)
% }{
%     Demanda_t
% }.
% \]

% \begin{figure}[H]
%     \centering
%     %\includegraphics[width=\textwidth]{fig/overload_index.pdf}
%     \caption{Índice de sobrecarga por período.}
%     \label{fig:overload-index}
% \end{figure}

% A \autoref{fig:overload-index} evidencia janelas nas quais há superdimensionamento da escala.

% % ============================================================
% \subsection{Índice de Subutilização}
% \label{subsec:underload-index}

% Definido por:

% \[
% Underload_t = 
% \frac{
%     \max(0, Demanda_t - \text{Motoristas}_t)
% }{
%     \text{Motoristas}_t
% }.
% \]

% \begin{figure}[H]
%     \centering
%     %\includegraphics[width=\textwidth]{fig/underload_index.pdf}
%     \caption{Índice de subutilização por período.}
%     \label{fig:underload-index}
% \end{figure}

% A \autoref{fig:underload-index} mostra períodos de déficit,
% fundamentais para análise de risco operacional.

% % ============================================================
% \subsection{Mapa de Calor de Cobertura}
% \label{subsec:coverage-heatmap}

% \begin{figure}[H]
%     \centering
%     % \includegraphics[width=\textwidth]{fig/coverage_heatmap.pdf}
%     \caption{Mapa de calor de cobertura por período.}
%     \label{fig:coverage-heatmap}
% \end{figure}

% Esse mapa facilita a identificação de padrões estruturais, como clusters de subcobertura.

% % ============================================================
% \subsection{Mapa de Calor da Margem de Segurança}
% \label{subsec:safety-margin}

% \begin{figure}[H]
%     \centering
%     % \includegraphics[width=\textwidth]{fig/safety_margin_heatmap.pdf}
%     \caption{Mapa de calor da margem de segurança operacional.}
%     \label{fig:safety-margin}
% \end{figure}

% A \autoref{fig:safety-margin} destaca períodos críticos associados a déficit ou superávit operacional.

% % ============================================================
% \subsection{Convergência do LNS}
% \label{subsec:lns-convergence}

% \begin{figure}[H]
%     \centering
%     % \includegraphics[width=\textwidth]{fig/lns_convergence.pdf}
%     \caption{Evolução da função objetivo nas iterações do LNS.}
%     \label{fig:lns-convergence}
% \end{figure}

% Esse gráfico é fundamental para demonstrar o comportamento da matheurística,
% evidenciando melhora progressiva ou estagnação.

% % ============================================================
% \subsection{Relaxação do LNS}
% \label{subsec:lns-relaxation}

% \begin{figure}[H]
%     \centering
%     % \includegraphics[width=\textwidth]{fig/lns_relaxation.pdf}
%     \caption{Nível de relaxação aplicado ao longo das iterações do LNS.}
%     \label{fig:lns-relaxation}
% \end{figure}

% Valores elevados sugerem exploração ampla;
% valores menores refletem intensificação.

% % ============================================================
% \subsection{Comparação entre Modos: Exato, Heurístico e LNS}
% \label{subsec:radar-chart}

% \begin{figure}[H]
%     \centering
%     % \includegraphics[width=\textwidth]{fig/radar_comparison.pdf}
%     \caption{Comparação entre os modos de solução (Exato, Heurístico e LNS).}
%     \label{fig:radar-comparison}
% \end{figure}

% O radar da \autoref{fig:radar-comparison} sintetiza várias métricas,
% permitindo comparar os métodos em estabilidade, custo, aderência e eficiência.

% % ============================================================
% \subsection{Matriz de Restrições (Antes e Depois das Operações)}
% \label{subsec:constraint-matrices}

% \begin{figure}[H]
%     \centering
%     % \includegraphics[width=\textwidth]{fig/initial_constraint_matrix.pdf}
%     \caption{Matriz inicial de restrições.}
%     \label{fig:initial-constraints}
% \end{figure}

% \begin{figure}[H]
%     \centering
%     % \includegraphics[width=\textwidth]{fig/final_constraint_matrix.pdf}
%     \caption{Matriz final após transformações.}
%     \label{fig:final-constraints}
% \end{figure}

% Comparar as Figuras~\ref{fig:initial-constraints} e \ref{fig:final-constraints}
% revela o impacto direto das operações elementares sobre a densidade e estrutura do modelo.


% Os resultados detalhados foram movidos para o Capítulo 6 (capitulo7.tex).

\section{Considerações Finais}

Com o objetivo de sintetizar e consolidar as diferenças observadas entre os métodos avaliados, a Tabela~\ref{tab:comparacao-metodos} apresenta uma comparação qualitativa entre as abordagens Exata, Heurística e \textit{Large Neighborhood Search} (LNS) no cenário operacional de 24 horas.

A análise considera critérios relevantes sob as perspectivas operacional e computacional, incluindo cobertura da demanda, eficiência no uso da capacidade, distribuição do esforço entre motoristas, fragmentação da alocação, tempo computacional e escalabilidade temporal. Esses critérios permitem avaliar não apenas a qualidade das soluções em termos de atendimento, mas também sua viabilidade prática em contextos reais de planejamento.

A comparação evidencia os trade-offs inerentes a cada abordagem. Enquanto o método exato fornece soluções analiticamente ótimas, sua aplicabilidade é limitada pelo elevado custo computacional. Em contrapartida, a heurística construtiva prioriza rapidez e viabilidade operacional imediata, ao custo de menor eficiência. O método LNS posiciona-se como uma alternativa intermediária, combinando elevada qualidade operacional com esforço computacional controlado, reforçando sua adequação para aplicações em larga escala.


\begin{table}[H]
\centering
\caption{Comparação entre os métodos Exact, Heuristic e LNS no cenário de 24 horas.}
\label{tab:comparacao-metodos}
\begin{tabular}{lccc}
\toprule
\textbf{Critério} & \textbf{Exact} & \textbf{Heuristic} & \textbf{LNS} \\
\midrule
Cobertura Global da Demanda & Alta ($\approx 95\%$) & Total (100\%) & Total (100\%) \\
Déficit de Demanda & Pontual & Nenhum & Nenhum \\
Número de Motoristas & Mínimo & Elevado & Intermediário \\
Eficiência de Capacidade & Alta & Baixa & Média-Alta \\
Distribuição do Esforço & Muito equilibrada & Equilibrada & Equilibrada \\
Fragmentação da Alocação & Baixa & Alta & Moderada \\
Tempo Computacional & Elevado & Muito baixo & Moderado \\
Escalabilidade Temporal & Limitada & Alta & Alta \\
Robustez Operacional & Média & Alta & Alta \\
Adequação Operacional & Analítica & Operacional & Operacional-Otimizada \\
\bottomrule
\end{tabular}
\end{table}

\section{Considerações Finais}

Este capítulo apresentou de forma integrada o modelo matemático de Programação Linear Inteira desenvolvido para o escalonamento de motoristas e a metodologia experimental adotada para sua validação. A formulação lexicográfica proposta reflete fielmente as prioridades operacionais do transporte rodoviário europeu, garantindo conformidade regulatória e eficiência na alocação de recursos.

A implementação computacional, realizada em Python com o solver CP-SAT, demonstrou viabilidade prática mesmo em cenários de grande escala. A comparação sistemática entre três métodos de resolução --- exato, heurístico e LNS --- evidenciou trade-offs claros entre qualidade analítica, eficiência operacional e custo computacional.

Os resultados experimentais confirmam que a abordagem matheurística híbrida proposta combina os pontos fortes de cada método, oferecendo flexibilidade para diferentes contextos operacionais: desde aplicações que exigem certificados de otimalidade até ambientes que priorizam respostas rápidas e robustas.

O modelo matemático e a metodologia experimental aqui descritos fornecem uma base sólida para os capítulos subsequentes, nos quais serão discutidas extensões do modelo, análises de sensibilidade e implicações práticas para o setor de transporte rodoviário.
