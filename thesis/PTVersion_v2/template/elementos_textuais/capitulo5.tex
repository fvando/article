\chapter{Modelo Matemático}\label{capitulo5}

\section{Introdução}

Este capítulo apresenta o modelo de Programação Linear Inteira (PLI)
desenvolvido para o problema de escalonamento de motoristas no transporte
rodoviário sob a regulamentação europeia, em particular o Regulamento
(CE) n.º 561/2006.
O objetivo do modelo é gerar escalas legalmente válidas, eficientes e
capazes de atender à demanda operacional, respeitando simultaneamente
limites de condução, pausas obrigatórias e períodos mínimos de descanso.

Formulações baseadas em PLI têm sido amplamente utilizadas para problemas
de escalonamento com fortes dependências temporais, sobretudo em contextos
regulados como transporte, saúde e serviços críticos
\cite{savelsbergh1997,pillac2013}.
Neste trabalho, o modelo proposto estende a formulação apresentada em
\cite{moreira2025}, incorporando maior granularidade temporal, variáveis
auxiliares e uma estrutura de otimização compatível com o solver CP-SAT
do OR-Tools \cite{googleORTools}.

Diferentemente de abordagens puramente multiobjetivo, o modelo adota uma
estrutura lexicográfica em duas fases, refletindo prioridades operacionais
reais: primeiro garantir cobertura da demanda e minimizar o número de
motoristas ativos, e somente depois refinar a solução por critérios de
balanceamento e suavização de carga.

\section{Discretização Temporal}

O horizonte de planejamento é discretizado em períodos fixos de duração
$\Delta = 15$ minutos, formando o conjunto:
\[
T = \{1,2,\ldots,|T|\}.
\]

Essa discretização reflete a frequência real de chegada dos pedidos de
carga na operação analisada e permite representar de forma linear
restrições originalmente não lineares, como janelas móveis de descanso
e limites acumulados de condução \cite{pillac2013,erdman2022}.

\section{Conjuntos e Índices}

\begin{itemize}
    \item $D$: conjunto de períodos discretizados ($d \in D$);
    \item $T$: conjunto de motoristas ($t \in T$);
    \item $W$: janelas agregadas de planejamento (dias, semanas e quinzenas).
\end{itemize}

\section{Parâmetros}

\begin{align*}
demanda_d      &: \text{demanda operacional no período } d,\\
\Delta         &: \text{duração do período (15 minutos)},\\
L^{dia}        &: \text{limite diário de condução},\\
L^{sem}        &: \text{limite semanal de condução},\\
L^{14d}        &: \text{limite quinzenal de condução},\\
R^{dia}        &: \text{descanso diário mínimo},\\
R^{sem}        &: \text{descanso semanal mínimo}.
\end{align*}

Esses parâmetros refletem diretamente as exigências do Regulamento
(CE) n.º 561/2006 e da Diretiva 2002/15/CE \cite{eu5612006,eu200215}.

\section{Variáveis de Decisão}

\subsection*{Ativação do motorista}
\[
Z_t =
\begin{cases}
1, & \text{se o motorista } t \text{ é ativado},\\
0, & \text{caso contrário}.
\end{cases}
\]

\subsection*{Presença no período}
\[
Y_{d,t} =
\begin{cases}
1, & \text{se o motorista } t \text{ está presente no período } d,\\
0, & \text{caso contrário}.
\end{cases}
\]

\subsection*{Atendimento de demanda}
\[
X_{d,t} \in \mathbb{Z}_{+}, \quad 0 \le X_{d,t} \le \text{capacidade por slot}.
\]

\subsection*{Demanda não atendida}
\[
U_d \ge 0.
\]

\subsection*{Carga por motorista e carga máxima}
\[
L_t = \sum_{d \in D} X_{d,t}, \qquad L_{\max} \ge L_t.
\]

A separação entre variáveis de ativação, presença e carga segue práticas
consolidadas na literatura de escalonamento, permitindo representar de
forma explícita decisões operacionais e restrições legais
\cite{savelsbergh1997,pinheiro2007staffscheduling}.

\section{Função Objetivo Lexicográfica}

O modelo utiliza uma estrutura de otimização em duas fases, resolvidas
sequencialmente pelo solver.

\subsection{Fase 1 — Cobertura e Minimização de Motoristas}

A Fase 1 possui prioridade absoluta e busca:

\begin{enumerate}
    \item minimizar a demanda não atendida;
    \item minimizar o número de motoristas ativados;
    \item incentivar maior utilização dos motoristas ativos.
\end{enumerate}

A função objetivo é dada por:
\[
\min \;
\alpha \sum_{d \in D} U_d
+ \beta \sum_{t \in T} Z_t
- \gamma \sum_{d \in D} \sum_{t \in T} X_{d,t},
\]
com $\alpha \gg \beta \gg \gamma$, garantindo prioridade lexicográfica.

Estruturas desse tipo são amplamente utilizadas quando múltiplos objetivos
conflitantes precisam ser tratados de forma hierárquica
\cite{pillac2013,erdman2022}.

\subsection{Fase 2 — Balanceamento e Refinamento}

Após fixar os valores ótimos de $\sum U_d$ e $\sum Z_t$, a Fase 2 refina
a solução, buscando:

\begin{itemize}
    \item reduzir presença desnecessária;
    \item penalizar excesso de carga acima de um limite suave;
    \item suavizar picos de carga entre motoristas.
\end{itemize}

A função objetivo da Fase 2 é:
\[
\min \;
\omega_1 \sum_{d,t} Y_{d,t}
+ \omega_2 \sum_{t} \max(0, L_t - L^{soft})
+ \omega_3 L_{\max}.
\]

Esses critérios promovem maior equidade e robustez operacional, sendo
frequentemente empregados como etapa de refinamento em problemas de
escalonamento \cite{savelsbergh1997}.

\section{Restrições do Modelo}

O modelo matemático incorpora um conjunto abrangente de restrições
legais e operacionais derivadas do Regulamento (CE) n.º 561/2006.
Entretanto, nem todas as restrições são necessariamente ativadas
simultaneamente em todos os experimentos.

No simulador desenvolvido, cada restrição pode ser ativada ou desativada
de forma independente por meio de parâmetros de configuração,
permitindo a análise controlada do impacto individual e combinado das
regras legais sobre a solução.
Essa abordagem possibilita estudos comparativos entre diferentes níveis
de rigor regulatório, sem alterar a estrutura fundamental do modelo.

Formalmente, o conjunto de restrições ativas em um experimento é definido
por um vetor binário de ativação, sendo que apenas as restrições
selecionadas são incluídas no modelo resolvido pelo solver.

\subsection{Atendimento da Demanda}

Quando a restrição de cobertura obrigatória está ativada, o atendimento
da demanda é modelado como uma restrição rígida:

\[
\sum_{t \in T} X_{d,t} + U_d = demanda_d, \quad \forall d \in D.
\]

Caso essa restrição não seja ativada, a variável $U_d$ passa a representar
déficits operacionais admissíveis, utilizados exclusivamente para fins
de avaliação e cálculo de indicadores, sem inviabilizar a solução.


\subsection{Vinculação entre presença e ativação}
\[
Y_{d,t} \le Z_t, \quad \forall d,t.
\]

\subsection{Capacidade por período}
\[
X_{d,t} \le \text{capacidade} \cdot Y_{d,t}.
\]

\subsection{Limites legais de condução}

As restrições apresentadas a seguir são incluídas no modelo apenas quando
explicitamente ativadas no cenário experimental considerado.

Restrições diárias, semanais e quinzenais são modeladas por janelas móveis,
conforme exigido pelo Regulamento (CE) n.º 561/2006.

\subsection{Pausas e descansos}
Pausas após 4,5 horas de condução contínua e descansos diários e semanais
são representados por restrições lineares baseadas em janelas deslizantes.

\section{Discussão sobre Linearização}

As regras legais apresentam dependências temporais não lineares, tratadas
por meio de discretização, variáveis auxiliares e constantes Big-M.
Essa abordagem é amplamente utilizada para manter a linearidade da
formulação e compatibilidade com solvers inteiros modernos
\cite{pillac2013,googleORTools}.

\section{Observações Computacionais}

O modelo apresenta dezenas de milhares de variáveis e restrições, mas
possui matriz esparsa e forte estrutura temporal. Solvers modernos, como
o CP-SAT, exploram essas características de forma eficiente, permitindo
resolver cenários reais em tempos compatíveis com uso operacional
\cite{erdman2022,googleORTools}.

\section{Considerações Finais}

O modelo matemático apresentado fornece uma base rigorosa e aderente à
legislação europeia para o escalonamento de motoristas. Sua formulação
lexicográfica reflete prioridades operacionais reais e viabiliza a
integração com heurísticas e métodos matheurísticos discutidos nos
capítulos seguintes, sem comprometer a consistência matemática.
