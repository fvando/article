\begin{resumo}
O escalonamento de motoristas no transporte rodoviário de cargas sob a regulamentação europeia constitui um problema complexo de otimização combinatória, especialmente quando se considera a integração simultânea de restrições legais, operacionais e de demanda temporal.

Este trabalho apresenta o desenvolvimento de um simulador para o escalonamento de motoristas no transporte rodoviário europeu, fundamentado em um modelo de Programação Linear Inteira (PLI) que incorpora as exigências legais estabelecidas pelo Regulamento (CE) n.{\textordmasculine} 561/2006, pela Diretiva 2002/15/CE e pelo Regulamento (UE) n.{\textordmasculine} 165/2014. 

O modelo matemático foi formulado de modo a representar, todas as restrições diárias, semanais e quinzenais de condução, pausas e períodos de descanso obrigatórios, incluindo dependências temporais complexas e janelas móveis de conformidade com a regulamentação.

A implementação computacional foi desenvolvida e integrada a uma interface interativa, permitindo a configuração de cenários, seleção dinâmica de cenários e visualização dos resultados. Além do modo exato de resolução, o simulador incorpora abordagens heurísticas e uma estratégia matheurística baseada em \textit{Large Neighborhood Search} (LNS), possibilitando a análise do comportamento do modelo em diferentes regimes de complexidade e escalabilidade.

Adicionalmente, o trabalho propõe a integração experimental de modelos de aprendizado de máquina supervisionado, utilizados para guiar decisões locais de alocação ($f_1$) e a seleção de vizinhanças no LNS ($f_2$), preservando mecanismos de \textit{fallback} heurístico e estabilidade operacional. O simulador disponibiliza um conjunto de indicadores de desempenho (KPIs), que auxiliam na interpretação dos resultados.

Os experimentos computacionais conduzidos em horizontes de curto e médio prazo indicam que a abordagem constitui uma alternativa em identificar soluções e analisar cenários sob diferentes combinações de restrições, fornecendo suporte à tomada de decisão estratégica, tática e operacional.

Adicionalmente, o trabalho propõe um ambiente unificado de modelagem, simulação e análise para o escalonamento de motoristas sob a legislação europeia, e pode contribuir para o estado da arte ao combinar otimização matemática, heurísticas, matheurísticas e aprendizado de máquina em um único ambiente experimental.
\par
\textbf{Palavras-chave}: Programação Linear Inteira; Escalonamento de Motoristas; Regulamento (CE) n.{\textordmasculine} 561/2006; Otimização; OR-Tools; Matheurísticas; Simulação; Aprendizado de Máquina.
\end{resumo}

