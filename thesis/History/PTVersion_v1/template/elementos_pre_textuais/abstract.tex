\begin{resumo}[Abstract]
    \begin{otherlanguage*}{english}
        This work presents the development of an Integer Linear Programming (ILP) model for driver scheduling in European road transport operations, fully aligned with the requirements established by Regulation (EC) No. 561/2006, Directive 2002/15/EC, and Regulation (EU) No. 165/2014. The model accurately represents all daily, weekly, and biweekly constraints regarding driving limits, mandatory breaks, and rest periods, incorporating complex temporal dependencies and rolling compliance windows. The computational implementation was developed in Python using the OR-Tools CP-SAT solver, together with an interactive Streamlit interface for scenario configuration, model execution, and operational indicator visualization. Experiments conducted over 7-, 15-, and 30-day planning horizons demonstrated high computational efficiency, consistent optimal solutions, and strong alignment between operational demand and allocated drivers. The results show that the proposed model is robust, scalable, and applicable to real-world road transport operations, providing support for strategic, tactical, and operational decision-making. Furthermore, the research advances the state of the art by offering a complete mathematical formulation for driver scheduling under European legislation, contributing to the academic literature and opening opportunities for future studies on integrated optimization, stochastic modeling, and SaaS-based decision-support systems for the transport sector.\par
        \textbf{Keywords}: Integer Linear Programming; Driver Scheduling; European Regulation 561/2006; Optimization; OR-Tools; Logistics.
    \end{otherlanguage*}
\end{resumo}