\begin{apendicesenv}
\partapendices

\chapter{Formulações Alternativas do Modelo}

\noindent
\textbf{Nota de consistência terminológica.}
Nos apêndices, o termo \emph{tarefa} é utilizado de forma genérica para representar
blocos de trabalho associados a períodos discretizados no modelo principal.
Quando aplicável, a correspondência tarefa $\leftrightarrow$ período é direta,
não implicando perda de generalidade da formulação.

Este apêndice apresenta formulações alternativas e extensões estruturadas a partir
do modelo principal de Programação Linear Inteira (PLI) desenvolvido para o
problema de escalonamento de motoristas profissionais sob o Regulamento (CE)
n.º~561/2006. O objetivo destes modelos complementares é demonstrar a
flexibilidade da abordagem proposta e documentar versões alternativas avaliadas
durante o processo de experimentação.

\section{Modelo com Janelas de Tempo Suaves (Soft Time Windows)}
Nesta formulação, janelas de tempo rígidas para início de turnos foram relaxadas
por meio de penalizações lineares. O objetivo consiste em permitir que o solver
explore soluções mais amplas em cenários de alta restrição, atribuindo custos a
atrasos e adiantamentos. A função objetivo passa a incorporar penalidades
adicionais:

\[
\min\; Z = \sum_{i,j} c_{ij} x_{ij} + \sum_{m} \alpha_m e_m + \sum_{m} \beta_m a_m,
\]

onde $e_m$ representa atraso permitido e $a_m$ adiantamento. Este modelo foi útil
em cenários de sobrecarga operacional.

\section{Modelo com Agrupamento de Motoristas por Categoria}
Em operações reais, empresas segmentam motoristas por categorias: nacional,
internacional, ADR, refrigerado, entre outras. Nesta formulação estendida,
restrições de compatibilidade foram adicionadas:

\[
x_{ij} \leq cat(i,j)
\]

onde $cat(i,j)$ é uma matriz binária que define compatibilidade entre categorias e
tarefas. Essa abordagem permitiu analisar impactos da especialização.

\section{Modelo com Múltiplos Períodos (Planeamento Semanal)}
Para planeamento semanal, foi desenvolvido um modelo multi-período com acumulação
de horas por motorista, respeitando limites de direção diários e semanais:

\[
H^\text{sem}_m = \sum_{d \in Dias} H_{m,d} \leq H^\text{max}_\text{sem}.
\]

Esta versão foi particularmente útil nos testes de escalabilidade, permitindo
avaliar desempenho do solver frente a horizontes temporais ampliados.

\section{Modelo com Preferências do Motorista}
Em extensões opcionais, foi testado um modelo com vetores de preferência
(motorista–rota–horário), introduzindo coeficientes de satisfação:

\[
\max \sum_{i,j} s_{ij} x_{ij},
\]

onde $s_{ij}$ representa o nível de preferência. Essa formulação foi útil no
estudo exploratório de equilíbrio entre eficiência e bem-estar do trabalhador.

\bigskip
As variações aqui documentadas constituem parte do processo iterativo de
refinamento do modelo e servem como base para potenciais trabalhos futuros.



\chapter{Cenários de Simulação e Parâmetros Utilizados}

Este apêndice documenta integralmente os cenários experimentais utilizados para
avaliar o modelo de escalonamento, bem como os parâmetros adotados no solver
OR-Tools e restrições específicas aplicadas durante os testes.

\section{Descrição Geral dos Cenários}
Foram configurados três cenários principais:

\begin{itemize}
    \item \textbf{Cenário 1 — Operação Diária Regular:}  
    25 motoristas, 40 tarefas, predominância de viagens locais, baixa variabilidade
    temporal.

    \item \textbf{Cenário 2 — Alta Demanda com Conflitos de Horário:}  
    40 motoristas, 70 tarefas, múltiplos períodos sobrepostos, ocorrências de
    turnos com início simultâneo e forte impacto das janelas temporais.

    \item \textbf{Cenário 3 — Operação Internacional:}  
    20 motoristas com categorias distintas, viagens transfronteiriças,
    restrições completas do Regulamento 561/2006, incluindo pausas e repousos
    semanais reduzidos.
\end{itemize}

\section{Configurações do Solver (CP-SAT / OR-Tools)}

Embora o solver CP-SAT não seja um MILP clássico, utiliza mecanismos análogos
a \emph{presolve}, cortes e heurísticas internas, configurados implicitamente
ou via parâmetros de alto nível.

As principais configurações aplicadas ao OR-Tools foram:

\begin{itemize}
    \item \textbf{MIP\_gap:} 0,01 (1\%).  
    \item \textbf{MaxTime:} 600 segundos (10 minutos).  
    \item \textbf{Presolve:} ativado.  
    \item \textbf{Cuts:} automáticas.  
    \item \textbf{Heurísticas:} ativadas (feasibility pump e LP relaxation).  
\end{itemize}

O tempo de execução variou entre 4s e 35s para os cenários básicos, e até 2m15s
para os cenários de escala ampliada.

\section{Restrições Consideradas}
Foram aplicadas integralmente as seguintes exigências:

\begin{itemize}
    \item limite de 9h de condução diária (com variação permitida para 10h)
    \item pausas mínimas de 45 minutos a cada 4h30 de direção
    \item repouso diário regular de 11h ou fracionado em 3h + 9h
    \item controle semanal de horas acumuladas
    \item compatibilidade entre motorista e tarefa
\end{itemize}

\section{Dados Sintéticos Gerados}
Para garantir reprodutibilidade, parte dos dados foi sintetizada usando
procedimentos controlados de geração, incluindo:

\begin{itemize}
    \item distribuição normal truncada para durações de tarefas
    \item distribuição uniforme para janelas de início
    \item matriz de distâncias obtida a partir de centroides simulados
\end{itemize}

Os dados reais utilizados na empresa cliente foram anonimizados e não são
expostos neste documento por razões de confidencialidade.

\bigskip
Os cenários aqui documentados constituem a base experimental para validação,
desempenho e análise de sensibilidade do modelo de escalonamento proposto.


\chapter{Estrutura de Entradas e Saídas do Solver}

Este apêndice descreve o formato dos dados utilizados no solver OR-Tools e as
estruturas retornadas após a resolução do modelo.

\section{Entradas do Modelo}

\begin{itemize}
    \item \textbf{Matriz de tarefas ($T$):}  
    Contém identificação, duração, janelas de início e compatibilidades.

    \item \textbf{Conjunto de motoristas ($M$):}  
    Inclui categoria, histórico de horas, limites semanais e tipo de operação.

    \item \textbf{Parâmetros regulatórios ($R$):}  
    Conjunto estruturado contendo:
    \begin{itemize}
        \item limites de condução diária e semanal,
        \item regras de pausa,
        \item repousos diários e semanais.
    \end{itemize}

    \item \textbf{Parâmetros do Solver ($S$):}  
    \begin{itemize}
        \item MIP\_gap,
        \item tempo máximo,
        \item níveis de corte,
        \item ativação de heurísticas.
    \end{itemize}
\end{itemize}

\section{Saídas do Modelo}

\begin{itemize}
    \item \textbf{Alocação motorista–tarefa ($x_{ij}=1$):}  
    Indica qual motorista foi designado a cada tarefa.

    \item \textbf{Horas acumuladas:}  
    Relatório do total de horas diárias e semanais por motorista.

    \item \textbf{Verificação de conformidade:}  
    Conjunto de flags apontando violações regulatórias.

    \item \textbf{Função objetivo atingida ($Z^*$):}  
    Custo total, número de motoristas utilizados ou métrica equivalente.

    \item \textbf{Relatório temporal:}  
    Tempo de execução do solver, número de nós explorados, cortes aplicados.
\end{itemize}

\chapter{Processo de Geração de Dados Sintéticos}

Este apêndice descreve o método empregado para gerar dados sintéticos utilizados
nos experimentos, garantindo reprodutibilidade e aderência ao comportamento real
do setor de transporte rodoviário.

\section{Estratégia de Geração}
O processo foi dividido em três etapas:

\subsection*{1. Geração das Tarefas}
Durations foram amostradas de uma distribuição normal truncada:
\[
D \sim \text{Normal}(\mu = 180, \sigma = 40) \quad \text{com} \quad D \in [60, 480].
\]

\subsection*{2. Janelas de Tempo}
Os intervalos de início foram gerados usando distribuição uniforme:
\[
S \sim \text{Uniform}(6h, 20h).
\]

\subsection*{3. Perfil dos Motoristas}
Atributos gerados:

\begin{itemize}
    \item categoria (Nacional, Internacional, ADR),
    \item horas acumuladas na semana anterior,
    \item disponibilidade,
    \item qualificações específicas.
\end{itemize}

\section{Validação dos Dados}
Foram aplicadas checagens consistentes:

\begin{itemize}
    \item nenhuma tarefa ultrapassa o limite diário permitido,
    \item janelas de tempo não se sobrepõem indevidamente,
    \item distribuição de categorias aproximada da realidade da empresa.
\end{itemize}

\chapter{Modelo Matemático Completo}

Este apêndice apresenta a formulação completa do modelo de Programação Linear
Inteira utilizado na dissertação.

\section{Variáveis de Decisão}

\[
x_{ij} = 
\begin{cases}
1 & \text{se o motorista } i \text{ realiza a tarefa } j,\\
0 & \text{caso contrário}.
\end{cases}
\]

\[
h_{i}^\text{dia},\; h_{i}^\text{sem}: \text{horas acumuladas diário e semanal}.
\]

\section{Função Objetivo}

\[
\min Z = \sum_{i,j} c_{ij}x_{ij},
\]

onde $c_{ij}$ representa custos operacionais, distância, categoria ou penalidades.

\section{Restrições}

\subsection{Atribuição Única}
\[
\sum_{i} x_{ij} = 1 \quad \forall j.
\]

\subsection{Limite de Condução Diária}
\[
h_i^\text{dia} = \sum_{j} d_j x_{ij} \le 9 + e_i.
\]

\subsection{Janelas de Tempo}
\[
s_j x_{ij} \ge a_j, \qquad s_j x_{ij} + d_j \le b_j.
\]

\subsection{Compatibilidade Motorista–Tarefa}
\[
x_{ij} \leq comp(i,j).
\]

\subsection{Acúmulo Semanal}
\[
h_i^\text{sem} = \sum_{j} d_j x_{ij} \le 56.
\]

\section{Modelo Final}
Versão consolidada inclui todas as restrições regulatórias (CE 561/2006), limites
de pausa, repouso e condução quinzenal, incorporadas por meio de estruturas
lineares ou restrições complementares.


\end{apendicesenv}

