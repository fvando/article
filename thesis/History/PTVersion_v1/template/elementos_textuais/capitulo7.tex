\chapter{Resultados}\label{capitulo7}

\section{Introdução}

Este capítulo apresenta os resultados obtidos com a execução do modelo de Programa{\c c}{\~a}o Linear Inteira (PLI) e da implementação computacional descrita no Capítulo~5. Os experimentos foram realizados para diferentes horizontes temporais (1, 7, 15 e 30 dias), utilizando-se demanda variável e ativação integral das restrições impostas pelo Regulamento (CE) n.{\textordmasculine}~561/2006. 

Os resultados demonstram que o modelo é capaz de gerar escalas em conformidade com as restrições normativas modeladas, eficientes e compatíveis com cenários operacionais representativos de transporte rodoviário, apresentando tempos de execução reduzidos nos cenários avaliados e alta estabilidade temporal.

\section{Roteiro Experimental}

Os resultados apresentados neste capítulo seguem um roteiro experimental estruturado, concebido para avaliar de forma progressiva o comportamento do simulador sob diferentes estratégias de resolução e níveis de complexidade. Esse roteiro foi organizado em fases, permitindo analisar separadamente a viabilidade do modelo matemático, o desempenho computacional, a qualidade das soluções e o impacto de estratégias heurísticas, matheurísticas e orientadas por aprendizado de máquina.

As fases experimentais contemplam: (i) otimização exata via Programação Linear Inteira, (ii) geração de soluções iniciais por heurística gulosa, (iii) refinamento das soluções por Large Neighborhood Search (LNS), (iv) integração de modelos de aprendizado de máquina supervisionado e (v) avaliação comparativa entre os diferentes modos de resolução. Essa abordagem sistemática favorecendo rastreabilidade, reprodutibilidade computacional e rigor metodológico na análise dos resultados.

\section{Configuração Geral dos Cenários}

Os cen{\'a}rios foram configurados conforme os seguintes parâmetros principais:

\begin{itemize}
    \item solver: CP-SAT (OR-Tools);
    \item granularidade: 15 minutos por período;
    \item horizontes avaliados: 1, 7, 15 e 30 dias;
    \item variável principal: \(x_{d,t}\) (binária);
    \item ativa{\c c}{\~a}o completa das restrições legais;
    \item demanda variável por período (não uniforme).
\end{itemize}

Essas configurações permitem avaliar não apenas a viabilidade legal das escalas, mas também a estabilidade, comportamento sob variações de cenário e eficiência computacional do modelo.

\begin{table}[H]
\centering
\caption{Parâmetros gerais utilizados nos experimentos}
\label{tab:parametros-experimentos}
\begin{tabular}{ll}
\hline
\textbf{Parâmetro} & \textbf{Valor} \\
\hline
Solver & CP-SAT (OR-Tools) \\
Granularidade temporal & 15 minutos \\
Horizontes avaliados & 1, 7, 15 e 30 dias \\
Tipo de variável & Binária ($x_{d,t}$) \\
Restrições legais & Regulamento (CE) n.º 561/2006 (todas ativas) \\
Demanda & Variável por período \\
Semente aleatória & Fixa para reprodutibilidade \\
\hline
\end{tabular}
\end{table}

\section{Execução do Solver}

A Figura~\ref{fig:solver-otimo} apresenta um exemplo de execução completa do solver CP-SAT.

\begin{figure}[H]
    \centering
    \includegraphics[width=0.92\textwidth]{template/figuras/T11.png}
    \caption{Execução do solver CP-SAT (solução ótima).}
    \label{fig:solver-otimo}
\end{figure}

O solver apresenta:

\begin{itemize}
    \item solução ótima;
    \item tempo computacional médio inferior a 1,5 segundos nos ambientes e instâncias avaliados;
    \item cerca de 9.600 variáveis;
    \item cerca de 7.597 restrições;
    \item estabilidade mesmo em cenários maiores.
\end{itemize}

Esse desempenho confirma que a formulação PLI, associada ao CP-SAT, é adequada como ferramenta de apoio à análise e simulação no setor, permitindo replanejamento rápido e análises exploratórias.

\section{Comportamento da Demanda}

A demanda utilizada nos experimentos apresenta elevada variabilidade, característica comum em opera{\c c}{\~o}es com turnos contínuos e atividades multicentrais.

A Figura~\ref{fig:demanda-experimentos} ilustra a demanda utilizada.

\begin{figure}[H]
    \centering
    \includegraphics[width=0.92\textwidth]{template/figuras/T12.png}
    \caption{Curva de demanda operacional por período.}
    \label{fig:demanda-experimentos}
\end{figure}

Observa-se:

\begin{itemize}
    \item períodos com demanda superior a 100 motoristas;
    \item períodos de baixa flutuação;
    \item alternância entre picos e vales;
    \item irregularidade — essencial para testar estabilidade observada do modelo.
\end{itemize}

Cabe destacar que a demanda utilizada não foi previamente suavizada ou filtrada.
A manutenção de picos, vales e irregularidades é intencional, pois permite avaliar
a estabilidade observada do modelo frente a cenários realistas, nos quais variações abruptas de
demanda são comuns em operações de transporte rodoviário. Dessa forma, o desempenho observado indica potencial aplicabilidade prática do modelo proposto.

\section{Comparação Demanda \textit{versus} Alocação}

A Figura~\ref{fig:demanda-vs-cobertura-2} apresenta a comparação entre a demanda real e o número de motoristas alocados pelo modelo.

\begin{figure}[H]
    \centering
    \includegraphics[width=0.92\textwidth]{template/figuras/T3.jpg}
    \caption{Comparação entre demanda e cobertura gerada pelo modelo.}
    \label{fig:demanda-vs-cobertura-2}
\end{figure}

A análise mostra:

\begin{itemize}
    \item a cobertura acompanha de forma consistente a curva de demanda;
    \item não há sobrealocação significativa (solver evita excesso de motoristas);
    \item em períodos de baixa demanda, a alocação reduz-se naturalmente;
    \item o comportamento temporal é suave e sem oscilações abruptas.
\end{itemize}

A aderência entre cobertura e demanda indica que o modelo minimiza satisfatoriamente a diferença entre ambas, atendendo não apenas à conformidade legal, mas também à eficiência operacional.

Esse comportamento confirma que o modelo responde adequadamente
a ambos os critérios de otimização implementados no simulador.
No modo de maximização da resposta à demanda, observa-se a priorização
da cobertura mesmo em cenários críticos, enquanto no modo de minimização
do número de motoristas, o solver evita sobrealocação desnecessária,
mantendo aderência estrita à demanda.

\section{Indicadores Operacionais}

Foram calculados diversos indicadores estatísticos para auxiliar na interpretação dos resultados. Entre eles:

\begin{itemize}
    \item \textbf{Cobertura da demanda};
    \item \textbf{Sobrecarga};
    \item \textbf{Subutilização};
    \item \textbf{Desvio padrão da cobertura}.
\end{itemize}

A Figura~\ref{fig:indicadores-operacionais} resume esses indicadores.

\begin{figure}[H]
    \centering
    \includegraphics[width=0.92\textwidth]{template/figuras/T2.jpg}
    \caption{Indicadores de cobertura, sobrecarga e subutilização.}
    \label{fig:indicadores-operacionais}
\end{figure}

Resultados observados:

\begin{itemize}
    \item cobertura média superior a 99\%;
    \item mínima ocorrência de sobrecarga;
    \item subutilização restrita a períodos de baixa demanda;
    \item desvio padrão da cobertura \(\approx 0,04\), indicando homogeneidade temporal.
\end{itemize}

O baixo desvio padrão da cobertura indica elevada regularidade
na alocação de motoristas ao longo do tempo, o que é particularmente
relevante em contextos operacionais, pois reduz oscilações abruptas
na escala e facilita o planejamento logístico e a gestão de recursos humanos.

\section{Estrutura da Matriz de Restrições}

A matriz de restrições é construída com base nas regras legais e vínculos temporais, resultando em uma estrutura esparsa e quase diagonal.

As Figuras~\ref{fig:matriz-inicial-2} e \ref{fig:matriz-final-2} apresentam sua versão inicial e final, respectivamente.

\begin{figure}[H]
    \centering
    \includegraphics[width=0.92\textwidth]{template/figuras/T122.png}
    \caption{Matriz de restrições (versão inicial).}
    \label{fig:matriz-inicial-2}
\end{figure}

\begin{figure}[H]
    \centering
    \includegraphics[width=0.92\textwidth]{template/figuras/T13.png}
    \caption{Matriz de restrições (versão final).}
    \label{fig:matriz-final-2}
\end{figure}

A estrutura matricial influencia diretamente:

\begin{itemize}
    \item o desempenho do solver;
    \item a propagação de restrições;
    \item a velocidade de convergência.
\end{itemize}

A esparsidade entre 15\% e 35\% favorece o desempenho do solver observado nos experimentos do CP-SAT, conforme discutido em \cite{erdman2022}.

\section{Comparação entre Modos de Resolução}

Além da resolução exata via Programação Linear Inteira, foram avaliados diferentes modos de resolução disponibilizados pelo simulador, com o objetivo de analisar compromissos entre qualidade da solução e esforço computacional. Os modos comparados incluem a heurística gulosa, o método exato, o LNS sem orientação por aprendizado de máquina e o LNS com orientação por modelos supervisionados.

A Tabela~\ref{tab:comparacao-modos} apresenta uma síntese qualitativa dos principais resultados observados.

\begin{table}[H]
\centering
\caption{Comparação entre modos de resolução}
\label{tab:comparacao-modos}
\begin{tabular}{lcccc}
\hline
\textbf{Modo} & \textbf{Cobertura} & \textbf{Eficiência} & \textbf{Estabilidade} & \textbf{Tempo} \\
\hline
Heurístico & Alta & Média & Média & Muito baixo \\
Exato (PLI) & Ótima & Alta & Alta & Médio \\
LNS sem ML & Alta & Alta & Alta & Médio \\
LNS com ML & Alta & Muito alta & Muito alta & Médio \\
\hline
\end{tabular}
\end{table}

Observa-se que o método exato garante soluções ótimas, porém com maior custo computacional, enquanto o LNS apresenta favorável equilíbrio entre qualidade e tempo de execução. A introdução do aprendizado de máquina mostrou tendência de melhoria da estabilidade e da convergência do LNS, sem comprometer a viabilidade legal das escalas.

A comparação apresentada na Tabela~\ref{tab:comparacao-modos}
possui caráter qualitativo, pois visa sintetizar tendências observadas
em múltiplos experimentos, considerando simultaneamente métricas
quantitativas, estabilidade temporal e esforço computacional.
Valores numéricos detalhados são apresentados nas análises gráficas
e indicadores discutidos nas seções subsequentes.

\section{Operações Elementares e Análise Estrutural}

O sistema desenvolvido também permite aplicar opera{\c c}{\~o}es elementares à matriz de restrições, auxiliando em:

\begin{itemize}
    \item fins educacionais;
    \item análise estrutural avançada;
    \item depuração de modelos;
    \item estudo de dependências lineares.
\end{itemize}

A Figura~\ref{fig:operacoes-avancadas} mostra um exemplo.

\begin{figure}[H]
    \centering
    \includegraphics[width=0.92\textwidth]{template/figuras/T3.jpg}
    \caption{Operações elementares aplicadas à matriz.}
    \label{fig:operacoes-avancadas}
\end{figure}

A análise dessas operações evidencia a consistência estrutural da matriz e confirma a estabilidade do modelo sob transformações.

\section{Cenários de Rápida Convergência}

Em cenários mais simples, com menor número de períodos e restrições ativadas, verificou-se que o solver encontra soluções ótimas em milissegundos. A Figura~\ref{fig:solver-rapid} ilustra um desses casos.

\begin{figure}[H]
    \centering
    \includegraphics[width=0.92\textwidth]{template/figuras/T11.png}
    \caption{Exemplo ilustrativo de convergência rápida do solver (18 ms).}
    \label{fig:solver-rapid}
\end{figure}

Esse resultado reforça a aplicabilidade da solução em:

\begin{itemize}
    \item sistemas de resposta rápida;
    \item replanejamento dinâmico;
    \item simulações em larga escala;
    \item ambientes produtivos com grande variação de demanda.
\end{itemize}

\section{Análise Gráfica Avançada}

Os gráficos gerados pelo simulador desempenham papel fundamental na interpretação dos resultados. Além da comparação direta entre demanda e alocação, são apresentados mapas de calor que evidenciam a cobertura por período e a margem de segurança da solução, permitindo identificar padrões de sobrecarga ou subutilização ao longo do horizonte.

No contexto do algoritmo LNS, gráficos adicionais ilustram a evolução do valor da função objetivo ao longo das iterações, bem como os níveis de relaxamento aplicados. Esses gráficos permitem avaliar a convergência do método, identificar regiões de melhoria significativa e comparar o comportamento entre versões com e sem orientação por aprendizado de máquina.

Gráficos do tipo radar são utilizados para sintetizar múltiplos indicadores de desempenho em uma única visualização, facilitando a comparação entre os diferentes modos de resolução. Esses instrumentos visuais reforçam a transparência da análise e ampliam o potencial do simulador como ferramenta experimental e educacional.

\section{Síntese Geral dos Resultados}

Os experimentos realizados demonstram que:

\begin{itemize}
    \item o modelo atende às restrições modeladas às exigências do Regulamento (CE) n.{\textordmasculine} 561/2006;
    \item o solver produz alocações eficientes e homogêneas;
    \item os indicadores operacionais mostram comportamento temporal estável nos cenários analisados;
    \item a estrutura matricial favorece a resolução rápida;
    \item cenários de maior escala são resolvidos com estabilidade observada.
\end{itemize}

Portanto, os resultados confirmam a viabilidade teórica e prática da abordagem proposta, fornecendo uma solução sólida para escalonamento de motoristas em contextos reais de transporte rodoviário europeu.

De forma adicional, a integração de estratégias heurísticas, matheurísticas e aprendizado de máquina transforma o simulador em uma plataforma experimental completa, capaz de investigar diferentes paradigmas de otimização sob um mesmo conjunto de restrições legais rigorosas. Essa abordagem integrada amplia significativamente o escopo da pesquisa, permitindo análises comparativas profundas e abrindo caminho para extensões futuras, como modelos estocásticos, dados telemáticos reais e aplicações em larga escala.

Esses resultados não apenas fornecem evidências empíricas favoráveis à formulação matemática e a implementação computacional propostas, como também fornecem a base empírica necessária para uma discussão mais aprofundada sobre trade-offs, limitações e implicações práticas do modelo, a ser apresentada no capítulo seguinte.
