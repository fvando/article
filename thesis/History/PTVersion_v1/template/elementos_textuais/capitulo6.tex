\chapter{Implementação Computacional}\label{capitulo6}
\section{Introdução}

Este capítulo descreve a implementação computacional do modelo apresentado no Capítulo~4, utilizando a linguagem Python, o solver CP-SAT do OR-Tools e uma interface interativa criada em Streamlit. O objetivo da implementação é proporcionar uma plataforma integrada e flexível para simulação, análise e avaliação de cenários diversos, preservando a estrutura e as restrições legais propostas pelo Regulamento (CE) n.{\textordmasculine}~561/2006.

A solução foi construída em camadas: interface, modelagem, solver e análise de resultados. Essa abordagem modular facilita manutenção, extensibilidade e integração futura com sistemas corporativos ou plataformas SaaS.

A implementação foi concebida para operar tanto com dados reais quanto com dados sintéticos de demanda.
Nos experimentos apresentados nesta dissertação, os dados de demanda utilizados correspondem a dados reais
provenientes de operações de transporte rodoviário específicas, previamente tratados e anonimizados, garantindo
conformidade ética e confidencialidade operacional.

A possibilidade de geração sintética de demanda permanece disponível no simulador exclusivamente
para fins de testes, validação metodológica e experimentação controlada, não sendo utilizada
na análise principal dos resultados apresentados neste trabalho.

\section{Arquitetura Geral da Solução}

A arquitetura computacional é composta por quatro módulos principais:

\begin{enumerate}
    \item \textbf{Interface (Streamlit)}: coleta de parâmetros, configuração de cenários e visualização dos resultados.
    \item \textbf{Modelagem}: construção das variáveis, função objetivo e restrições.
    \item \textbf{Solver (OR-Tools)}: execução do CP-SAT, retorno de soluções, logs e tempos de execução.
    \item \textbf{Pós-processamento}: geração de gráficos, métricas e indicadores operacionais.
\end{enumerate}

Essa organização segue recomendações da literatura contemporânea em problemas combinatórios \cite{googleORTools}.

Além da formulação exata via Programação Linear Inteira, o simulador foi estendido para incorporar estratégias heurísticas e matheurísticas, permitindo a análise comparativa entre diferentes abordagens de resolução. Em particular, foram implementados três modos de solução: (i) solução exata via CP-SAT, (ii) heurística construtiva gulosa para geração de soluções iniciais e (iii) uma abordagem Large Neighborhood Search (LNS), que combina heurísticas com reotimizações locais via solver inteiro. Essa arquitetura híbrida amplia o escopo experimental da plataforma, possibilitando avaliar compromissos entre qualidade da solução e esforço computacional.

Adicionalmente, o simulador incorpora mecanismos de aprendizado de máquina supervisionado com o objetivo de guiar decisões heurísticas. Dois modelos foram introduzidos: um modelo local de atribuição motorista--período, responsável por estimar a atratividade de alocações individuais, e um modelo de avaliação de vizinhanças, utilizado para priorizar regiões promissoras no processo LNS. Esses modelos são treinados a partir de dados sintéticos gerados pelo próprio simulador, garantindo consistência entre o processo de aprendizado e o domínio do problema. Ressalta-se que a ativação do aprendizado de máquina é opcional, sendo o comportamento padrão totalmente determinístico e compatível com a heurística clássica.

Do ponto de vista metodológico, a implementação computacional foi concebida como um simulador híbrido, no qual diferentes estratégias de resolução coexistem e podem ser avaliadas sob um mesmo conjunto de parâmetros e restrições legais. A Programação Linear Inteira constitui o núcleo normativo do sistema, preservando a conformidade jurídica e consistência matemática, enquanto heurísticas, metaheurísticas e modelos de aprendizado de máquina são integrados como mecanismos auxiliares, com o objetivo de ampliar a eficiência computacional e o escopo experimental da plataforma.

Todas as soluções finais permanecem integralmente validadas pelo modelo de Programação Linear Inteira e pelo solver CP-SAT, garantindo conformidade legal e rigor matemático independentemente da ativação ou não do aprendizado de máquina.

\section{Interface de Parametrização}

A interface inicial permite ao usuário configurar:

\begin{itemize}
    \item número total de períodos;
    \item granularidade temporal;
    \item tipo de variáveis (binárias ou relaxadas);
    \item solver (CP-SAT ou SCIP);
    \item ativação ou desativação de restrições legais;
    \item demanda por período.
\end{itemize}

A Figura~\ref{fig:parametrizacao} apresenta a tela inicial da interface do sistema.

\begin{figure}[h]
    \centering
    \includegraphics[width=0.92\textwidth]{template/figuras/T0.png}
    \caption{Interface principal de parametrização da solução.}
    \label{fig:parametrizacao}
\end{figure}

\section{Configuração das Restrições Legais}

O sistema permite ativar ou desativar individualmente as restrições associadas ao Regulamento (CE) n.{\textordmasculine} 561/2006, o que possibilita:

\begin{itemize}
    \item realizar análises comparativas;
    \item simular cenários educacionais;
    \item testar hipóteses de flexibilização em ambiente experimental;
    \item avaliar impacto de restrições específicas.
\end{itemize}

A Figura~\ref{fig:restricoes} ilustra essa funcionalidade.

\begin{figure}[h]
    \centering
    \includegraphics[width=0.92\textwidth]{template/figuras/T2.jpg}
    \caption{Seção de ativação das restrições do Regulamento (CE) n.{\textordmasculine} 561/2006.}
    \label{fig:restricoes}
\end{figure}

\section{Inserção da Demanda Operacional}

A demanda por período pode ser definida:

\begin{itemize}
    \item manualmente pelo usuário;
    \item carregada a partir de arquivos externos;
    \item gerada aleatoriamente para simulação.
\end{itemize}

A Figura~\ref{fig:demanda} apresenta a demanda utilizada nos experimentos.

\begin{figure}[h]
    \centering
    \includegraphics[width=0.92\textwidth]{template/figuras/T1.jpg}
    \caption{Distribuição da demanda operacional por período.}
    \label{fig:demanda}
\end{figure}

\section{Construção da Matriz de Restrições}

A matriz de restrições é um dos componentes centrais da implementação. Ela representa formalmente:

\begin{itemize}
    \item regras legais;
    \item vínculos operacionais;
    \item continuidade temporal;
    \item acúmulo e reinício de condução;
    \item períodos de descanso.
\end{itemize}

As Figuras~\ref{fig:matriz-inicial} e~\ref{fig:matriz-final} mostram, respectivamente, a matriz inicial e a versão final após transformações elementares.

\begin{figure}[h]
    \centering
    \includegraphics[width=0.92\textwidth]{template/figuras/T122.png}
    \caption{Matriz de restrições (versão inicial).}
    \label{fig:matriz-inicial}
\end{figure}

\begin{figure}[h]
    \centering
    \includegraphics[width=0.92\textwidth]{template/figuras/T13.png}
    \caption{Matriz de restrições (versão final).}
    \label{fig:matriz-final}
\end{figure}

A estrutura esparsa da matriz (entre 15\% e 35\% de preenchimento) é favorável ao solver, conforme discutido por \cite{erdman2022}.

\section{Execução do Solver}

Após a modelagem, o CP-SAT é executado, retornando:

\begin{itemize}
    \item estado da solução (ótimo, viável, limite de tempo);
    \item custo total;
    \item valores das variáveis;
    \item métricas de desempenho;
    \item tempo computacional.
\end{itemize}

Embora o modelo seja formulado como um problema de Programação Linear Inteira,
a resolução é realizada por meio do solver CP-SAT, que combina técnicas de
programação por restrições, satisfatibilidade booleana e otimização inteira.
Essa abordagem híbrida permite explorar de forma eficiente a estrutura temporal
e esparsa do modelo, mitigando limitações observadas de solvers MILP tradicionais.

Um exemplo de execução é mostrado na Figura~\ref{fig:solver}.

\begin{figure}[h]
    \centering
    \includegraphics[width=0.92\textwidth]{template/figuras/T11.png}
    \caption{Execução do solver CP-SAT (solução ótima).}
    \label{fig:solver}
\end{figure}

Tempos inferiores a ordem de segundos foram observados nos experimentos realizados mesmo para cenários com milhares de variáveis.

\section{Fluxo Computacional do Simulador}

Do ponto de vista computacional, o simulador segue um pipeline bem definido, composto pelas
seguintes etapas sequenciais:

\begin{enumerate}
    \item leitura e validação dos parâmetros de entrada;
    \item construção do modelo matemático (variáveis, restrições e função objetivo);
    \item execução do método de resolução selecionado (exato, heurístico ou LNS);
    \item pós-processamento da solução obtida;
    \item cálculo de indicadores operacionais e geração de visualizações.
\end{enumerate}

Esse pipeline é executado de forma determinística para um conjunto fixo de parâmetros,
favorecendo reprodutibilidade dos experimentos realizados.

O fluxo de execução do simulador segue uma sequência bem definida de etapas, independentemente do modo de resolução selecionado. Inicialmente, o usuário configura os parâmetros do problema por meio da interface interativa, incluindo horizonte temporal, demanda, restrições legais e estratégia de solução. Em seguida, o sistema constrói internamente a representação matricial do problema, instanciando variáveis, restrições e a função objetivo conforme o modelo matemático descrito no Capítulo~4.

No modo exato, o modelo completo é submetido diretamente ao solver CP-SAT, que realiza a busca por soluções ótimas respeitando todas as restrições. Nos modos heurístico e LNS, o sistema primeiro gera uma solução inicial viável, que pode ser utilizada isoladamente ou como ponto de partida para processos iterativos de melhoria. No caso do LNS, subconjuntos de períodos são liberados a cada iteração, mantendo o restante da solução fixo, e um problema restrito é resolvido novamente pelo solver inteiro.

Após a obtenção da solução final, o simulador executa uma etapa de pós-processamento, responsável pelo cálculo de indicadores operacionais, métricas de desempenho e geração de visualizações gráficas. Esse fluxo modular permite comparar diferentes estratégias sob condições controladas, garantindo reprodutibilidade e consistência nos experimentos realizados.

\section{Modos de Resolução e Estratégias de Otimização}

É importante distinguir, no contexto do simulador, os conceitos de \textit{modo de resolução}
e \textit{modo de otimização}. O modo de resolução define a estratégia algorítmica utilizada
(exata, heurística ou matheurística), enquanto o modo de otimização define a função objetivo
adotada pelo modelo matemático (maximização da resposta à demanda ou minimização do número
total de motoristas).

Essa separação conceitual permite combinar livremente diferentes estratégias de resolução
com diferentes objetivos operacionais, ampliando o escopo experimental da plataforma.

O simulador permite a execução do problema de escalonamento em diferentes modos de resolução. No modo exato, o modelo completo de Programação Linear Inteira é resolvido pelo solver CP-SAT, permitindo a obtenção de soluções ótimas quando o tempo computacional é suficiente. No modo heurístico, uma estratégia construtiva gulosa gera rapidamente uma solução viável, priorizando o atendimento da demanda com menor complexidade computacional. Já no modo LNS (Large Neighborhood Search), o sistema explora iterativamente vizinhanças da solução corrente, liberando subconjuntos de períodos e reotimizando-os via solver inteiro, o que permite melhorias progressivas da solução com controle explícito do esforço computacional.

\section{Geração de Indicadores e Gráficos}

O sistema produz gráficos essenciais para avaliação da qualidade relativda da solução:

\begin{itemize}
    \item cobertura da demanda;
    \item sobrecarga;
    \item subutilização;
    \item distribuição da quantidade de motoristas ao longo dos períodos;
    \item desvio padrão da cobertura.
\end{itemize}

A Figura~\ref{fig:indicadores} apresenta alguns desses resultados.

\begin{figure}[h]
    \centering
    \includegraphics[width=0.92\textwidth]{template/figuras/T3.jpg}
    \caption{Indicadores de cobertura, sobrecarga e subutilização.}
    \label{fig:indicadores}
\end{figure}

Já a Figura~\ref{fig:demanda-vs-cobertura} destaca a comparação entre demanda e cobertura.

\begin{figure}[h]
    \centering
    \includegraphics[width=0.92\textwidth]{template/figuras/T12.png}
    \caption{Demanda comparada à cobertura gerada pelo modelo.}
    \label{fig:demanda-vs-cobertura}
\end{figure}

\section{Módulo de Operações Elementares}

Além de resolver o problema, a aplicação permite aplicar operações elementares à matriz de restrições, possibilitando:

\begin{itemize}
    \item fins educacionais (ensino de álgebra linear);
    \item análise estrutural da matriz;
    \item diagnósticos estruturais.
\end{itemize}

Exemplo ilustrado na Figura~\ref{fig:operacoes}.

\begin{figure}[h]
    \centering
    \includegraphics[width=0.92\textwidth]{template/figuras/T3.jpg}
    \caption{Operações elementares aplicadas à matriz de restrições.}
    \label{fig:operacoes}
\end{figure}

Os gráficos analíticos gerados pelo simulador desempenham papel central na avaliação da qualidade das soluções. O gráfico de cobertura da demanda permite verificar, período a período, se o número de motoristas alocados atende ou excede a demanda operacional. O gráfico de margem de segurança evidencia excessos ou déficits de alocação, auxiliando na identificação de sobrecargas ou subutilizações. Já os gráficos temporais de estabilidade e eficiência fornecem uma visão agregada do comportamento da solução ao longo do horizonte, indicando o grau de regularidade das escalas e o nível de aproveitamento da força de trabalho.

No modo LNS, gráficos adicionais apresentam o histórico das iterações, incluindo a evolução do valor da função objetivo e a identificação de melhorias sucessivas. Esses gráficos permitem analisar o processo de convergência do algoritmo, evidenciando como a solução é refinada ao longo das iterações e em quais momentos ocorrem ganhos significativos. Essa visualização é particularmente útil para estudos experimentais, pois permite comparar estratégias de vizinhança, critérios de aceitação e parâmetros de relaxação.

\section{Extensões Heurísticas e Metaheurísticas}

Além da resolução exata via Programação Linear Inteira, o simulador foi estendido para incorporar métodos heurísticos e metaheurísticos, com o objetivo de ampliar a capacidade de experimentação, reduzir tempos computacionais em cenários de maior escala e permitir análises comparativas entre diferentes estratégias de solução.

Inicialmente, foi implementada uma heurística gulosa (\textit{greedy initial allocation}), responsável por gerar soluções viáveis iniciais a partir da demanda por período. Essa heurística prioriza a alocação de motoristas nos períodos de maior demanda, respeitando limites operacionais básicos, e serve tanto como solução independente quanto como ponto de partida para métodos mais avançados.

Em seguida, foi integrado um método de \textit{Large Neighborhood Search} (LNS), caracterizado pela destruição e reconstrução parcial da solução. A cada iteração, subconjuntos de períodos são liberados, mantendo o restante da solução fixo, e um problema restrito é resolvido novamente via solver inteiro. Essa abordagem combina características da otimização exata com a flexibilidade de exploração típica das metaheurísticas.

\section{Integração de Aprendizado de Máquina}

O simulador incorpora um módulo experimental de aprendizado de máquina com o objetivo de orientar decisões locais e estratégicas durante o processo de otimização. Essa integração não substitui o modelo matemático, mas atua como um mecanismo de apoio, mantendo compatibilidade com a solução exata.

Foram definidos dois modelos supervisionados distintos. O primeiro, denominado $f_1$, atua no nível local, atribuindo escores à decisão de alocar um determinado motorista em um período específico durante a heurística gulosa. O segundo modelo, $f_2$, opera em nível agregado, avaliando vizinhanças candidatas no algoritmo LNS, priorizando aquelas com maior potencial de melhoria.

A arquitetura foi concebida de forma resiliente: na ausência de modelos treinados ou do ambiente de aprendizado de máquina, o sistema automaticamente recorre a regras heurísticas clássicas, garantindo que o comportamento do simulador permaneça estável e reproduzível. Dessa forma, o aprendizado de máquina atua como uma camada opcional de inteligência adicional, sem comprometer a validade dos experimentos.

\section{Geração de Datasets e Treinamento dos Modelos}

Para viabilizar a integração com aprendizado de máquina, foi desenvolvido um pipeline automático de geração de dados supervisionados. A partir de múltiplas instâncias sintéticas de demanda, o simulador executa tanto soluções heurísticas quanto soluções ótimas via solver inteiro, registrando decisões e resultados intermediários.

Esses dados são organizados em dois conjuntos distintos: um dataset voltado ao treinamento do modelo $f_1$, contendo informações locais sobre motorista, período e contexto operacional; e um segundo dataset para o modelo $f_2$, composto por características agregadas das vizinhanças exploradas pelo LNS.

O treinamento dos modelos é realizado externamente ao fluxo principal de otimização, utilizando algoritmos de \textit{gradient boosting} (LightGBM). Os modelos resultantes são então carregados dinamicamente pelo simulador, permitindo a avaliação do impacto do aprendizado supervisionado sobre a qualidade e a estabilidade das soluções geradas.

\section{Visualização Avançada e Análise Iterativa}

Além dos gráficos tradicionais de cobertura e atendimento da demanda, o simulador passou a oferecer visualizações avançadas que permitem uma análise mais profunda do comportamento dos algoritmos ao longo do tempo.

No contexto do LNS, são gerados gráficos que mostram a evolução do valor da função objetivo por iteração, a identificação de melhorias sucessivas e o comportamento do nível de relaxamento aplicado. Esses gráficos permitem avaliar empiricamente a convergência do método e comparar o desempenho entre estratégias exatas, heurísticas e metaheurísticas.

Adicionalmente, foram incluídos mapas de calor e métricas temporais que evidenciam padrões de subutilização, sobrecarga e estabilidade da solução, ampliando a capacidade analítica do simulador e fortalecendo sua utilização tanto para pesquisa quanto para fins educacionais.

\section{Posicionamento do Simulador como Plataforma Experimental}

Com as extensões implementadas, o simulador ultrapassa o papel de uma simples ferramenta de resolução de problemas de Programação Linear Inteira, consolidando-se como uma plataforma experimental híbrida. Ele permite a comparação sistemática entre métodos exatos, heurísticos, metaheurísticos e abordagens híbridas com aprendizado de máquina, sob um mesmo conjunto de restrições legais rigorosas.

Essa flexibilidade torna o sistema particularmente adequado para estudos de sensibilidade, avaliação de escalabilidade e investigação de estratégias avançadas de otimização aplicada ao transporte rodoviário. Além disso, a arquitetura modular facilita futuras extensões, como modelos estocásticos, integração com dados telemáticos reais e aplicações em ambientes corporativos ou SaaS.

É importante destacar que todas as extensões implementadas — heurísticas, metaheurísticas e aprendizado de máquina — foram concebidas como camadas complementares ao modelo matemático exato, não comprometendo a conformidade normativa no escopo do modelo. Em particular, o uso de aprendizado supervisionado limita-se à priorização de decisões locais e à seleção de vizinhanças promissoras, sem alterar diretamente as restrições ou a função objetivo do problema. Dessa forma, o simulador preserva rigor científico e transparência metodológica, ao mesmo tempo em que amplia significativamente seu potencial experimental.

\section{Considerações Finais}

A implementação computacional desenvolvida mostrou-se:

\begin{itemize}
    \item eficiente;
    \item escalável;
    \item robusta;
    \item flexível para análise de cenários;
    \item eficiente em termos computacionais;
    \item escalável nos cenários avaliados;
    \item estável sob diferentes configurações;
    \item flexível para análise de cenários.

\end{itemize}

O uso combinado de PLI, OR-Tools e Streamlit permite visualizar, interpretar e validar facilmente soluções complexas, oferecendo um ambiente completo para experimentação científica e aplicação prática no setor de transporte rodoviário europeu.

Dessa forma, a implementação computacional transcende a simples resolução de um modelo de Programação Linear Inteira, constituindo uma plataforma experimental completa para o estudo de escalonamento de motoristas sob restrições legais complexas. A combinação de modelagem exata, heurísticas, matheurísticas, aprendizado de máquina e visualização interativa fornece uma base metodológica para investigações científicas e estudos aplicados avançadas, alinhando-se aos objetivos desta dissertação.

A implementação computacional descrita neste capítulo fornece a base operacional
necessária para a realização dos experimentos apresentados no capítulo seguinte.
Todos os resultados, indicadores e análises discutidos no Capítulo~7 foram gerados
diretamente a partir do simulador aqui descrito, mantendo consistência entre modelo matemático, implementação computacional e análise empírica.