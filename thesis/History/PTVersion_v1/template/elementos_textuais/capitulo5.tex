\chapter{Modelo Matemático}\label{capitulo5}

\section{Introdução}

Este capítulo apresenta o modelo de Programa{\c c}{\~a}o Linear Inteira (PLI) desenvolvido para o escalonamento de motoristas sob o Regulamento (CE) n.{\textordmasculine} 561/2006. O objetivo do modelo é gerar escalas em conformidade com as restrições legais, eficientes e capazes de atender à demanda operacional por período, incorporando as restrições definidas pela legislação europeia consideradas neste trabalho.

O modelo construído estende a formulação proposta em \cite{moreira2025}, incorporando maior detalhamento temporal, variáveis auxiliares e mecanismos de controle lógico adequados para implementação no solver CP-SAT do OR-Tools.

O modelo foi concebido de forma flexível, permitindo a adoção de diferentes funções objetivo conforme o cenário operacional analisado. Em particular, duas estratégias de otimização são consideradas no simulador desenvolvido: (i) a maximização da resposta à demanda, priorizando a cobertura operacional, e (ii) a minimização do número total de motoristas alocados, priorizando eficiência de recursos. Ambas as abordagens compartilham o mesmo conjunto de variáveis e restrições legais consideradas, diferenciando-se apenas na função objetivo e no tratamento da restrição de atendimento da demanda.

\section{Discretização Temporal}

O horizonte de planejamento é dividido em um conjunto de períodos discretos:
\[
T = \{1,2,\ldots, |T|\},
\]
onde cada período representa uma janela fixa de tempo com duração:
\[
\Delta \ \ (\text{em horas}).
\]

Essa discretização permite rastrear a condução acumulada, pausas, descansos e demais restrições temporais do modelo.

\section{Conjuntos e Índices}

\begin{itemize}
    \item \(D\): conjunto de motoristas (\(d \in D\));
    \item \(T\): conjunto de períodos discretizados (\(t \in T\));
    \item \(W\): janelas agregadas de planejamento:
    \begin{itemize}
        \item dias,
        \item semanas,
        \item quinzenas;
    \end{itemize}
    \item \(P\): parâmetros de demanda por período.
\end{itemize}

\section{Parâmetros do Modelo}

\begin{align*}
demanda_t     &: \text{demanda mínima de motoristas no período } t, \\
\Delta               &: \text{duração de cada período (em horas)}, \\
L^{\text{dia}}       &= 9 \quad \text{(limite diário de condução)}, \\
L^{\text{ext}}       &= 10 \quad \text{(extensão diária permitida 2 vezes/semana)}, \\
L^{\text{sem}}       &= 56 \quad \text{(limite semanal)}, \\
L^{14d}              &= 90 \quad \text{(limite em 14 dias)}, \\
R_{\text{dia}}       &= 11 \quad \text{(descanso diário normal)}, \\
R_{\text{dia-red}}   &= 9 \quad \text{(descanso diário reduzido)}, \\
R_{\text{sem}}       &= 45 \quad \text{(descanso semanal)}, \\
P_{\text{pausa}}     &= 4.5 \quad \text{(máximo contínuo de condução antes da pausa)}.
\end{align*}

Esses parâmetros refletem as exigências legais discutidas em \cite{eu5612006}, \cite{eu200215}.

\section{Variáveis de Decisão}

\subsection*{Variável Principal de Alocação}
\[
x_{d,t} =
\begin{cases}
1, & \text{se o motorista } d \text{ trabalha no período } t, \\
0, & \text{caso contrário}.
\end{cases}
\]

\subsection*{Variável de Descanso}
\[
r_{d,t} =
\begin{cases}
1, & \text{se o motorista } d \text{ descansa no período } t, \\
0, & \text{caso contrário}.
\end{cases}
\]

\subsection*{Variável de Início de Jornada}
\[
s_{d,t} =
\begin{cases}
1, & \text{se a jornada de } d \text{ inicia em } t, \\
0, & \text{caso contrário}.
\end{cases}
\]

\subsection*{Variável de Condução Acumulada}
\[
h_{d,t} \ge 0.
\]

\subsection*{Variável de Extensão Diária}
\[
z_{d,w} =
\begin{cases}
1, & \text{se o motorista utiliza extensão diária na semana } w, \\
0, & \text{caso contrário}.
\end{cases}
\]

\section{Funções-Objetivo do Modelo}

O modelo matemático proposto admite duas funções-objetivo alternativas,
selecionadas de acordo com o objetivo operacional definido no simulador.
Ambas compartilham o mesmo conjunto de variáveis e restrições legais,
diferindo apenas no critério de otimização adotado.

A variável \(u_t\) utilizada no modo de maximização da resposta à demanda
é diretamente explorada nos indicadores apresentados no Capítulo~6,
em especial nos índices de subutilização, cobertura e risco operacional.
Dessa forma, o modelo matemático estabelece uma ligação direta entre
a formulação teórica e os mecanismos analíticos do simulador.

\subsection{Maximização da Resposta à Demanda}

Neste modo, o objetivo do modelo é maximizar o atendimento da demanda
operacional ao longo do horizonte de planejamento, penalizando períodos
não cobertos.

Define-se a variável auxiliar:
\[
u_t \ge 0, \quad u_t \in \mathbb{R} \quad \text{(demanda não atendida no período } t).
\]

Sujeita à restrição:
\[
\sum_{d \in D} x_{d,t} + u_t = demanda_t, \quad \forall t \in T.
\]

A função-objetivo é então dada por:
\[
\max Z_1 = \sum_{t \in T} (demanda_t - u_t),
\]
ou, de forma equivalente,
\[
\min \sum_{t \in T} u_t.
\]

Esse critério prioriza a maximização da cobertura, sendo particularmente útil em cenários de alta variabilidade ou restrição severa de recursos.

\subsection{Minimização do Número Total de Motoristas}

Neste modo, o objetivo consiste em minimizar o uso total de motoristas
ao longo do horizonte de planejamento, considerando a hipótese de que toda a demanda seja integralmente atendida.

A função-objetivo é definida como:
\[
\min Z_2 = \sum_{d \in D} \sum_{t \in T} x_{d,t}.
\]

Esse critério busca eficiência operacional e redução de custos de mão de obra,
sendo apropriado para cenários estáveis ou de planejamento estratégico.

Observa-se que a minimização do somatório \(\sum_{d,t} x_{d,t}\)
corresponde à redução do volume total de trabalho alocado.
Caso seja desejado minimizar explicitamente o número de motoristas distintos,
pode-se introduzir uma variável binária \(u_d\) indicando se o motorista \(d\)
é utilizado no horizonte, com a restrição \(x_{d,t} \le u_d\).
No simulador desenvolvido, optou-se pela formulação apresentada,
por refletir diretamente o custo operacional associado à utilização efetiva da mão de obra.

\subsection{Observação sobre a Seleção do Critério}

No simulador desenvolvido, o usuário pode selecionar dinamicamente
qual função-objetivo será utilizada em cada experimento.
Essa escolha não altera a estrutura do modelo, apenas o critério de otimização,
permitindo análises comparativas entre estratégias focadas em cobertura
ou eficiência operacional.

\section{Restrições do Modelo}
\subsection{Atendimento da Demanda}

O tratamento da demanda depende do modo de otimização selecionado:

\begin{itemize}
    \item No modo de minimização de motoristas, a demanda constitui uma restrição rígida;
    \item No modo de maximização da resposta à demanda, admite-se demanda não atendida por meio de variáveis auxiliares.
\end{itemize}

Essa formulação permite analisar compromissos entre cobertura operacional e disponibilidade de recursos, sem comprometer a conformidade legal no escopo do modelo.

\subsection{Exclusividade de Estado por Período}
Cada motorista deve estar em trabalho ou descanso:
\[
x_{d,t} + r_{d,t} = 1, \quad \forall d,t.
\]

\subsection{Acúmulo de Condução}
\[
h_{d,t} \le h_{d,t-1} + \Delta x_{d,t}.
\]

A dinâmica de reinício do contador de condução após períodos de descanso
é implementada por meio de inequações com Big-M,
evitando a introdução direta de igualdades não lineares e mantendo a compatibilidade
com o solver CP-SAT.

Reinício após descanso (linearização com Big-M):
\[
h_{d,t} \le M (1 - r_{d,t}).
\]

\subsection{Limite Diário de Condução}
Para cada motorista e cada dia:
\[
\sum_{t \in w_{\text{dia}}} \Delta x_{d,t} \le L^{\text{dia}} + z_{d,w}.
\]

\subsection{Limite Semanal e Quinzenal}
\[
\sum_{t \in w_{\text{sem}}} \Delta x_{d,t} \le L^{\text{sem}},
\]
\[
\sum_{t \in w_{14d}} \Delta x_{d,t} \le L^{14d}.
\]

\subsection{Pausa Obrigatória após 4,5 horas}
\[
h_{d,t} \le P_{\text{pausa}} + M r_{d,t}.
\]

\subsection{Descanso Diário Normal ou Reduzido}

Para cada janela deslizante de 24 horas:
\[
\sum_{t \in \text{janela}_{24h}} \Delta r_{d,t} \ge R_{\text{dia}},
\]
ou, caso reduzido:
\[
\sum_{t \in \text{janela}_{24h}} \Delta r_{d,t} \ge R_{\text{dia-red}}.
\]

\subsection{Descanso Semanal}
\[
\sum_{t \in w_{\text{sem}}} \Delta r_{d,t} \ge R_{\text{sem}}.
\]

\subsection{Limite de Extensões Diárias}
\[
\sum_{w \in \text{semana}} z_{d,w} \le 2.
\]

\subsection{Continuidade da Jornada}
Evita alternância inválida entre estados:
\[
x_{d,t-1} - x_{d,t} \le r_{d,t}.
\]

\section{Discussão sobre Linearização}

O Regulamento (CE) n.{\textordmasculine} 561/2006 contém várias dependências temporais não-lineares, especialmente em regras de:

\begin{itemize}
    \item descanso diário e semanal,
    \item condução acumulada,
    \item pausas após condução contínua,
    \item limites quinzenais,
    \item inícios de jornada.
\end{itemize}

Para permitir resolução via PLI, foi necessário:

\begin{itemize}
    \item discretizar o horizonte em períodos fixos;
    \item introduzir variáveis auxiliares (\(h_{d,t}, r_{d,t}, s_{d,t}, z_{d,w}\));
    \item usar técnicas de linearização com Big-M;
    \item incorporar janelas móveis de 24h, 7 dias e 14 dias.
\end{itemize}

Esses cuidados permitem a resolução do modelo por solvers inteiros como CP-SAT, que se aproveita de sua estrutura esparsa e temporal \cite{googleORTools}.

\section{Observações Computacionais}

O modelo completo apresenta:

\begin{itemize}
    \item dezenas de milhares de variáveis binárias;
    \item milhares de restrições lineares;
    \item matriz esparsa com estrutura quase diagonal;
    \item encadeamento temporal forte.
\end{itemize}

Ainda assim, o CP-SAT resolve cen{\'a}rios de 7 a 30 dias em poucos segundos, conforme validado no Capítulo~6, o que indica seu potencial de aplicabilidade prática e alinhamento com estudos que aplicam otimização a problemas temporais complexos \cite{pillac2013,erdman2022}.

\section{Considerações Finais}

O modelo matemático apresentado formaliza o escalonamento de motoristas sob a legislação europeia. Com isso, fornece a base científica necessária para a implementação computacional descrita no capítulo seguinte e para as análises empíricas realizadas posteriormente.

Embora o modelo seja resolvido de forma exata, sua estrutura também permite a integração com heurísticas e métodos de aprendizado supervisionado, utilizados posteriormente para geração de soluções iniciais e exploração de vizinhanças, sem alterar a formulação matemática.