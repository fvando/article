\chapter{Introdução}\label{capitulo1}

Os elementos pré-textuais apresentados a seguir têm como finalidade orientar o leitor acerca da estrutura gráfica, simbólica e organizacional desta dissertação, dedicada ao desenvolvimento de um modelo de Programação Linear Inteira para o escalonamento de motoristas sob o Regulamento (CE) n.{\textordmasculine} 561/2006.

Ao longo do trabalho, diferentes tipos de representações são utilizados para apoiar a formulação matemática, a implementação computacional e a análise dos resultados. A Lista de Ilustrações reúne as figuras que apresentam, entre outros elementos, a interface do sistema construído, a estrutura da matriz de restrições e os gráficos derivados das simulações realizadas. As listas de Quadros e Tabelas são mantidas para conformidade com as normas acadêmicas, mesmo que o texto não contenha quadros ou tabelas formais. 

A Lista de Abreviaturas e Siglas consolida os termos técnicos utilizados no campo da Pesquisa Operacional, da otimização e da legislação europeia aplicada ao transporte rodoviário, facilitando o acompanhamento da leitura. A Lista de Símbolos organiza as notações matemáticas empregadas na formulação do modelo, servindo como referência rápida para os capítulos que apresentam o formalismo teórico. 

Por fim, o Sumário estrutura a organização geral do documento, permitindo uma navegação clara entre as seções que tratam desde o contexto e fundamentação teórica até os experimentos, resultados e conclusões finais. Esses elementos auxiliam na compreensão integral da pesquisa, conferindo ao trabalho maior clareza, rigor e acessibilidade.

%\addcontentsline{toc}{chapter}{Elementos Pré-Textuais}

\section{Contextualização}

O transporte rodoviário de cargas é um dos pilares da economia europeia, assegurando fluxos contínuos de bens e garantindo a integração logística entre os Estados-Membros. Nesse contexto, os motoristas profissionais constituem o recurso humano central da operação, tornando a gestão das suas jornadas um fator crítico para a eficiência, segurança e conformidade legal do setor.

Mais recentemente, abordagens híbridas que combinam otimização exata, heurísticas e, de forma complementar, técnicas de aprendizado de máquina têm sido exploradas como forma de acelerar a convergência e melhorar a qualidade das soluções iniciais, especialmente em problemas de grande escala e natureza temporal.

A Uni{\~a}o Europeia possui uma das regulamenta{\c c}{\~o}es mais rigorosas do mundo no que diz respeito aos tempos de condu{\c c}{\~a}o, pausas e per{\'\i}odos de descanso dos motoristas. O \textit{Regulamento (CE) n.{\textordmasculine} 561/2006}, complementado pela \textit{Diretiva 2002/15/CE} e pelo \textit{Regulamento (UE) n.{\textordmasculine} 165/2014}, estabelece limites estritos que visam proteger a sa{\'u}de dos motoristas, prevenir acidentes e promover condi{\c c}{\~o}es de trabalho seguras e humanizadas \cite{eu5612006,eu200215,eu1652014}. Essas normas afetam diretamente a forma como as escalas s{\~a}o constru{\'\i}das, exigindo aten{\c c}{\~a}o constante ao cumprimento de requisitos di{\'a}rios, semanais e quinzenais.

Entretanto, na pr{\'a}tica, a elabora{\c c}{\~a}o manual de escalas {\'e} altamente propensa a erros, especialmente em opera{\c c}{\~o}es de grande escala, caracterizadas por alta variabilidade de demanda, m{\'u}ltiplas janelas temporais, restri{\c c}{\~o}es acumulativas e depend{\^e}ncias temporais entre os per{\'\i}odos de trabalho. Pequenas decis{\~o}es tomadas em um per{\'\i}odo podem invalidar toda a escala futura, resultando em infra{\c c}{\~o}es legais, custos adicionais e riscos operacionais.

Nesse cen{\'a}rio, abordagens matem{\'a}ticas baseadas em Programa{\c c}{\~a}o Linear Inteira (PLI) emergem como ferramentas adequadas para formalizar e resolver problemas de escalonamento, oferecendo rigor, precis{\~a}o e capacidade de lidar com m{\'u}ltiplas restri{\c c}{\~o}es simultaneamente \cite{hillier2015}, \cite{taha2017}, \cite{nemhauser1988}. De forma complementar, solvers modernos como o OR-Tools disponibilizam recursos computacionais avan{\c c}ados que permitem resolver modelos inteiros de grande escala em tempos reduzidos \cite{googleORTools}.

Além do modelo de otimização em si, esta pesquisa também materializa um ambiente de simulação e análise, no qual o usuário pode configurar parâmetros operacionais, habilitar/desabilitar restrições legais e observar, de forma visual, como tais escolhas impactam a viabilidade, o número de motoristas necessários e a qualidade da cobertura da demanda. Esse ambiente inclui métricas e gráficos de diagnóstico (por exemplo, cobertura por slot, subutilização, sobrecarga, risco e estabilidade temporal), bem como análises estruturais do modelo por meio de visualizações de matrizes e medidas como densidade, oferecendo suporte à interpretação do comportamento do solver e à validação do modelo.


Esta disserta{\c c}{\~a}o insere-se nesse contexto, propondo um modelo completo de PLI para o escalonamento de motoristas em conformidade com o Regulamento (CE) n.{\textordmasculine} 561/2006, apoiado por uma implementa{\c c}{\~a}o computacional interativa e orientada {\`a} simula{\c c}{\~a}o de cen{\'a}rios operacionais. Parte desse desenvolvimento foi previamente publicada em \cite{moreira2025}, que fundamenta e inspira a formula{\c c}{\~a}o aprofundada apresentada nesta pesquisa.

\section{Problema da Pesquisa}

O problema desta disserta{\c c}{\~a}o consiste em determinar como alocar motoristas ao longo de um horizonte temporal discretizado, de forma a:
\begin{itemize}
    \item atender {\`a} demanda operacional por per{\'\i}odo;
    \item respeitar integralmente o Regulamento (CE) n.{\textordmasculine} 561/2006;
    \item minimizar o n{\'u}mero total de motoristas utilizados;
    \item garantir escalas cont{\'\i}nuas, regulares e operacionalmente vi{\'a}veis;
    \item permitir an{\'a}lises de sensibilidade para diferentes cen{\'a}rios.
\end{itemize}

Adicionalmente, quando o modelo exato se torna computacionalmente custoso em instâncias maiores, torna-se relevante investigar estratégias de busca e melhoria incremental de soluções, incorporadas no próprio simulador, tais como heurísticas construtivas e busca em grandes vizinhanças, preservando a conformidade legal e permitindo análises comparativas entre modos de resolução.

Trata-se de um contexto em que as {\`a}s restri{\c c}{\~o}es acumulativas sobre condu{\c c}{\~a}o di{\'a}ria, semanal e quinzenal; {\`a} necessidade de pausas obrigat{\'o}rias; {\`a} variabilidade da demanda; {\`a}s depend{\^e}ncias temporais entre per{\'\i}odos sucessivos; e ao grande n{\'u}mero de combina{\c c}{\~o}es poss{\'\i}veis entre motoristas e per{\'\i}odos.

\section{Quest{\~a}o da Pesquisa}

A pesquisa {\'e} guiada pelo seguinte contexto:

\begin{quote}
Como formular e implementar um modelo de Programa{\c c}{\~a}o Linear Inteira capaz de gerar escalas de motoristas legalmente v{\'a}lidas, operacionalmente eficientes e alinhadas ao Regulamento (CE) n.{\textordmasculine} 561/2006?
\end{quote}

\section{Objetivo Geral}

Desenvolver, implementar e avaliar um modelo de Programa{\c c}{\~a}o Linear Inteira para o escalonamento de motoristas no transporte rodovi{\'a}rio europeu, assegurando conformidade com o Regulamento (CE) n.{\textordmasculine} 561/2006 e promovendo efici{\^e}ncia operacional.

\section{Objetivos Espec{\'\i}ficos}

\begin{itemize}
    \item Formular matematicamente o problema, definindo vari{\'a}veis, par{\^a}metros e restri{\c c}{\~o}es;
    \item Representar as exig{\^e}ncias legais do Regulamento (CE) n.{\textordmasculine} 561/2006 sob a forma de inequa{\c c}{\~o}es lineares;
    \item Implementar o modelo utilizando Python e o solver OR-Tools;
    \item Desenvolver uma interface interativa para simula{\c c}{\~a}o de cen{\'a}rios e visualiza{\c c}{\~a}o de resultados;
    \item Validar o modelo em diferentes horizontes (7, 15 e 30 dias);
    \item Comparar a aloca{\c c}{\~a}o gerada com a demanda real ou simulada;
    \item Avaliar m{\'e}tricas de desempenho, como cobertura, subutiliza{\c c}{\~a}o e sobrecarga;
    \item Analisar a estabilidade do modelo e o impacto das transforma{\c c}{\~o}es matriciais.
    
    \item Implementar e avaliar modos complementares de resolução e melhoria de soluções (por exemplo, geração de solução inicial e busca em vizinhanças), bem como mecanismos de relaxação e registro de histórico de convergência;
    \item Estruturar rotinas de experimentação com geração de instâncias/datasets e comparação sistemática de desempenho por meio de métricas e visualizações no simulador.
    
\end{itemize}

\section{Justificativa}

Esta pesquisa {\'e} relevante por diversas raz{\~o}es:

\begin{itemize}
    \item \textbf{Relev{\^a}ncia legal e seguran{\c c}a rodovi{\'a}ria}: o descumprimento do Regulamento (CE) n.{\textordmasculine} 561/2006 implica multas, riscos {\`a} vida dos motoristas e impactos negativos na reputa{\c c}{\~a}o das empresas;
    \item \textbf{Complexidade operacional}: a variabilidade de demanda e a natureza acumulativa das restri{\c c}{\~o}es dificultam a elabora{\c c}{\~a}o manual de escalas;
    \item \textbf{Avan{\c c}o cient{\'\i}fico}: poucos trabalhos integram o Regulamento (CE) n.{\textordmasculine} 561/2006 a modelos inteiros formais; esta disserta{\c c}{\~a}o contribui para preencher essa lacuna, em linha com \cite{erdman2022}, \cite{moreira2025};
    \item \textbf{Aplicabilidade industrial}: o modelo possui potencial de uso imediato em empresas de transporte, podendo evoluir para solu{\c c}{\~o}es SaaS e integra{\c c}{\~a}o com sistemas telem{\'a}ticos.
\end{itemize}

\section{Estrutura da Disserta{\c c}{\~a}o}

Esta disserta{\c c}{\~a}o est{\'a} organizada nos seguintes cap{\'\i}tulos:

\begin{itemize}
    \item \textbf{Cap{\'\i}tulo 1} -- Introdu{\c c}{\~a}o.
    \item \textbf{Cap{\'\i}tulo 2} -- Revis{\~a}o da literatura sobre Programa{\c c}{\~a}o Linear Inteira, escalonamento, legisla{\c c}{\~a}o europeia e trabalhos relacionados.
    \item \textbf{Cap{\'\i}tulo 3} -- Metodologia, detalhando a abordagem e as etapas da pesquisa.
    \item \textbf{Cap{\'\i}tulo 4} -- Modelo matem{\'a}tico proposto.
    \item \textbf{Cap{\'\i}tulo 5} -- Implementa{\c c}{\~a}o computacional.
    \item \textbf{Cap{\'\i}tulo 6} -- Resultados obtidos e an{\'a}lise.
    \item \textbf{Cap{\'\i}tulo 7} -- Conclus{\~a}o e trabalhos futuros.
\end{itemize}
