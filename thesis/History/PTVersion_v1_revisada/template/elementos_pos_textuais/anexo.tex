\begin{anexosenv}
\partanexos

\chapter{Resumo Estruturado do Regulamento (CE) n.º 561/2006}

Este anexo apresenta um resumo das principais disposições do Regulamento (CE)
n.º~561/2006, organizado em formato tabular para facilitar consulta durante a
interpretação do modelo de escalonamento.

\section*{Limites de Condução}

\begin{table}[h!]
\centering
\caption{Limites de Condução}
\begin{tabular}{|p{5cm}|p{8cm}|}
\hline
\textbf{Parâmetro} & \textbf{Descrição} \\ \hline
Condução diária & Máximo de 9 horas; permitido estender para 10 horas até 2 vezes por semana. \\ \hline
Condução semanal & Máximo de 56 horas. \\ \hline
Condução quinzenal (2 semanas) & Máximo de 90 horas acumuladas. \\ \hline
\end{tabular}
\end{table}

\section*{Pausas Obrigatórias}

\begin{table}[h!]
\centering
\caption{Regras de Pausa}
\begin{tabular}{|p{5cm}|p{8cm}|}
\hline
\textbf{Condição} & \textbf{Descrição} \\ \hline
Pausa após condução contínua & 45 minutos após no máximo 4h30 de condução. \\ \hline
Divisão da pausa & Permitido dividir em 15 min + 30 min. \\ \hline
\end{tabular}
\end{table}

\section*{Repousos Diários}

\begin{table}[h!]
\centering
\caption{Repousos Diários}
\begin{tabular}{|p{5cm}|p{8cm}|}
\hline
\textbf{Tipo} & \textbf{Descrição} \\ \hline
Repouso regular diário & Mínimo 11 horas. \\ \hline
Repouso diário fracionado & 3h + 9h (total mínimo 12h). \\ \hline
Repouso reduzido diário & Mínimo de 9 horas, até 3 vezes entre dois repousos semanais. \\ \hline
\end{tabular}
\end{table}

\section*{Repouso Semanal}

\begin{table}[h!]
\centering
\caption{Repousos Semanais}
\begin{tabular}{|p{5cm}|p{8cm}|}
\hline
\textbf{Tipo} & \textbf{Descrição} \\ \hline
Repouso semanal regular & Mínimo de 45 horas. \\ \hline
Repouso semanal reduzido & Mínimo de 24 horas, compensação obrigatória em até 3 semanas. \\ \hline
\end{tabular}
\end{table}

\chapter{Glossário Técnico}

\begin{description}

\item[PLI — Programação Linear Inteira:]  
Modelo matemático onde todas as variáveis são inteiras, utilizado para problemas
de decisão e alocação.

\item[MIP — Mixed Integer Programming:]  
Problemas que combinam variáveis inteiras e contínuas, resolvidos via métodos
como Branch-and-Bound.

\item[MIP Gap:]  
Medida do quão distante a solução corrente está da solução ótima, calculada como:
\[
\text{Gap} = \frac{|Z^* - Z_{best}|}{Z^*}.
\]

\item[Presolve:]  
Conjunto de transformações aplicadas pelo solver antes da otimização, reduzindo
o tamanho do modelo.

\item[Cutting Planes (Cuts):]  
Restrições adicionais que eliminam regiões inviáveis da relaxação linear,
acelerando convergência.

\item[OR-Tools:]  
Pacote de otimização do Google usado para resolver problemas de rota,
scheduling e alocação.

\item[Window Time (Janela de Tempo):]  
Intervalo permitido para início ou término de uma tarefa.

\item[Feasibility Pump:]  
Heurística para encontrar soluções inteiras rapidamente em MIP.

\chapter{Fluxo Geral da Aplicação}

Este anexo apresenta o fluxo geral da aplicação desenvolvida para implementação do modelo de Programação Linear Inteira aplicado ao escalonamento de motoristas sob o Regulamento (CE) n.{\textordmasculine} 561/2006.

O diagrama foi construído utilizando a linguagem PlantUML e ilustra, de forma abstrata e didática, todas as etapas principais da ferramenta: configuração dos parâmetros via interface, construção programática das variáveis e restrições, execução do solver CP-SAT e geração dos resultados analíticos. 

Esse nível de detalhamento permite compreender a arquitetura e o funcionamento da solução sem expor o código-fonte propriamente dito, que é de natureza confidencial e constitui segredo comercial da organização.

A Figura~\ref{fig:plantuml-fluxo} apresenta a visão geral do fluxo da aplicação.

\begin{figure}[h!]
\centering
\includegraphics[width=0.95\textwidth]{template/figuras/sequence.png}
\caption{Fluxo geral da aplicação desenvolvido em PlantUML.}
\label{fig:plantuml-fluxo}
\end{figure}

\end{description}

\end{anexosenv}
