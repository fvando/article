% \chapter{Modelo Matemático}\label{capitulo5}

% \section{Introdução}

% Este capítulo apresenta o modelo de Programa{\c c}{\~a}o Linear Inteira (PLI) desenvolvido para o escalonamento de motoristas sob o Regulamento (CE) n.{\textordmasculine} 561/2006. O objetivo do modelo é gerar escalas legalmente válidas, eficientes e capazes de atender à demanda operacional por período, incorporando todas as restrições definidas pela legislação europeia.

% O modelo construído estende e aprofunda a formulação proposta em \cite{moreira2025}, incorporando maior detalhamento temporal, variáveis auxiliares e mecanismos de controle lógico adequados para implementação no solver CP-SAT do OR-Tools.

% O modelo foi concebido de forma flexível, permitindo a adoção de diferentes funções objetivo conforme o cenário operacional analisado. Em particular, duas estratégias de otimização são consideradas no simulador desenvolvido: (i) a maximização da resposta à demanda, priorizando a cobertura operacional, e (ii) a minimização do número total de motoristas alocados, priorizando eficiência de recursos. Ambas as abordagens compartilham o mesmo conjunto de variáveis e restrições legais, diferenciando-se apenas na função objetivo e no tratamento da restrição de atendimento da demanda.

% \section{Discretização Temporal}

% O horizonte de planejamento é dividido em um conjunto de períodos discretos:
% \[
% T = \{1,2,\ldots, |T|\},
% \]
% onde cada período representa uma janela fixa de tempo com duração:
% \[
% \Delta \ \ (\text{em horas}).
% \]

% Essa discretização permite rastrear a condução acumulada, pausas, descansos e demais restrições temporais do modelo.

% \section{Conjuntos e Índices}

% \begin{itemize}
%     \item \(D\): conjunto de motoristas (\(d \in D\));
%     \item \(T\): conjunto de períodos discretizados (\(t \in T\));
%     \item \(W\): janelas agregadas de planejamento:
%     \begin{itemize}
%         \item dias,
%         \item semanas,
%         \item quinzenas;
%     \end{itemize}
%     \item \(P\): parâmetros de demanda por período.
% \end{itemize}

% \section{Parâmetros do Modelo}

% \begin{align*}
% demanda_t     &: \text{demanda mínima de motoristas no período } t, \\
% \Delta               &: \text{duração de cada período (em horas)}, \\
% L^{\text{dia}}       &= 9 \quad \text{(limite diário de condução)}, \\
% L^{\text{ext}}       &= 10 \quad \text{(extensão diária permitida 2 vezes/semana)}, \\
% L^{\text{sem}}       &= 56 \quad \text{(limite semanal)}, \\
% L^{14d}              &= 90 \quad \text{(limite em 14 dias)}, \\
% R_{\text{dia}}       &= 11 \quad \text{(descanso diário normal)}, \\
% R_{\text{dia-red}}   &= 9 \quad \text{(descanso diário reduzido)}, \\
% R_{\text{sem}}       &= 45 \quad \text{(descanso semanal)}, \\
% P_{\text{pausa}}     &= 4.5 \quad \text{(máximo contínuo de condução antes da pausa)}.
% \end{align*}

% Esses parâmetros refletem fielmente as exigências legais discutidas em \cite{eu5612006}, \cite{eu200215}.

% \section{Variáveis de Decisão}

% \subsection*{Variável Principal de Alocação}
% \[
% x_{d,t} =
% \begin{cases}
% 1, & \text{se o motorista } d \text{ trabalha no período } t, \\
% 0, & \text{caso contrário}.
% \end{cases}
% \]

% \subsection*{Variável de Descanso}
% \[
% r_{d,t} =
% \begin{cases}
% 1, & \text{se o motorista } d \text{ descansa no período } t, \\
% 0, & \text{caso contrário}.
% \end{cases}
% \]

% \subsection*{Variável de Início de Jornada}
% \[
% s_{d,t} =
% \begin{cases}
% 1, & \text{se a jornada de } d \text{ inicia em } t, \\
% 0, & \text{caso contrário}.
% \end{cases}
% \]

% \subsection*{Variável de Condução Acumulada}
% \[
% h_{d,t} \ge 0.
% \]

% \subsection*{Variável de Extensão Diária}
% \[
% z_{d,w} =
% \begin{cases}
% 1, & \text{se o motorista utiliza extensão diária na semana } w, \\
% 0, & \text{caso contrário}.
% \end{cases}
% \]

% \section{Funções-Objetivo do Modelo}

% O modelo matemático proposto admite duas funções-objetivo alternativas,
% selecionadas de acordo com o objetivo operacional definido no simulador.
% Ambas compartilham o mesmo conjunto de variáveis e restrições legais,
% diferindo apenas no critério de otimização adotado.

% A variável \(u_t\) utilizada no modo de maximização da resposta à demanda
% é diretamente explorada nos indicadores apresentados no Capítulo~6,
% em especial nos índices de subutilização, cobertura e risco operacional.
% Dessa forma, o modelo matemático estabelece uma ligação direta entre
% a formulação teórica e os mecanismos analíticos do simulador.

% \subsection{Maximização da Resposta à Demanda}

% Neste modo, o objetivo do modelo é maximizar o atendimento da demanda
% operacional ao longo do horizonte de planejamento, penalizando períodos
% não cobertos.

% Define-se a variável auxiliar:
% \[
% u_t \ge 0, \quad u_t \in \mathbb{R} \quad \text{(demanda não atendida no período } t).
% \]

% Sujeita à restrição:
% \[
% \sum_{d \in D} x_{d,t} + u_t = demanda_t, \quad \forall t \in T.
% \]

% A função-objetivo é então dada por:
% \[
% \max Z_1 = \sum_{t \in T} (demanda_t - u_t),
% \]
% ou, de forma equivalente,
% \[
% \min \sum_{t \in T} u_t.
% \]

% Esse critério prioriza a maximização da cobertura, sendo especialmente
% adequado para cenários de alta variabilidade ou restrição severa de recursos.

% \subsection{Minimização do Número Total de Motoristas}

% Neste modo, o objetivo consiste em minimizar o uso total de motoristas
% ao longo do horizonte de planejamento, assumindo que toda a demanda
% deve ser integralmente atendida.

% A função-objetivo é definida como:
% \[
% \min Z_2 = \sum_{d \in D} \sum_{t \in T} x_{d,t}.
% \]

% Esse critério busca eficiência operacional e redução de custos de mão de obra,
% sendo apropriado para cenários estáveis ou de planejamento estratégico.

% Observa-se que a minimização do somatório \(\sum_{d,t} x_{d,t}\)
% corresponde à redução do volume total de trabalho alocado.
% Caso seja desejado minimizar explicitamente o número de motoristas distintos,
% pode-se introduzir uma variável binária \(u_d\) indicando se o motorista \(d\)
% é utilizado no horizonte, com a restrição \(x_{d,t} \le u_d\).
% No simulador desenvolvido, optou-se pela formulação apresentada,
% por refletir diretamente o custo operacional associado à utilização efetiva da mão de obra.

% \subsection{Observação sobre a Seleção do Critério}

% No simulador desenvolvido, o usuário pode selecionar dinamicamente
% qual função-objetivo será utilizada em cada experimento.
% Essa escolha não altera a estrutura do modelo, apenas o critério de otimização,
% permitindo análises comparativas entre estratégias focadas em cobertura
% ou eficiência operacional.

% \section{Restrições do Modelo}
% \subsection{Atendimento da Demanda}

% O tratamento da demanda depende do modo de otimização selecionado:

% \begin{itemize}
%     \item No modo de minimização de motoristas, a demanda constitui uma restrição rígida;
%     \item No modo de maximização da resposta à demanda, admite-se demanda não atendida por meio de variáveis auxiliares.
% \end{itemize}

% Essa formulação permite analisar compromissos entre cobertura operacional e disponibilidade de recursos, sem comprometer a conformidade legal.

% \subsection{Exclusividade de Estado por Período}
% Cada motorista deve estar em trabalho ou descanso:
% \[
% x_{d,t} + r_{d,t} = 1, \quad \forall d,t.
% \]

% \subsection{Acúmulo de Condução}
% \[
% h_{d,t} \le h_{d,t-1} + \Delta x_{d,t}.
% \]

% A dinâmica de reinício do contador de condução após períodos de descanso
% é implementada por meio de inequações com Big-M,
% evitando a introdução de igualdades não lineares e mantendo a compatibilidade
% com o solver CP-SAT.

% Reinício após descanso (linearização com Big-M):
% \[
% h_{d,t} \le M (1 - r_{d,t}).
% \]

% \subsection{Limite Diário de Condução}
% Para cada motorista e cada dia:
% \[
% \sum_{t \in w_{\text{dia}}} \Delta x_{d,t} \le L^{\text{dia}} + z_{d,w}.
% \]

% \subsection{Limite Semanal e Quinzenal}
% \[
% \sum_{t \in w_{\text{sem}}} \Delta x_{d,t} \le L^{\text{sem}},
% \]
% \[
% \sum_{t \in w_{14d}} \Delta x_{d,t} \le L^{14d}.
% \]

% \subsection{Pausa Obrigatória após 4,5 horas}
% \[
% h_{d,t} \le P_{\text{pausa}} + M r_{d,t}.
% \]

% \subsection{Descanso Diário Normal ou Reduzido}

% Para cada janela deslizante de 24 horas:
% \[
% \sum_{t \in \text{janela}_{24h}} \Delta r_{d,t} \ge R_{\text{dia}},
% \]
% ou, caso reduzido:
% \[
% \sum_{t \in \text{janela}_{24h}} \Delta r_{d,t} \ge R_{\text{dia-red}}.
% \]

% \subsection{Descanso Semanal}
% \[
% \sum_{t \in w_{\text{sem}}} \Delta r_{d,t} \ge R_{\text{sem}}.
% \]

% \subsection{Limite de Extensões Diárias}
% \[
% \sum_{w \in \text{semana}} z_{d,w} \le 2.
% \]

% \subsection{Continuidade da Jornada}
% Evita alternância inválida entre estados:
% \[
% x_{d,t-1} - x_{d,t} \le r_{d,t}.
% \]

% \section{Discussão sobre Linearização}

% O Regulamento (CE) n.{\textordmasculine} 561/2006 contém várias dependências temporais não-lineares, especialmente em regras de:

% \begin{itemize}
%     \item descanso diário e semanal,
%     \item condução acumulada,
%     \item pausas após condução contínua,
%     \item limites quinzenais,
%     \item inícios de jornada.
% \end{itemize}

% Para permitir resolução via PLI, foi necessário:

% \begin{itemize}
%     \item discretizar o horizonte em períodos fixos;
%     \item introduzir variáveis auxiliares (\(h_{d,t}, r_{d,t}, s_{d,t}, z_{d,w}\));
%     \item usar técnicas de linearização com Big-M;
%     \item incorporar janelas móveis de 24h, 7 dias e 14 dias.
% \end{itemize}

% Esses cuidados tornam o modelo compatível com solvers inteiros como CP-SAT, que se aproveita de sua estrutura esparsa e temporal \cite{googleORTools}.

% \section{Observações Computacionais}

% O modelo completo apresenta:

% \begin{itemize}
%     \item dezenas de milhares de variáveis binárias;
%     \item milhares de restrições lineares;
%     \item matriz esparsa com estrutura quase diagonal;
%     \item encadeamento temporal forte.
% \end{itemize}

% Ainda assim, o CP-SAT resolve cen{\'a}rios de 7 a 30 dias em poucos segundos, conforme validado no Capítulo~6, o que confirma sua aplicabilidade prática e alinhamento com estudos que aplicam otimização a problemas temporais complexos \cite{pillac2013,erdman2022}.

% \section{Considerações Finais}

% O modelo matemático apresentado formaliza rigorosamente o escalonamento de motoristas sob a legislação europeia. Com isso, fornece a base científica necessária para a implementação computacional descrita no capítulo seguinte e para as validações empíricas realizadas posteriormente.

% Embora o modelo seja resolvido de forma exata, sua estrutura também permite a integração com heurísticas e métodos de aprendizado supervisionado, utilizados posteriormente para geração de soluções iniciais e exploração de vizinhanças, sem alterar a formulação matemática.

\chapter{Modelo Matemático}\label{capitulo5}

\section{Introdução}

Este capítulo apresenta o modelo de Programação Linear Inteira (PLI)
desenvolvido para o problema de escalonamento de motoristas no transporte
rodoviário sob a regulamentação europeia, em particular o Regulamento
(CE) n.º 561/2006.
O objetivo do modelo é gerar escalas legalmente válidas, eficientes e
capazes de atender à demanda operacional, respeitando simultaneamente
limites de condução, pausas obrigatórias e períodos mínimos de descanso.

Formulações baseadas em PLI têm sido amplamente utilizadas para problemas
de escalonamento com fortes dependências temporais, sobretudo em contextos
regulados como transporte, saúde e serviços críticos
\cite{savelsbergh1997,pillac2013}.
Neste trabalho, o modelo proposto estende a formulação apresentada em
\cite{moreira2025}, incorporando maior granularidade temporal, variáveis
auxiliares e uma estrutura de otimização compatível com o solver CP-SAT
do OR-Tools \cite{googleORTools}.

Diferentemente de abordagens puramente multiobjetivo, o modelo adota uma
estrutura lexicográfica em duas fases, refletindo prioridades operacionais
reais: primeiro garantir cobertura da demanda e minimizar o número de
motoristas ativos, e somente depois refinar a solução por critérios de
balanceamento e suavização de carga.

\section{Discretização Temporal}

O horizonte de planejamento é discretizado em períodos fixos de duração
$\Delta = 15$ minutos, formando o conjunto:
\[
T = \{1,2,\ldots,|T|\}.
\]

Essa discretização reflete a frequência real de chegada dos pedidos de
carga na operação analisada e permite representar de forma linear
restrições originalmente não lineares, como janelas móveis de descanso
e limites acumulados de condução \cite{pillac2013,erdman2022}.

\section{Conjuntos e Índices}

\begin{itemize}
    \item $D$: conjunto de períodos discretizados ($d \in D$);
    \item $T$: conjunto de motoristas ($t \in T$);
    \item $W$: janelas agregadas de planejamento (dias, semanas e quinzenas).
\end{itemize}

\section{Parâmetros}

\begin{align*}
demanda_d      &: \text{demanda operacional no período } d,\\
\Delta         &: \text{duração do período (15 minutos)},\\
L^{dia}        &: \text{limite diário de condução},\\
L^{sem}        &: \text{limite semanal de condução},\\
L^{14d}        &: \text{limite quinzenal de condução},\\
R^{dia}        &: \text{descanso diário mínimo},\\
R^{sem}        &: \text{descanso semanal mínimo}.
\end{align*}

Esses parâmetros refletem diretamente as exigências do Regulamento
(CE) n.º 561/2006 e da Diretiva 2002/15/CE \cite{eu5612006,eu200215}.

\section{Variáveis de Decisão}

\subsection*{Ativação do motorista}
\[
Z_t =
\begin{cases}
1, & \text{se o motorista } t \text{ é ativado},\\
0, & \text{caso contrário}.
\end{cases}
\]

\subsection*{Presença no período}
\[
Y_{d,t} =
\begin{cases}
1, & \text{se o motorista } t \text{ está presente no período } d,\\
0, & \text{caso contrário}.
\end{cases}
\]

\subsection*{Atendimento de demanda}
\[
X_{d,t} \in \mathbb{Z}_{+}, \quad 0 \le X_{d,t} \le \text{capacidade por slot}.
\]

\subsection*{Demanda não atendida}
\[
U_d \ge 0.
\]

\subsection*{Carga por motorista e carga máxima}
\[
L_t = \sum_{d \in D} X_{d,t}, \qquad L_{\max} \ge L_t.
\]

A separação entre variáveis de ativação, presença e carga segue práticas
consolidadas na literatura de escalonamento, permitindo representar de
forma explícita decisões operacionais e restrições legais
\cite{savelsbergh1997,pinheiro2007staffscheduling}.

\section{Função Objetivo Lexicográfica}

O modelo utiliza uma estrutura de otimização em duas fases, resolvidas
sequencialmente pelo solver.

\subsection{Fase 1 — Cobertura e Minimização de Motoristas}

A Fase 1 possui prioridade absoluta e busca:

\begin{enumerate}
    \item minimizar a demanda não atendida;
    \item minimizar o número de motoristas ativados;
    \item incentivar maior utilização dos motoristas ativos.
\end{enumerate}

A função objetivo é dada por:
\[
\min \;
\alpha \sum_{d \in D} U_d
+ \beta \sum_{t \in T} Z_t
- \gamma \sum_{d \in D} \sum_{t \in T} X_{d,t},
\]
com $\alpha \gg \beta \gg \gamma$, garantindo prioridade lexicográfica.

Estruturas desse tipo são amplamente utilizadas quando múltiplos objetivos
conflitantes precisam ser tratados de forma hierárquica
\cite{pillac2013,erdman2022}.

\subsection{Fase 2 — Balanceamento e Refinamento}

Após fixar os valores ótimos de $\sum U_d$ e $\sum Z_t$, a Fase 2 refina
a solução, buscando:

\begin{itemize}
    \item reduzir presença desnecessária;
    \item penalizar excesso de carga acima de um limite suave;
    \item suavizar picos de carga entre motoristas.
\end{itemize}

A função objetivo da Fase 2 é:
\[
\min \;
\omega_1 \sum_{d,t} Y_{d,t}
+ \omega_2 \sum_{t} \max(0, L_t - L^{soft})
+ \omega_3 L_{\max}.
\]

Esses critérios promovem maior equidade e robustez operacional, sendo
frequentemente empregados como etapa de refinamento em problemas de
escalonamento \cite{savelsbergh1997}.

\section{Restrições do Modelo}

O modelo matemático incorpora um conjunto abrangente de restrições
legais e operacionais derivadas do Regulamento (CE) n.º 561/2006.
Entretanto, nem todas as restrições são necessariamente ativadas
simultaneamente em todos os experimentos.

No simulador desenvolvido, cada restrição pode ser ativada ou desativada
de forma independente por meio de parâmetros de configuração,
permitindo a análise controlada do impacto individual e combinado das
regras legais sobre a solução.
Essa abordagem possibilita estudos comparativos entre diferentes níveis
de rigor regulatório, sem alterar a estrutura fundamental do modelo.

Formalmente, o conjunto de restrições ativas em um experimento é definido
por um vetor binário de ativação, sendo que apenas as restrições
selecionadas são incluídas no modelo resolvido pelo solver.

% \subsection{Atendimento da Demanda}
% \[
% \sum_{t \in T} X_{d,t} + U_d = demanda_d, \quad \forall d \in D.
% \]

\subsection{Atendimento da Demanda}

Quando a restrição de cobertura obrigatória está ativada, o atendimento
da demanda é modelado como uma restrição rígida:

\[
\sum_{t \in T} X_{d,t} + U_d = demanda_d, \quad \forall d \in D.
\]

Caso essa restrição não seja ativada, a variável $U_d$ passa a representar
déficits operacionais admissíveis, utilizados exclusivamente para fins
de avaliação e cálculo de indicadores, sem inviabilizar a solução.


\subsection{Vinculação entre presença e ativação}
\[
Y_{d,t} \le Z_t, \quad \forall d,t.
\]

\subsection{Capacidade por período}
\[
X_{d,t} \le \text{capacidade} \cdot Y_{d,t}.
\]

\subsection{Limites legais de condução}

As restrições apresentadas a seguir são incluídas no modelo apenas quando
explicitamente ativadas no cenário experimental considerado.

Restrições diárias, semanais e quinzenais são modeladas por janelas móveis,
conforme exigido pelo Regulamento (CE) n.º 561/2006.

\subsection{Pausas e descansos}
Pausas após 4,5 horas de condução contínua e descansos diários e semanais
são representados por restrições lineares baseadas em janelas deslizantes.

\section{Discussão sobre Linearização}

As regras legais apresentam dependências temporais não lineares, tratadas
por meio de discretização, variáveis auxiliares e constantes Big-M.
Essa abordagem é amplamente utilizada para manter a linearidade da
formulação e compatibilidade com solvers inteiros modernos
\cite{pillac2013,googleORTools}.

\section{Observações Computacionais}

O modelo apresenta dezenas de milhares de variáveis e restrições, mas
possui matriz esparsa e forte estrutura temporal. Solvers modernos, como
o CP-SAT, exploram essas características de forma eficiente, permitindo
resolver cenários reais em tempos compatíveis com uso operacional
\cite{erdman2022,googleORTools}.

\section{Considerações Finais}

O modelo matemático apresentado fornece uma base rigorosa e aderente à
legislação europeia para o escalonamento de motoristas. Sua formulação
lexicográfica reflete prioridades operacionais reais e viabiliza a
integração com heurísticas e métodos matheurísticos discutidos nos
capítulos seguintes, sem comprometer a consistência matemática.
